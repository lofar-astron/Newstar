% doc_release_9511.tex
%
%
\chapter{ Changes to the \NEWSTAR documentation system, November 1995}
{\center\it by Johan Hamaker, 951012}

\tableofcontents

\section{ New contents page}

	A new contents page, hb\_contents.tex, replaces the old one that was
generated by ndoc overview. This latter command is no longer valid and the
files \$n\_doc/hb\_contents.tbl and \$n\_doc/hb\_contents.tex must be deleted.

	The new contents page is hand-coded, which leads to a much more compact
format with a very much higher information density.


\subsection{ Up-to-date versus obsolescent documents}

	In the contents page, an attempt has been made to visually distinguish
references to obsolescent documents from those to up-to-date ones, the latter
being shown in boldface.



\section{ Maintaining PostScript copies of all documents}

	It was found that ndoc print does not work in Westerbork because it
relies on a system environment. This was to be expected because the official
\NEWSTAR policy is that ndoc needs to work only in Dwingeloo.

	To give remote users access to printed copies, the .ps versions of all
documents are now stored in \$n\_hlp and should be exported from there.


\subsection{ New extension .fps for PostScript diagrams}

	ndoc fig produces PostScript versions of diagram graphics (.fig files);
these are then integrated into LaTeX documents through ,,fig commands in .cap
files that are themselves ,,input'ed by .tex files. To distinguish these
PostScript files from printable documents, the former carry the extension .fps
rather than .ps.



\section{ Elimination of diagrams as independent documents}

	Until now copies of many diagrams have existed as independent .html
documents that could be referenced directly. This is a reasonable idea but
leads to problems when a caption refers to another diagram: If that diagram is
in the same document, the reference must be to a label, but if the diagram is
in a stand-alone diagram file, the reference must be to that file.

	The solution is to eliminate the standalone hypertext diagrams. Instead
of referring to the standalone document, one refers to the diagram in a text
file in which it is embedded.



\section{ Referencing of Fortran and other source documents}

	It was found desirable to refer to files in the \$n\_src tree: Examples
are the \$n\_src/doc/txt/*.txt ASCII documents and the .dsc files that define
data structures. The obvious way to do this is through soft links, but these do
not work between the WWW server and the \$n\_src disk.

	The only solution is to make copies of the referenced documents on the
WWW server. For this purpose, a directory \$n\_hlp/src must be created with a
shadow directory tree containing copies of only those files that are referenced
(see ndoc options below)..

	A new procedure, doc\_copy.csh, has been created for this purpose. It
is driven by an internal table showing which files must be copied.


\section{ ndoc options }

	ndoc overview has been phased out.

	ndoc test will carry out a series of consistency checks between the
document source tree and the \$n\_hlp tree, eliminating files in the latter for
which no corresponding source exists and reporting anomalies. It also

	ndoc all compiles all program-interface descriptions, all LaTeX
documents and performs an ndoc test.

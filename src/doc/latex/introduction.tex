% introduction.tex
%
% JPH   941124
% JPH   941221  Expand processing paradigm: First order calibration from
%               calibrator, later refinement through Selfcal. Add UPDATE.
%
\newcommand{\bi}{ \begin{itemize} }
\newcommand{\ei}{ \end{itemize} }
\newcommand{\bn}{ \begin{enumerate} }
\newcommand{\en}{ \end{enumerate} }
\newcommand{\eg}{e.g.}

\chapter{ Introduction to \NEWSTAR for new and prospective users } {\em
Contributed by Johan Hamaker, november 1994 \centering \par}

\tableofcontents


\section{ Preface }
\label{.preface}

	\NEWSTAR is an acronym for the '{\bf N}etherlands {\bf E}ast-{\bf W}est
{\bf S}ynthesis {\bf T}elescope {\bf A}nalysis and {\bf R}eduction'. It stands
for a small collection of programs whose purpose is to process 'raw' visibility
data from the WSRT into sky maps, source lists and the like that are ready for
astronomical interpretation.

	Its functionality allows one to fully exploit the strong points of
East-West systhesis arrays in general and the WSRT in particular: Wide-field
mapping, and high dynamic range both in 'total-power' and polarisation
observations. To achieve this, \NEWSTAR contains a number of powerful and
refined algorithms not known to exist in other systems. Their presence makes
\NEWSTAR quite overwhelming for the uninitiated. This document is intended to
provide a simple introduction of a tutorial type.

	The documentation generally refers to the WSRT as the source of
observations. \NEWSTAR also includes provisions for the processing of data from
the Australian Telecope (ATNF) compact array (ATCA), but these are few and
simple and we skip them here.


\subsection{ Portability }
\label{.portability}

	\NEWSTAR's primary maintenance is carried out by NFRA on Sun and HP
systems. It is actively supported elsewhere for a DECstation and a DEC Alpha;
it has been supported in the past for a Convex and is therefore likely to be
readily portable to there. VAX support has been discontinued.

	For general operations, such as moving or deleting files, manipulating
text files, {\em etc.} you must rely on your host system. Inasmuch as \NEWSTAR
now only runs in Unix environments, a generic knowledge of that environment
should be sufficient, but also necessary.


\subsection{ Terminology}
\label{.terminology}

	A perceptive user may well notice that \NEWSTAR programs are rather
careless in their use of terminology. This is a heritage that we are to some
extent stuck with: Automatic 'batch' procedures in several on-going large WSRT
projects ({\em WENSS} and {\em WHISP}) rely on the present structure and
parameter names of the user interface. We try to clarify ambiguities in the
documentation and on-line Help, but the user must be wary that names do not
always mean what they seem to suggest.


\section{ NEWSTAR's basic processing paradigm }
\label{.basic}

\input {basic_functions.cap}

\subsection{ Source and instrumental-error modeling }
\label{.source.error.modeling}

	The most common aim of processing visibility data is to obtain a
representation of the sources observed, with artefacts from the observing
process adequately removed. Such artefacts arise both from the finite sampling
of the data (primary and synthesised antenna beams) and from instrumental
errors.

	If errors were absent, the problem would be 'simply' to deconvolve the
known antenna pattern (synthesised beam) from the observed data. This is an
ambiguous process: There is an infinite number of possible solutions.
Acceptable ones are favoured by the {\em a priori} knowledge incorporated in
radio-astronomy algorithms; in addition, human judgment is needed to control
the reduction process.

	Instrumental errors complicate the problem. To first order they may be
determined from {\em calibrator} sources observed before and/or after the
source. This typically yields a {\em dynamic range} in the order of one in a
few hundred.

	Much higher dynamic ranges can only be obtained through {\em
self-calibration}. This is a generic name for processes in which a source model
and a model of the instrumental errors are constructed simultaneously.
Self-calibration relies on the fact that the signature of true sources in an
observation is quite different from that imposed by instrumental errors.
Algorithms have been invented that discriminate between the two quite well. The
separation is not perfect, however, and even more than for source modeling
alone, human judgment must keep the process on course.


\subsection{ Outline of the reduction process in \NEWSTAR }
\label{.outline}

	\Figref{.basic.functions} shows the fundamental functions and the
primary data files with their most important contents. The connecting arrows
show the data flow in the basic reduction process. We enumerate the steps here
in order for a typical reduction process. In practice, the complexity of the
observed field and/or the quality of the observation may give rise to a wide
variety of situations; the same steps or only part of them may have to be
combined in a different sequence. If you need really high dynamic range, take
your time to become an expert...

	The basic reduction steps shown combined in \figref{.basic.functions}
are:
\bn
%1
\item   Visibilities are read in from a data archive by NSCAN; calibrator
observation(s) are included.
%2
\item   The calibrator data are processed in NCALIB to find the instrumental
errors. A source model is used as a reference; such models are available in
\NEWSTAR for all calibrators routinely used by the WSRT. The errors found are
stored as {\em corrections} in the calibrator observation.
%3
\item   NCALIB is invoked to copy the corrections from the calibrator to the
astronomical observation. This provides the 'first-order' error correction
referred to \textref{above}{.source.error.modeling}.
%4
\item   A map is made from the astronomical observation, using the corrections.
At this point, reduction may be complete. If the quality of the map is
inadequate, one must continue.
%5
\item   NMODEL FIND and NCLEAN are used to construct a {\em source model} for
the astronomical field; such a model is a list of components, each of which can
be arbitrarily positioned and have a finite extent. ({\em I.e.}, they are much
more general than conventional CLEAN components.)
%6
\item   The positions and other parameters defining the source components may
be improved through an NMODEL UPDATE. This procedure is unique to \NEWSTAR. It
refines the parameters estimated by NMODEL FIND for each source component by
comparing its computed visibilities against the observed ones.
%7
\item   A 'self-calibration' {\em selfcal} can now be done with NCALIB in the
same way as was done for the calibrator in step 2 above. This will refine the
correction parameters stored in the .SCN file.
%8
\item   NFLAG may be used to flag as 'bad' those scans for which the Selfcal
operation produced a poor result (either an error message or an excessive mean
error in the solution).
%9
\item   Form this point on, steps 3-8 can be repeated {\em ad libitum}.
\en

	With a general picture of the reduction process in mind, we can take a
look first at the data files and then at the programs involved.


\section{ The \NEWSTAR data files }

	We first discuss the three types of data files shown in
\figref{.basic.functions}. After that, we consider the common {\em indexing}
system that is common to most of these files.

\subsection{ The scan (.SCN) file }
\label{.scn.file}

\input{ scn_summary.tef}


\subsubsection{ Error categories and the Selfcal paradigm }
\label{.selfcal}

\input {error_model.cap}

	Errors, - and consequently corrections for them -, can be broken down
in four categories, as shown schematically in \figref{.error.model}:

\bi
\item   {\em Global:} Errors represented by a single parameter for the
instrument as a whole: \eg a clock error or refraction.

\item   {\em Telescope(-based):} Errors that occur {\em per telescope} and
therefore affect all interferometers including that telescope in the same way:
\eg tropospheric cloud effects and telescope position errors.

\item   {\em IF-based:} Errors that occur {\em per dipole/polarisation channel}
 and therefore affect all interferometers including that IF channel in the same
way: \eg elctronic gain and phase errors.

\item   {\em Interferometer(-based):} Errors that occur in the correlator and
can therefore not be attributed to a telescope/polarisation channel: \eg
correlator zero offsets. In a well-designed correlator these errors can be very
small and for the WSRT one can almost always ignore them.
\ei

	In the somewhat inaccurate terminology of \NEWSTAR no dictinction is
made between telescope- and IF-based errors; the two are jointly known as {\em
telescope errors} as opposed to {\em interferometer errors}. The distinction
between the two types of error is fundamental to the {\em Selfcal} methods
which are used to separate instrumental errors from the true source
visibilities. A basic understanding of the Selfcal paradigm is assumed
throughout \NEWSTAR.

	In addition to the Selfcal constraint, WSRT data are subject to the
{\em Redundancy} constraint: Baselines of the same length (and the same
orientation as they always have for the WSRT) must produce the same
visibilities.


\subsection{ The map (.WMP) file }
\label{.wmp.file}

	.WMP files are the containers for two-dimensional {\em images}: sky
maps and antenna patterns. Each immage is an array of data points with
horizontal and vertical positions {\em l, m} as coordinates. It represents a
projection of the celestial sphere on a plane tangent in the field centre; the
direction of the projection is {\em parallel to the polar axis}.

	The organisation of the images in a .WMP file will be discussed
\textref{shortly}{.indices}.


\subsection{ The source-model (.MDL) file }
\label{.mdl.file}

	Unlike the .SCN and .WMP files that may contain a collection of sectors
or images, the .MDL file harbours only a single {\em source model}. The concept
of such a model is familiar from {\em CLEAN} as a list of point sources with a
position and flux for each.

	The \NEWSTAR source mode is more refined: It may also contain for each
source an (elliptical-Gaussian) extent, its 4 Stokes parameters, its spectral
index and its intrinsic rotation measure. Traditional CLEAN components are a
degenerate case of this more general concept.


\subsection{ Organisation of collections of data structures: Indices}
\label{.indices}
\label{.SCNSUM.indices}

	We have seen that \NEWSTAR data files are collections of data units:
Let us first consider the .SCN file, where these units are the
\textref{sectors}{.scn.file}. Each such unit is self-describing, but we need
additional information to understand its relation to other units, viz. what are
its channel number and field number (for a mosaic observation). Moreover it is
possible to store more than one observation in a scan file and we need a way of
knowing which is which.

	The organisation in \NEWSTAR data files is by means {\em hierarchical
indexing}: Each unit is addressed through a {\em composite index}, consisting
of a string of integers {\em 'indices'} separated by dots, e.g.

\verb/  3.1.0.5/

	For the .SCN file the indices are:

\verb/  <group>.<observation>.<field>.<channel>.<sequence\ nr/
where

\bi
\item   \verb/<group/ is a purely administrative index that allows one, \eg to
collect observations that are intrinsically unrelated in a single file,
\eg a scientific observation with the calibrator observations to be used with
it.

\item   \verb/<observation>/ is related to {\em labels} and {\em observations}
in which WSRT data tapes are structured.

\item   \verb/<field>/ is a subfield number in a mosaic observation.

\item   \verb/<channel>/ is a frequency-channel number (also known as band
number).

\item   \verb/<sequence number>/ is an extra index that allows for the
existence of multiple data structures having all preceding indices the same,
\ei

	For the .WMP file the indices are:

\verb/  <group>.<field>.<channel>.<polarisation>.<image type>.<sequence nr>/
%
where \verb/<group>/, \verb/<field>/ and \verb/<channel>/ are the same as for
the .SCN file; in addition we have

\bi
\item   \verb/<polarisation>/ represents either a WSRT dipole combination or a
Stokes parameter.

\item   \verb/<image type>/ indicates either a map or an antenna pattern.
\ei
	A number of mechanisms exist in the user interface that allow you to
select set(s) of data units, \eg 'all sectors', 'all channels for fields 3
through 7', etc.

	In closing, we note that the indexing structure is purely
administrative. The data units are entirely self-contained in terms of their
data contents and associated headers. A sector, for example, can be copied from
one .SCN file to another without any loss of the information necessary for its
interpretation.


\section{ The \NEWSTAR programs}
\label{.programs}

	\Figref{.basic.functions} includes only the most basic functions of the
most important programs. Each of the programs includes a considerable number of
other {\em functions}: \NEWSTAR is a small collection of large multifunctional
programs rather than a large collection of small programs (like AIPS or GIPSY).

	We will briefly mention the most interesting functions in each of the
programs, but ignore the {\em utility functions} which are part of most
programs. These will, \eg:

\bi
\item   list the contents of files;

\item   show headers and data blocks in detail and optionally edit them;

\item   convert files between number formats for different host systems;

\item   track version changes;

\item   write and/or read FITS tapes/files.

\ei


\subsection{ NSCAN}
\label{.nscan}

	NSCAN is the program that handles .SCN files. Apart from utilities, its
only operation is:

\bi
\item   LOADing data from an external medium (classical magtape, DAT or Exabyte
tape, optical disk) into a .SCN file.
\ei

It is capable to read WSRT data in all existing formats including the old
'Leiden format'.


\subsection{ NCALIB}
\label{.ncalib}

	NCALIB is the program that determines corrections from the visibilities
in a .SCN file, optionally using a \textref{source model}{.mdl.file} as a
reference. Its primary functions include:

\bi
\label{.redun}
\item   REDUN: This is the Selfcal function of NEWSTAR. It includes several
methods of estimating telescope errors from observed visibilities and
(optionally) a source model, using the \textref{Selfcal and
Redundancy}{.selfcal} constraints.

\item   POLAR: Determining deviations from the nominal dipole responses to
polarised radiation from calibrator observations.

\item   SET: Function for reading corection parameters from external sources,
such as

  \bi
  \item estimates of Faraday rotations calculated from ionosonde or
chirp-sounder data;

  \item clock corrections from the International Earth Rotation Service (IERS,
formerly the Bureau Internationale de l'Heure).
  \ei
\ei


\subsection{ NMAP}
\label{.nmap}

	NMAP is the program that handles .WMP files. Its primary functions
include:

\bi
\item   MAKE: Its first and foremost function, {\em viz.} to Fourier-transform
visibilities into sky maps with their assosiated antenna patterns.

\item   FIDDLE: A family of functions to modify maps or combine them into new
ones in several useful ways, \eg:

  \bi
  \item BEAM, DEBEAM: Apply/de-apply the primary-beam correction.

  \item POL, ANGLE: Convert a pair of Stokes Q and U maps into a pair
representing the polarisation direction and magnitude.

  \item MOSCOM: Merge adjacent overlapping maps (\eg mosaic subfields or
adjacent full mosaics) into a single large map.

\ei\ei


\subsection{ NFLAG}
\label{.nflag}

	NFLAG is the {\em only} program that changes flag settings. Its FLAG
function is composed of a large number of subfunctions that set different flags
on the basis of criteria that may be

\bi
\item   {\em deterministic,} i.e. comparing data's coordinate parameters
against fixed limits (\eg elevation, shadowing, short baselines);

\item   {\em data-derived,} i.e. comparing data values against a fixed limit
(\eg an upper limit for data considered to be free of interference);

\item   derived from {\em data statistics}, such as comparing the noise in a
\textref{selfcal}{.redun} solution against a fixed limit.
\ei

	NFLAG is still evolving. It is hoped that both NFRA and the \NEWSTAR
users will contribute new ideas and algorithms, to further the art of
eliminating faulty data with a minimum of efforts as well as minimal accidental
loss of healthy data.


\subsection{ NMODEL}
\label{.nmodel}

	NMODEL is the program that handles source-model (.MDL) files. Its most
vital function is FIND, {\em i.e.} finding source components of small extent
and fitting elliptical Gaussians to them. This operation differs from CLEAN in
two ways:

\bi
\item   Components are merely located; to subtract them, one must make a new
map as in the major cycles of Cotton-Schwab CLEAN.

\item   Sources may be at fractional grid positions and extended, which allows
for a much more accurate representation of sources of small size.
\ei

	NMODEL also harbours the \textref{UPDATE function}{.outline}, which
fine-tunes the components of a model by comparing the model visibilities with
the observed ones.


\subsection{ NCLEAN}
\label{.nclean}

	NCLEAN is \NEWSTAR's CLEAN program. It's purpose is to create source
models for thos source components that cannot be adequately represented by
elliptical Gaussian's. It implements only the H\"ogbom, Clark and Cotton-Schwab
CLEAN algorithms.


\subsection{ Other programs}
\label{.other.programs}

	\Figref{.basic.functions} shows only the most vital programs. There are
several others:

\bi
\item   NATNF loads data from the Australian Telescope (ATNF) into a .SCN file.

\item   NCOPY copies selected parts of a .SCN file into another .SCN file. (In
this process one may change visibility values irreversibly through the
application of selected corrections.)

\item   NPLOT can display or print a variety of plots as well as images in
various representations.

\item   NGIDS is an adaptation of {\em GIPSY}'s display program GIDS. It
primary function is to display images in various modes, but it can also be used
to select faulty data in a display of visibilities (the {\em TVFLAG} function
of AIPS).

\item   NGCALC is a program to analyse one-dimensional {\em cuts}, either
through the visibility \whichref{hypercube}{} or through the associated
hypercube of correction values. It can be used, \eg, to analyse the 'light
curve' of a variable source or to fit a polynomial to a cut of visibilities
against baseline.
\ei


\section{ Salient features of the user interface }
\label{.user.interface}


\subsection{ Controlling program runs }

	\NEWSTAR programs communicate with the user through a uniform parameter
interface. (Historically this subsystem is referred to as {\em DWARF}, but this
name now serves only to cause confusion.) We briefly mention the features one
normally uses in an interactive session.

\bi
\item   A program is started from the shell through the commands
  \bi
  \item[]       dwe \verb/<program name>/ or dwexe \verb/<program name>/
  \ei

\item   Except for those parameters that are intrinsically of an interactive
nature, a program collects all necessary parameter values before any real
processing and/or writing to data files is done.

\item   Programs can be aborted at any time through control-C. This is safe
{\em except when modification of data files is in progress.} Where the latter
is the case, one would risk corrupting those files.

\item   Parameters to be specified by the user are known by their name (also
known as the {\em keyword}.

\item   When a program needs a parameter value, it will prompt on the terminal
showing the keyword, a short text explaining what is being asked for, a list of
options where applicable, and the default value if one is available.

\item   The user's reply is immediately checked to the extent possible: For
numeric values a range of validity may have been defined, options must match
the list of valid names (unique abbreviations being allowed in most cases.

\item   When you reply with an EOF character (control-D in Unix) or a hash sign
(\#), the program will backtrack over one or several prompts, allowing you to
recover from mistakes without having to abort and restart the program.
\ei


\subsection{ On-line Help }
\label{.help}

	On-line help in a running program is available in two modes:
\bi
\item   The {\em dumb-terminal mode}: When you type a '?' in reply to a prompt,
the help text for the parameter will be shown on your terminal (window).

\item   You switch to {\em hypertext mode} by typing '??' in reply to a prompt;
this mode then remains in effect until the program terminates. Help texts are
displayed in the xmosaic window; the advantage is that you now have access to
associated documentation through hypertext links.
\ei



\subsection{ Log files }
\label{.log.files}

	Each program run produces a .LOG file in your current directory. These
files are named

	\verb/<prog><timestamp>A.LOG/,

where \verb/<prog>/ is the first three characters of the program name and
\verb/<timestamp>/ is a string of 12 digits (yymmddhhmmss). The newest log for
a program is also known under the name

	\verb/<program>.LOG/
\\
The log file contains a record of all the parameter values that were used in
the program run, a copy of all messages and other output that appeared on the
terminal and in some cases a lot more information.

	In many cases the best way to digest a log file is not to print it.
Instead, take a quick look at it on your terminal to decide what information is
most relevant; then use the Unix utility {\em grep} to extract it.


\section{ Disk-space management }

	\NEWSTAR is rather lavish in its use of disk space, - partly because
its functionality demands it, partly also because disk space is so cheap that
programming for space economy is not cost-effective.

	As an example, a .SCN file for a 12-hour observation with 87
interferometers, 8 frequency channels, 4 polarisations and a sampling interval
of 1 minute occupies 24 Mbytes; including a source model in all sectors almost
doubles this space. Yet this type of observation is relatively modest in its
demands; a mosaic observation easily runs into the hundreds of Mb.

	It is therefore important to avail yourself of sufficient disk space,
and you may also have to familiarise yourself with the methods for saving files
on DAT tape on your host machine.

	\NEWSTAR programs have been designed to crash gracefully when running
out of disk space. That is, a data file they were writing to will be left in an
incomplete but probably readable state. Yet it bis better not to rely on this,
as recovering from a crash may still be a time-consuming affair.



\section{ Interfacing with other astronomical reduction packages }
\label{.interfacing}

	The possibilities to exchange data between \NEWSTAR and 'the rest of
the world' are limited to the following:

\bi
\item   The sectors from a .SCN file can be exported in the UVFITS format that
AIPS can read. The inverse operation is {\em not} provided for.

\item   Images can be written and read in 16- and 32-bit integer as well as
32-bit floating-point FITS formats. Other packages are likely to interface to
at least one of these.

\item   Visibility data from ATNF/Miriad can be read.

\item   Interfaces between \NEWSTAR and AIPS++ will probably be developed as a
by-product of NFRA's active involvement in the latter project. The eventual
conversion policy will be determined on the basis of the user's needs.
\ei


\section{ Acknowledgements }

	\NEWSTAR's history dates back to the pioneering work on Redundancy by
J.E. Noordam in the early 1980s. W.N. Brouw systematised and expanded his work
in subsequent years, in a suite of programs known as the 'R-series'. In
1990-1991, this series served as the prototype for an entirely new collection
of programs that eventually became \NEWSTAR. During this entire development,
A.G.de Bruyn played an indispensable role as an active and stimulating user.

	Several users contributed to the documentation; they are acknowledged
in the individual Document chapters.

	Incorporated in \NEWSTAR are:

\bi
\item   The Dwingeloo-Westerbork Astronomical Reduction Facility ({\em DWARF})
user-parameter interface from NFRA.

\item   The Groningen Image Display Syetem ({\em GIDS}) display package from
Groningen's GIPSY system.

\item   The {\em PGPLOT} public-domain screen-graphics package.
\ei



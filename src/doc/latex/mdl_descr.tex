%
% @(#) mdl_descr.tex  v1.2 04/08/93 JEN
% HjV  950619	Correct some typo's
%
\chapter{The MDL-file (model components)}
\tableofcontents 

A \NEWSTAR MDL file contains a collection of source model components.
This may be a mixture of multi-parameter components and ordinary CLEAN
components. The former are much more versatile, and represent one of the
main differences between \NEWSTAR and other uv-data reduction packages.

%==============================================================================
%\include{fig_file1_MDL_structure}	% figure 
%==============================================================================

%==============================================================================
\section{Inspecting the list of model components}
\label{mdl.descr.example}


******* Insert new script here ******


The above model is in `local mode', which means that no reference position or
reference frequency is known. This information may be taken from a SCN-file
(use NMODEL option CONVERT), and stored with the model in the MDL file. The
model then looks like this: 


******* Insert new script here ******

\newpage
The coordinates may als be viewed in RA and DEC:


******* Insert new script here ******


Note that not all source parameters (see below) are visible here. They may
be inspected in the log-file whenever the model has been used.

%==============================================================================
\section{Source component parameters}
\label{mdl.descr.param}

The source components in a \NEWSTAR model have the following parameters:

\begin{itemize}
\item {\bf I:}		
total flux in W.U. (even when extended or polarised)
\item {\bf l, m:} 	
offset from the reference point (map centre) in arcsec
\item {\bf ID:}		
identification number (1,2,3,...) in the list.
\item {\bf Q, U, V:}	
Stokes Q, U, V in percent (of I)
\item {\bf long, short, PA:} 
long and short axes (full halfwidth in arcsec) of an elliptic gaussian extended
source, and the position angle of the long axis (degrees North thru East).
\item {\bf si:}		
spectral index ($flux \div frequ^{si}$).
Only used if reference frequency is known.
\item {\bf rm:}		
Faraday rotation measure ($rot~=~ rm \times (c/frequ^{2}) ~in~rad/m^{2}$).
Only used if reference frequency is known.
\item {\bf flags:} 
(e.g. a `proper' source (0) or a CLEAN component (1)) 
\end{itemize}


NB: CLEAN components:
~\\ - Are not corrected for beam smearing in SELFCAL.
~\\ - Are confined to map grid postions, which may result in poor subtraction.
~\\ - Their positions cannot be "updated" automatically after SELFCAL.
~\\ Therefore, use ``proper sources'' for strong point sources, 
and CLEAN components for weak extended sources (but only if necessary).




%==============================================================================
\newpage
\section{Overview of model `handling' options.}
\label{mdl.descr.handle}

In various \NEWSTAR programs (NMODEL, NCALIB, NMAP etc) the user is able to 
{\bf handle} the list of model components in various ways. 
The keyword often used is {\bf MODEL\_OPTION}:

\begin{itemize}
\item {\bf READ/WRITE:}		Read/write from/to an external MDL file
\item {\bf CLEAR:}		Clear the source component list, 
				while resetting reference coordinates
\item {\bf ZERO:}		Empty the source component list, 
				but keep the coordinates of the field centre
				and frequency.
\item {\bf SHOW/LIST:}		Show source list on terminal screen, 
				or both terminal and LOG-file.
				See example above.
\item {\bf RSHOW/RLIST:}	Show source list in RA/DEC coordinates
				See example above.
\item {\bf TOT:}		Show source list statistics (for a specified
				range of sources). The result looks like:
\sline{Sources at epoch 1950 at 05:38:43.51, 49.49.42.8, 1417.248 MHz}
\sline{10 sources (0 deleted) with 4453.700 W.U. (Max= 1900.000, Min= 2.200)}\\
\item {\bf ADD:}		Add sources to the list by hand.
\item {\bf CALIB:}		Convert the source list by scaling intensities
				and/or moving l,m positions.
\item {\bf EDIT:}		Edit the sources in the list 
				(an amplitude of zero will delete the source)
\item {\bf FEDIT:}		Edit a `field' (parameter) for a range of sources
\item {\bf MERGE:}		Combine sources components that have 
				the same position
\item {\bf SORT:}		Sort the source list in decreasing amplitude 
				(sorting will always precede a write)
\item {\bf FSORT:}		Sort on a specified `field' (parameter)
				in the source list
\item {\bf DEL:}		Delete sources
\item {\bf DNCLOW:}		Delete non-CLEAN components with low amplitudes
\item {\bf DCLOW:}		Delete CLEAN components with low amplitudes
\item {\bf DAREA:}		Delete sources in specified area
\end{itemize}

In many cases, the user may specify sources in the list upon which some
`handling' options is to act. The specification may be in the form of
a {\bf SOURCE\_LIST} of id-numbers, separated by comma's (e.g. 3,6,7,2,34).
If the user answers with \scr, a {\bf SOURCE\_RANGE} is asked,
i.e. the first and last id-number of a contiguous sublist:

\skeyword{SOURCE\_LIST}
\sprompt{(Source number list)}
\sdefault{ = $\ast$:}
\suser{?}
\skeyword{SOURCE\_RANGE}
\sprompt{(Source number range)}
\sdefault{ = "":}
\suser{*}
\sinline{$\ast$ means all sources}

%----------------------------------------------------------------------------
\newpage
\subsection{Calibration}
\label{mdl.descr.handle-calib}

If the flux or position of one of the sources in the list is known, it may
be used as a calibrator. In the example, all 10 fluxes of the example model
used above are multiplied by a factor (2000/1900), to make the flux of the
first one equal to 2000 W.U. At the same time, all positions are shifted.


******* Insert new script here ******


%----------------------------------------------------------------------------
\newpage
\subsection{Sorting}
\label{mdl.descr.handle-sort}

The source components may be sorted in many different ways. The option 
SORT orders all components in order of decreasing flux.
{\it This operation is performed before each WRITE, so that the components
in an MDL file are always sorted this way!}.

\skeyword{MODEL\_OPTION}
\sprompt{(READ,WRITE,CLEAR,ZERO,SHOW, $\ldots$)}
\sdefault{= QUIT:}~
\suser{sort}

FSORT allows the sorting on a range of criteria. In this example, the soorces
are sorted in the order of decreasing m-coordinate:


****** Insert new script here *****


Other possibilities are:
~\\- I:  amplitude
~\\- L,M:  l or m
~\\- LM,ML:  l and m, or m and l
~\\- ID:  identification
~\\- Q,U,V:  Q or U or V
~\\- SI,RM:  spectral index or rotation measure
~\\- LA,SA,PA: long or short axis, position angle
~\\- BITS:  bits (i.e. if source is extended or has polarisation)
~\\- TYP:  source type (0 is the standard)
~\\- CC:  clean component
~\\- TP2:  reserved
~\\- DIST:  distance to a specified centre
~\\- POL:  polarised intensity


%----------------------------------------------------------------------------
\newpage
\subsection{Editing}
\label{mdl.descr.handle-edit}

Model components may be edited in various ways, using `handling' options
CLEAR, ZERO, ADD, MERGE, DEL, DNCLOW, DCLOW (see the overview above).
The handling options EDIT and FEDIT allow specific parameters (`fields')
to be edited separately. EDIT acts on a specified source, and FEDIT acts
on a range of sources. The possibilities are:

~\\- I:  amplitude
~\\- L,M:  l or m
~\\- ID:  identification
~\\- Q,U,V:  Q or U or V
~\\- SI,RM:  spectral index or rotation measure
~\\- LA,SA,PA: long or short axis, position angle
~\\- BITS:  bits (i.e. if source is extended or has polarisation)
~\\- TYP:  source type (0 is the standard)
~\\- CC:  clean component
~\\- TP2:  reserved
 


%==============================================================================
\newpage
\section{Conversion to a uv-model for the .SCN file}
\label{mdl.descr.uvmodel}

The calculation of uv-model data (for comparison with uv-data in NCALIB)i
s time consuming. Therefore, a calculated uv-model
is saved in the .SCN file for which it is calculated, together with the source
list used. In all programs that use the model data it can be specified how the
model calculation should be done, and if the calculation should be saved. The
relevant question is MODEL\_ACTION, which expects a list of three answers. The
first one can be one of:

\begin{itemize}
\item	{\bf merge:}	
	replace the model saved in the SCN node with the one
                     specified by the user. However, first compare these two
                     lists, and only add the difference to the saved model data.
                     I.e. make the calculation as short as possible if the new
                     list differs only slightly from the saved one.
\item	{\bf add:}	
		replace the saved model with the sum of the saved one and
                     the one specified by the user. The model calculated on the
                     basis of the user specified list is added to the saved
                     data.
\item	{\bf new:}
		replace the saved model by the model specified by the user,
                     and calculate a completely new set of model data
\item	{\bf temporary:}
		do not use any data in the SCN node, or write anything, but
                     use the data based on the list specified by the user
\item	{\bf increment:}	
	use the saved model data, and add to it the model data
                    based on the user specified list, but do not save anything
\end{itemize}

The second answer can be BAND or NOBAND, and specifies if in the model
calculation source data should be corrected for bandsmearing to match
the actual data better.  The third answer can be TIME or NOTIME to
indicate the use of integration time smearing. 










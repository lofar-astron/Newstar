%models_descr.tex

\newcommand{\NMODEL}{\textref{NMODEL}{nmodel\_descr} }
\newcommand{\local}{ {\em local} }
\newcommand{\apparent}{ {\em apparent} }
\newcommand{\epoch}{ {\em epoch} }
\newcommand{\header}{\textref{header}{.model.header} }
\newcommand{\source}{\textref{source}{.source.components} }

\chapter{ Models in Newstar}

{\it (contributed by W.N. Brouw 931011, edited by J.P. Hamaker 940411) }

\tableofcontents


\section{General}
\label{.general}

	Source models in Newstar are in principle meant to belong to a certain
observation, although models can be converted from/to belonging to other
observations.

\section{Model header}
\label{.model.header}	

	Each model has a header.  In this header the following
information is stored:

\begin{itemize}

\item	{\em TYPE}
	of coordinates used. This can be:
%
	\begin{itemize}
		\item	
		{\em local:}
		no information is available about the observation to which the
model pertains.  Whenever the model is used in connection with an
observation, it will be assumed that the model header data are identical
to the corresponding data in the observation. 
%
		\item
	 	\apparent:
		it is assumed that all coordinates are apparent.
%
		\item
		\epoch:
		it is assumed that all coordinates belong to B1950 or J2000
	\end{itemize}

\item	{\em EPOCH:} 1950 or 2000 (if {\em TYPE} = \epoch)

\item	{\em RA, DEC:}	\label{.ref.coord} 
	the reference right-ascension and declination (\apparent or
\epoch) to which the \textref{\em l,m}{.lm} coordinates of the \source
components refer.  It is assumed that this is also the centre of the
primary beam of the individual telescopes. 

\item	{\em FREQ:}
	the frequency ...

\item	{\em INST:}
	the instrument used in the observation.
\end{itemize}
%

\subsection{ Conversions between coordinate types}
\label{.coord.conversions}

	If the model list was made through the {\bf ADD} and {\bf EDIT}
options of \NMODEL, the type will be \local.  If it is generated through
\textref{NCLEAN}{nclean_descr} or the {\bf FIND} option of \NMODEL, the
type and other data are taken from the corresponding map. 

	A \local list can be converted to an \apparent or \epoch type
though the {\bf CONVERT} or {\bf EDIT} options of \NMODEL.  In this case
the reference data provided (either from a scan header or manually) will
be copied into the model header. 

	An \apparent or \epoch list can be converted to \local in the
same way. 

	An \apparent or \epoch list can be converted to another
\apparent or \epoch list with the \NMODEL {\bf EDIT} or {\bf CONVERT}
options.  For {\bf EDIT}, the new reference data from the user will
replace the existing \header data, without changing the actual \source
components.  {\bf CONVERT} will also convert the existing \source and
\header data to the new coordinates, i.e.  the new source list will be
for the new RA, DEC, FREQ, INST, using the model components' spectral
index, rotation measure, primary beam and sky position. 

	An \apparent or \epoch list can be converted to an \epoch or
\apparent list by \NMODEL {\bf EDIT} or {\bf CONVERT}.  For {\bf EDIT},
the new values will replace the old \header data, and only the change in
direction to the North Pole ('phi') will be applied to the individual
sources.  If {\bf CONVERT}, only a reference scan can be used, and the
\source components will be shifted, rotated, beamed, spectrally indexed
etc.  to the new coordinates. 


\section{ Source components}
\label{.source.components}

	Each source in the source model has the following information:
\begin{itemize}

\item	{\em INTENSITY:}
	intensity (STOKES $I$, in general in Westerbork Units) of the
source.  This value is always posaitive. 

\item	{\em L, M:}	\label{.lm}
 	
	the {\em l} and {\em m} coordinate on the sky for an E-W
interferometer pointed at the 
\textref{reference coordinates}{.ref.coord} 
in the model header; {\em l} and {\em m} are
thus given as offsets to the header position. 

\item	{\em Q,U,V:}
	Stokes polarisation parameters as percentages of Stokes {\em I}

\item	{\em SI:}
	spectral index of the source

\item	{\em RM:}
	the rotation measure of the source

\item	{\em extension:}
	extension parameters of the source modeled as a two-dimensional
Gaussian: length major axis, minor axis, position angle major axis.  For
a point source, both axes have length 0. 


\item	{\em source type:}	\label{.type}
	a number to be used to select particular sources only for
certain operations.  In general only type 0 will be used in the programs
(e.g.  self-calibration, updates, subtractions), but sometimes type = 1
are used (in a different way, e.g.  in
\textref{UPDATE}{nmodel_descr.update} they will not be updated, but
subtracted from the observations before updating starts).  Other types
are not used at present.  The user can manually set them (NMODEL
\textref{EDIT}{nmodel_descr.edit}) to e.g.  temporarily 'delete' them
from his model. 

\item	{\em ID:}
	a simple sequence number for easily keeping track of sources
across sorting and updates etc. 

\item	{\em MODE:}
	this is a bit mask with bits to indicate that the source is:
%
	\begin{itemize}
	\item	{\em clean component:}
	the source has been found by \textref{NCLEAN}{nclean_descr} and
is constrained to lie on a map grid point and to be a point source. 

	\item	{\em beamed:}	\label{.beamed}	 
	the source intensity has been corrected for the primary beam of
the instrument. 
	\end{itemize}
\end{itemize}

	Whenever \textref{NMODEL}{nmodel_descr} {\bf CONVERT} is used,
the above data will be adjusted to reflect the new \header data.  Note
that a {\em beamed} source will remain unchanged if the frequency and/or
instrument is changed but no spectral index is available. 

	All of the above information can be changed by the \NMODEL
HANDLE {\bf EDIT} and {\bf FEDIT} options. 

	The {\em beam} mode can be changed (and the intensity changed)
by the \NMODEL {\bf [DE]BEAM} options.  This operation will only be
applied to sources for which it is appropriate. 


\section{ Merging model lists}
\label{.merge.lists}

	Models can be combined with the \NMODEL {\bf xxx} option,
provided either 

\begin{itemize}

\item one of the types is \local (in which case the other type will be
assumed to be the correct one); or

\item both type are either \apparent or \epoch.  \end{itemize}

	The two models will be automatically converted and merged into a
single model with one of the {\header}s being used. 


\section{ Conversion of a model to visibilities}
\label{.model.to.vis}

	Conversion of a model to visibilities always occurs in the
context of a particular scan in a \textref{SCN file}{scn_descr}.  In
this process, the model is automatically converted from the
model-\header coordinates to those in the \textref{scan header}{scn_descr}. 

	In this process, \textref{\em beamed}{.beamed} \source
components will be de-beamed before being used; {\em de-beamed}
components will first be {\em beamed} for the header coordinates, then
{\em de-beamed} for the scan coordinates).  In addition non-clean
components will be {\em smeared} for the appropriate bandwidth and
integration time. 

	The user can suppress the smearing and the beaming operations
(\NMODEL {\bf MODEL\_ACTION} keyword).  He has no control over the other
conversions (spectral index, different position etc.). 


\subsection{ Instrumental polarisation}
\label{.instr.pol}

	In addition to the corrections discussed so far, {\em
position-dependent} instrumental polarisation (somewhat unfortunately
often referred to as {\em cross polarisation}) should be corrected for
in the calculation of model visibilities.  It is understood in principle
how this must be done, but the implementation has been suspended until
proper measured parameyters become available. 


\section{ Primary beam model}
\label{.primary.beam}

	The primary beam for the {\em WSRT} is currently modeled as a
circularly symmetric $\cos^{6}$ function.  Pointing measurements show
that this is a good approximation out to the first null.  If positions
further out were to be included, it would probably be better to model
the beam as a J0 (Bessel) function or as an $\exp(\cos^{6})$. 

	For the ATCA the beam is given as an inverse polynomial, which
is correct to the first side-lobe (except for non-symmetry around the
legs). 

\section{ Updating a model}
\label{.update}

	The \NMODEL {\bf UPDATE}, {\bf XUPDATE} and {\bf SUPDATE}
options try to match a givel source model the data as well as possible
to an observation in a .SCN file.The update process extracts its
information from residuals of the observed visibilities after all model
components of \textref{types}{.type} 0 and 1 have been subtracted. 
Updating is a process of fine-tuning the parameters of the model
components and is therefore not applicable to clean components. 

	To make the process not too slow, each \source component is
fitted separately, special care being taken that components close
together do not unduly influence each other's updates.  For the process
to work it is necessary that observation(s) span as large an HA range as
possible; otherwise the exact distance between to closely-spaced
components is difficult to determine. 

	The fitting process is linear, and uses both the real and
imaginary parts of the visibilities.  Since the equations used are
linearized, more than one iteration is in general needed for the process
to converge.  In the equations no account is taken of secondary effects
(e.g.  a shift in {\em l,m} will lead to different amplitudes due to
primary beam and frequency dependencies).  It is assumed that these
effects will also iterate out.  In those cases where this does not
happen, it probably would not have happened if they were taken into
account; due to dependencies between equations. 



% mongo_graphics.txt
%
% History:
%	JPH 940919	Make compilable in new doc. system
%
%
\chapter{ MONGO graphics}

In many cases, the user will need the results of \NEWSTAR processing in
graphical form. In order to make this easier, the output of programs
like NGCALC can be specified to be in `MONGO' format. These are
ASCII files, organised in columns, which can be processed by the
MONGO graphics program. Because it is ASCII, the knowledgeable user
can easily edit the file with a standard editor, to fine-tune the 
resulting representation.

MONGO \copyright is an interactive graphics program, written by John L. Tonry.
For a detailed description, see the MONGO manual. %% (\cite{MONGO}).
In this section, a few simple examples are given to give the user the
general idea, and to get him/her over the initial barrier.

In Dwingeloo, MONGO is only available on the VAX.
Create the MONGO command by typing
\\ \verb&MONGO :== @user5:[mongo]mongo.com& 

MONGO can be used interactively, or by means of a command ``macro'' (.MAC).
The data usually reside in an ASCII data file, which may be edited by
the user. 

%==============================================================================
\subsection{A simple MONGO plot}
\label{.simple.plot}

Create a MONGO data file (extension .DAT).
This is an ASCII, in which the data is lined up in columns.
In this case, replace the string "....." in the first column with 0,1,2,3 etc
to indicate channel nr. 

Then make a MONGO command macro (ext .MAC):
\begin{tabbing}
+++\=+++++++++++++\=   \kill			% set tabs
\\ \> \verb&data gain64a.dat& 	\>MONGO data file
\\ \> \verb&limits -5 70 -10 10&\>
\\ \> \verb&xcol 1&		\>x-values in 1st column of data file
\\ \> \verb&ycol 2&		\>y-values in 2nd column of data file
\\ \> \verb&box&		\>draw a box around the plot
\\ \> \verb&ptype 4 3&		\>
\\ \> \verb&connect&		\>
\\ \> \verb&xlabel channel #&	\>horizontal axis label string
\\ \> \verb&ylabel ....&	\>vertical axis label string
\\ \> \verb&id&			\>
\end{tabbing}

Alternative: use MONGO interactively.

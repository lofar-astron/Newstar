%  ncalib_redun.tex  v1.2 04/08/93 JEN  
%       JPH 940406      Technical changes  
%       JPH 951107      Break a paragraph that was too long for ndoc  
%       JPH 960326      Further fix for this problem  

\newcommand{\be}{\begin{enumerate}} 
\newcommand{\ee}{\end{enumerate}} 
\newcommand{\I}{\item} 


\chapter{NCALIB REDUN: Redundancy, Align and Selfcal} %  
\tableofcontents %  
\section{Overview}  

        The NCALIB option {\bf REDUN} actually covers three related methods of
estimating telescope gain and phase errors from the uv-data itself: Redundancy,
Selfcal and Align. All three methods require the socalled `Selfcal assumption',
which states that all phase and gain errors are {\em telescope-based}.   
% (line break for sake of ndoc - donot remove leading blank)  
 This means that the errors can be fully decomposed into contributions from
individual telescopes, and that interferometer-based errors can be ignored.
This assumption implies a drastic reduction in the number of {\em independent}
errors in the data taken with an N-telescope array: from $N(N-1)/2$ to $N-1$
per integration interval.  This relatively small number of independent errors
can be determined with a {\em least-squares fitting} technique.  

        Any {\em interferometer-based} errors (e.g.  thermal noise or
correlator errors) violate the basic Selfcal assumption, and cause propagating
errors in the solution.  Fortunately, the WSRT correlators only contribute very
small interferometer errors (typically $<$0.01\%),  
but the S/N per uv-sample should be at least 2-5 for a good solution.  

        Independent solutions can be obtained for telescope gain and phase
errors, because they are mathematically `orthogonal'.  

\begin{itemize}  
\item {\bf Selfcal:}  
        The telescope gain and phase errors are estimated by comparing the
uv-data with a model of the observed source.  In WSRT Selfcal, the information
from redundant spacings can be added as {\em extra constraints} on the Selfcal
solution.  Since this extra information is model-independent, the Selfcal
process is less likely to converge to the wrong result.  

\item {\bf Redundancy:}  
        Telescope gain and phase errors are estimated by comparing the uv-data
of `redundant' interferometers, i.e.  interferometers that have the same
baseline length and orientation.  Since this is a comparative method, the
absolute gain (flux) and the absolute phase gradient (position) cannot be
determined.  The result is a set of `internally perfect' HA-scans, that still
have to be `aligned' (see below) to each other in flux and position.  

\item {\bf Align:}  
        The absolute gain and the absolute phase gradient for misaligned
HA-scans can be determined with the help of a source model.  This is similar to
Selfcal, except that one only solves for one parameter per HA-scan, rather than
for N telescope errors.  This has the advantage that the source model may be
less perfect, and the SNR of the uv-data may be lower.  Therefore, this method
may also be used to remove ionosferic phase gradients from data that have too
little SNR to warrant Redundancy or Selfcal.  

\end{itemize}  

        The figures of this section may help to illustrate the effects of these
three methods.  There is also a description of the relevant mathematical
formalism.  Finally, this section contains processing examples and an
explanation of the output that is produced on the screen and in the log file.  

\input{../fig/ncalib_3c48.cap}  
\input{../fig/ncalib_scan.cap}  


%=============================================================================  
%\include{form_redun}                   % REDUN mathematical formalism  
%=============================================================================  

\input{../fig/ncalib_matrix.cap}  
\input{../fig/ncalib_vispace.cap}  


%=============================================================================  
\section{Redundancy}  
\label{.redundancy}  

        Redundancy-only is selected by {\bf not} specifying a source model. The
resulting telescope gain and phase errors are stored in the Scan file headers,
as REDC.  

**** Put new script here ****  


%=============================================================================  
\section{Align}  
\label{.align}  

        Align is selected by specifying a source model, and explicitly
specifying the {\bf ALIGN\_OPTION}.  It is then assumed by default that the
Scan is `perfect', i.e.  that all 14 telescopes are grouped together.  In that
case, only two parameters have to be determined: the absolute gain, and the
absolute phase gradient over the array. Experienced users can specify
multi-parameter solutions for more than one independent groups of telescopes by
manipulating the {\bf FORCE\_FREEDOM} keyword.  The {\bf MWEIGHT} keywords are
used to give greater weight to those baselines (by length), for which the model
is `known' to be most accurate.  

        The resulting telescope gain and phase corrections are stored in the
Scan headers, as ALGC.  

*** Put new script here ****  

%=============================================================================  
\section{Selfcal}  
\label{.selfcal}  

        Selfcal is selected by specifying a source model, and explicitly
specifying `Selfcal' to the {\bf ALIGN\_OPTION}.  Redundancy constraints
(equations) are included automatically if redundant spacings have been selected
with {\bf SELECT\_IFRS}.  The {\bf MWEIGHT} keywords are used to give greater
weight to those baselines (by length), for which the model is `known' to be
most accurate.  

        The resulting telescope gain and phase corrections are stored in the
Scan headers, as ALGC.  

*** Put new script here ****  


%=============================================================================  
\newpage  
\section{Discussion of the screen/log output}  
\label{.log.output}  

The following information per HA-scan may be printed in the LOG-file and/or
displayed on the terminal screen (keyword {\bf SHOW\_LEVEL}).  

\slong{  HA    Rk  A(\%) P(deg) A(WU)  P(WU) Amax    Aavg   Arms    dAmax  
dPmax  I}  
\sskip  
\slong{Set: 0.0.0.0}  
\slong{ 11.53X New phase constraints:}  
\slong{           1   1   1   1   1   1   1   1   1   1   1   1   1   1}  
\slong{ 11.53X 01   0.6   0.2   12.9    7.1  2012  2000.0   5.2    1.9 67 0.5
58 3}  
\setc  

\vspace*{-10mm}  
\begin{tabbing} +++++\=+++++\=+++++\=+++++\=+++++\=+++++\=+++++\=+++++\=+++++\=
\kill %tabs  
\\ \> HA        \> (e.g. -83.86X) \>\> Hour-angle (degr) and polarisation  
\\ \> Rk        \> (e.g. 12)    \>\> Rank of the gain and phase solution
matrices  
\\ \> A(\%)     \> (e.g. 3.3)   \>\> RMS gain residual (\%)  
\\ \> P(deg)    \> (e.g. 2.2)   \>\> RMS phase residual (degr)  
\\ \> A(WU)     \> (e.g. 1.6)   \>\> RMS gain residual (WU=Westerbork Unit)  
\\ \> P(WU)     \> (e.g. 1.8)   \>\> RMS phase residual (WU)  
\\ \> Amax      \> (e.g. 63)    \>\> Maximum ampl (WU)  
\\ \> Aavg      \> (e.g. 50.1)  \>\> Average ampl (WU)  
\\ \> Arms      \> (e.g. 5.1)   \>\> RMS ampl (WU)  
\\ \> dAmax     \> (e.g. -12.4 23) \>\> largest gain  
\\ \> dPmax     \> (e.g. 6.3 CD) \>\> largest phase residual (WU), ifr=CD  
\end{tabbing}  


\slong{X average amplitude= 2000.005 (0.892)}  
\sskip  
\slong{X overall noise (gain, phase in W.U.):     14.2     11.1}  
\sskip  

\begin{itemize}  
\item   {\bf Average amplitude:}  

\item   {\bf Average gain and phase errors per telescope:}  
        Over the whole observation (or rather, the part that has just been
processed). These numbers can also be calculated separately by means of NCALIB
option SHOW.  

\item   {\bf Overall noise:}  
        Useful for automatic deletion of `bad' scans (see NSCAN).
Redundancy-only is model-independent, so the overall noise should be equal to
the thermal noise. If not, it is an indication of problems. A difference
between SELFCAL noise and Redundancy-only noise is an  indication of the
completeness of the SELFCAL model (caution: there are different interferometers
involved).  

\item   {\bf Graphs:}  
        There are three kinds of line-printer graphs produced in the log-file.
They give various overall gain and phase quantities per interferometer. The
baseline length increases to the right. The gain axis (A) is on the left, and
the phase axis (P) on the right.  
  \begin{itemize}  
  \item {\bf Graph: Average residual error X (W.U.):}  
  \item {\bf Graph: Average residual error X (\%, deg):}  
  \item {\bf Graph: RMS X (W.U.):}  
  \end{itemize} %  
\end{itemize}  


        The SELFCAL and Redundancy residuals contain a wealth of information
about the quality of the data and the completeness of the SELFCAL model. The
user is urged to make residual plots by means of the program NPLOT.  


%=============================================================================
 
\section{QDETAILS: Hidden parameters}  
\label{.qdetails}  

        For all REDUN options, the user is prompted for `more details?' by the
NCALIB keyword QDETAILS.  Hidden behind this are a number of keywords that may
be manipulated by experienced users in special cases. Their default values are
optimised for normal use, and wil be printed in the NCALIB LOG-file.  

\skeyword{BASEL\_CHECK}  
\sprompt{(M)}  
\sdefault{ = 0.5 M:}  
\suser{\scr}  
\scomment{ Criterion for two baselines to be considered redundant, i.e. of
identical length.}  

\skeyword{WEIGHT\_MIN}  
\sprompt{(Minimum weight accepted)}  
\sdefault{= 0.01:}  
\suser{\scr}  
\scomment{The weight of Selfcal and Redundancy equations is proportional to the
amplitude of their uv-data. If the amplitude is very small, the information
will be very noisy, and may do more harm than good to the solution. Therefore,
it may be excluded by raising the value of WEIGHT\_MIN, as a fraction of the
maximum weight.}  

\skeyword{SOLVE}  
\sprompt{(Solve for gain, phase (Y/N)}  
\sdefault{= YES,YES:}  
\suser{\scr}  
\scomment{Normally, both a gain and a phase solution will be required. But it
is possible to ask for only one at a time.}  

\skeyword{COMPLEX}  
\sprompt{(Complex solution (Y/N)}  
\sdefault{ = NO:}  
\suser{\scr}  
\scomment{The non-linear conversion to gain and phase skews the gaussian
distribution of the noise on the measured cos/sine values. Therefore, gain and
phase solutions will produce a `noise bias', which is more serious for low S/N
data. This can be avoided by specifying a `complex solution' (see also section  
...)}  

\skeyword{FORCE\_PHASE}  
\sprompt{()}  
\sdefault{ = 0, 0, 0, 0, 0, 0, 0, 0, 0, 0, 0, 0, 0, 0 }  
\suser{\scr}  
\scomment{If the visibility phases are close to $\pm 180~degr$, the conversion
from cos/sine to phase may cause phase ambiguities (jumps) of 360 degr. This
will cause problems in the phase solution, where the phases are assumed to be
on a linear scale between $\pm \infty$. In order to avoid this, the user may
specify initial phase corrections for all telescopes, which will be used to
move the data away from $\pm 180~degr$ before processing. This is of course
taken into account for the total correction afterwards.}  

\skeyword{CONTINUITY}  
\sprompt{(Continuity in solution (Y/N)}  
\sdefault{= YES:}  
\suser{\scr}  
\scomment{Normally, the HA-scans are processed in HA-order. The gain and phase
errors determined for the last Scan may be used to correct the data of the next
Scan {\em before} processing. (This is of course taken into account for the
total correction afterwards).  This approach is useful to keep phases away from
$\pm 180~degr$, where they may cause unwanted phase-ambiguity problems.}  

\skeyword{CHECKS}  
\sprompt{(Maximum deviations)}  
\sdefault{ = 5,5,3:}  
\suser{\scr}  
\scomment{The user may set some threshold values for an automatic check on the
quality of the solution of each HA-scan. A warning will be issued in the log
whenever the tresholds are exceeded. However, no further action is taken by the
program!}  

\newcommand{\void}[1]{} 
\void{  
\section{Pathological situations} 

        It may happen that NCALIB fails to produce a decent solution from the
latter observations: The errors reported for the amplitude and phase noises are
mucj larger than one could reasonably expect. This is a sure sign of trouble:
Apart from obvious causes such as interference, it may be a consequence of
improper phase corrections applied in the on-line observing system. 

        The trouble comes from the fact that under such conditions NCALIB may
not be able to correctly resolve the 360-degree phase ambiguity inherent in
determining a visibility's phase. This kind of problem can be fixed with a
little bit of work. 

\be 
\I      Determine whether the problem occurs in the XX or YY interferometers or
both. The remaining steps must be applied to each separately. 

\I      Use the \textref{NSCAN}{nscan_descr} SHOW option to display the table
of amplitudes and phases in a representative scan.  

\I      Take a good look at the lower triangle where the phases are displayed.
What you must expect is to find (probably one) telescope(s) for which the
phases in different interferometers flip wildly between two values, one in the
vicinity of +180 degrees and the other of -180 degrees. 

\I      Re-execute your Selfcal run, but this time use the FORCE_PHASE
parameter to force the phases for the suspected telecope to either + or - 180
degrees, e.g. if Telescope 3 give problems, specify FORCE_PHASE = ,,,180. 
\ee      }  



%
% @(#) ncopy_descr.tex  v1.1 08/02/94 JPH
%
%	JPH 940927	Remove non-standard constructs

%
\chapter{The Program NCOPY}
\tableofcontents 

\section{General}

	NCOPY is a program to selectively copy data from an input to an
output SCN file. Its basic use is to condense SCN files by
\begin{itemize}

\item   selecting only those sectors, scans and polarisations that one wants
to retain;

\item   eliminating "holes" in a file that are no longer used.
\end{itemize}
As an ancillary function, NCOPY can produce overviews of all sectors in a SCN
file.



\section{NCOPY options and keywords}
\begin{itemize}
\item   OVERVIEW: Produce on terminal and in the log file a listing of all
sectors in a file. In addition to the layout (as for NFLAG SHOW) and file
size, the following is shown for each sector:

                {\sf g.o.f.c.s} and absolute sector numbers;
                field name and WSRT observation number;
                frequency and bandwidth;
                hour-range and number of scans;
                numbers of interferometers and polarisations;
                technical information for testing/debugging purposes (subject
to change).

The information relevant to the user is crammed into the width of a terminal
window.

\item   COPY: Copy sectors. The following additional information is requested
once for every time the COPY option is selected:

\indent       INPUT\_SCAN \\
\indent       OUTPUT\_SCAN \\
\indent       INPUT\_SECTORS \\
\indent       HA\_RANGE \\
\indent       POLARISATION \\

	Data will be appended to the output file. If the latter does not
exist, a new file is created and its creation reported.

	A new output group is created for every input group, the
remaining indices, o.f.c.s, being taken from the input.
\end{itemize}


\section{Multiple references to the same physical sector}

	This section is relevant only for cases where the input SCN file
contains sectors that are referred to through more than one g.o.f.c.s
reference. The only ways such multiple references may have been created is
through the NSCAN REGROUP option or through an earlier NCOPY COPY operation as
described below.

	To understand the implications of multiple references for NCOPY,
one must understand the distinction between {\em physical sectors} and {\em
sector references}. The former hold the physical data and are therefore
unique. The references are small data structures embodying the g.o.f.c.s
hierarchical indexing structure through which the physical sectors are
accessed.

	An input physical sector specified repeatedly in a {\em single} COPY
operation will be copied only once. If the INPUT\_SECTORS specification either
explicitly or implicitly refers to the sector more than once, a single
physical copy is made, along with a copy of each of the references to it.

	If an input sector already copied is selected again in a {\em new}
COPY operation, the correct response would be to make a new reference to the
output sector already in existence. However, NCOPY has no way of knowing which
sectors were already copied in the past, so a new independent copy of the
entire sector will be made.


\section{Possible later extensions}

	As future options, to be realised only upon user demand, we
consider:
\begin{itemize}

\item   integration of data over hour-angle intervals;

\item   selection of interferometers;

\item   definition of output sector numbers by the user and (partial)
correction of the deficiencies noted above;

\item   permanent application of corrections to the stored data as foreseen in
the original Newstar design.

\end{itemize}



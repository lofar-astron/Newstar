%
% @(#) nflag_descr.tex  v1.2 04/08/93 JEN
% HjV 950615    Separate parts by indenting text or by using two blank lines
% JPH 951016    Revision
%
\newcommand{\Em}[1]{{\em #1\/}}
\newcommand{\bi}{ \begin{itemize} }
\newcommand{\ei}{ \end{itemize} }


\chapter{The Program NFLAG}

\Em{\center
Contributed by Johan Hamaker, October 1995.\\
Revision of original docyment by Jan Noordam, August 1993.
}

\Em{
This is a preliminary version of a new document replacing the old version of
August 1993 by J.E. Noordam. The information in it is thought to be accurate
but known to be incomplete. It should, however, in its present from help the
user to get to grips with NFLAG's overwhelming user interface and functionality.
}

\tableofcontents

%================================================= Standard subsection =======
\section{Overview}
\label{.overview}

	The basic function of NFLAG is to mark visibility points in a .SCN file
as faulty by 'flagging' them, i.e. by setting bits in the 'flag mask'
associated with each point. In general, visibilities thus flagged will be
considered no longer to exist; in this sense flagging is equivalent to
deletion, but unlike deletion it is reversible, i.e. flag bits can be cleared
at a later time. It is also possible to instruct processing programs to ignore
flag bits and treat flagged visibilities as valid.

	A secondary function in NFLAG, not further described here, is the SHOW
option through which one may explore the contents of a .SCN file. For a
description, see the descriptions of \textref{NSCAN}{nscan_descr} and its
\textref{parameter interface}{nscan_private_intfc}.


\section{ The \NEWSTAR flags concept}
\label{.flag.concept}

	The simplest way to get rid of bad visibilities is by \Em{deleting}
them. Typically this is done by replacing each bad value by a special value
that processing programs recognise as an instruction to ignore the data point.
Since its original value is overwritten, a deleted data point is irreversibly
lost.

	The first step toward refinement is to \Em{flag} visibilities as bad
without overwriting them; processing programs ignore those data for which the
flag is set. Since the original value is retained, is can be 'undeleted' by
clearing the flag; flagging is therefore also known as 'reversible deletion'.
This method is a well-known one, because it is implemented AIPS.

	In \NEWSTAR, it is carried a step further by differentiating between a
number of different reasons for which a visibility may be considered bad.
Rather than a single flag, \NEWSTAR associates a complete \Em{flag byte} with
each visibility point: Thus one has not a single bit to play with, but eight
different ones.     The advantage of this approach is that different flagging
operations can be treated independently of one another. This makes it possible
in many cases to backtrack form a flagging operation that produced undesirable
results and thus introduces an important margin of safety: For instance, a
selection made in a laborious manual flagging session cannot be accidentally
undone by a subsequent automatic clipping operation going wrong.

	Five of the eight bits in the flag byte are used by NFLAG, the user may
use the remaining three in whichever way he wishes. By default, programs bypass
all visibilities for which any one of the eight flags is set; it is also
possible, however, to instruct a program to ignore certain flags. The three
user-definable flag-types can be used for experiments with different selections
of uv-data. (This is analogous to the use of different flag tables in AIPS.)

	Below is a list of the flags defined by \NEWSTAR. The numbers shown in
parentheses are the bit number (LSB=0) and hexadecimal values for these flags:
\bi

\item   MANUAL (bit 7, 80): NFLAG sets this flag when the user explicitly
selects data points for flagging by specifying their coordinates (sector index,
hour-angle range etc.)

\item   CLIP (bit 6, 40): This flag is set in operations that select bad data
by comparing them against some limit, e.g. a maximum for their cos or sin
components or their selfcal residual amplitiude.

\item   NOISE (bit 5, 20):  This flag is set in operations that select groups
of bad data by checking their noise levels, e.g. those that NCALIB has
calculated and stored in the .SCN file in a \whichref{REDUN}{} or
\whichref{SELFCAL} operation.

\item   ADDitive (bit 4, 10):  This flag is set in operations that select
groups of bad data by comparing their average value against some limits

\item   SHADowing (bit 3, 08):  This flag is set in operations that select bad
data on the basis of some function of their coordinates, e.g. the length of the
projected baseline or a geometry in which shadowing occurs.

\ei
The remaining three bits are known as U1 (bit 2, 04), U2 (bit 1, 02) and U3
(bit 0, 01).

	The \NEWSTAR visibility flagging scheme gives the user unprecedented
flexibility for reversible data-editing, although its very power makes it a
little more difficult to understand and use than a simple scheme such as AIPS's.


\section{ Flagging logistics}
\label{.logistics}

	The only program that actiually sets and clears flags is NFLAG. It
contains a number of options to select bad data on a varied range of criteria
and set flags accordingly. To verify the effect, it also offers the option to
\whichref{count flags}{.} over various cross-sections or projections of the
\textref{data hypercube}{scn_file.hypercube} and \whichref{tabulate}{.} the
results.

	Another option is to select flags on a display of some two-dimensinal
cross section through the hypercube, using the program \whichref{NGIDS}{}.
NGIDS then produces a file with flagging commands which may subsequently be
\whichref{read and executed} by NFLAG.

	On a micro-level, it is possible to inspect the individual flag bytes
in individual scans through the \textref{SHOW}{.overview} option.


\subsection{ Visibility and scan-header flags}
\label{.header.flags}

	In addition to the flag bytes associated with the individual
visibilities, the \textref{scan header}{scn_file.scan} also contains a flag
byte. Through the flags in this byte one may invalidate the scan as a whole.
This is obviously a more expedient way to handle flagging, since a single
computer instruction will flag or test the entire scan.

	In the present implementation of NFLAG, visibility flagging and
scan-header flagging operate independently. That is, when the header is flagged
the corresponding flags on the individual visibilities are left unchanged, and
if the header is later unflagged those flags come back into force. On one hand
this introduces some additional independence between different flagging
operations as discussed \textref{above}{.flag.concept}. On the other hand it
forces upon the user a technical distinction between two types of flagging that
are functionally equivalent: The user must contend with such flagging
operations as
%
\textref{HASCANS}{nflag_private_intfc.ops.scans},
\textref{CLHEAD}{nflag_private_intfc.ops.manual},
\textref{TOHEAD, TODATA}{nflag_private_intfc.ops.fcopy},
statistics specifiers such as
\textref{SCANS}{nflag_private_intfc.ops.statist}.
(Unfortunately, the keywords used are not an example of clarity and
consistency...)



\section{ Primary and secondary visibility (hyper)cubes}
\label{.hypercubes}

	We assume here that the concept of the visibility
\textref{hypercube}{scn_file.hypercube} is understood. We recall that the
group/observation, mosaic-subfield, frequency-channel and (for a mosaic
observation) the successive hour-angle 'cuts' are selected through the compound
\textref{sector index}{scn_file.SCNSUM.indices}; the scan's hour-angle, and
within a scan the interferometer and polarisation are selected by direct
specification.

	At the very start, NFLAG asks you to select a \Em{primary data
hypercube} to operate on, through the parameters
\bi{}
\item\Textref{SCN\_NODE}{scnnode_public_intfc.scn.node}:
	.SCN file;
\item   \Textref{SCN\_SETS}{scnsets_public_intfc.scn.sets}:
	sets of sectors, i.e. one ore more combinations of group, observation,
	mosaic subfield, frequency channels and hour angle cuts;
\item   \Textref{HA\_RANGE}{select_public_intfc.ha.range}:
	hour-angle range;
\item   \Textref{SELECT\_IFRS}{select_public_intfc.select.ifrs}:
	interferometers;
\item   \Textref{SELECT\_XYX}{select_public_intfc.select.xyx}
	polarisations.
\ei

\noindent This primary cube may later be redefined partly or completely through
the MODE branch of NFLAG which is accessible from \whichref{several places}{.}
in the program.

	For certain operations, one may narrow down the range of visibilities
to be operated on by defining a \Em{secondary hypercube}; the data volume
operated upon will then be the cross section of the primary and secondary
cubes. Secondary-cube definition is limited to the latter three of the
parameters listed above; note, in particular, that it is not presently possible
to define a secondary frequency-channel range because that definition is part
of the SCN\_SETS parameter.

	A word of caution is in order about the effect of data-cube definitions
in operations that affect the \textref{\Em{scan-header}}{.header.flags} flags.
Since such actions operate on the entire scan, \Em {the ranges of
interferometers and polarisations (parameters \Em{SELECT\_IFRS} and
\Em{SELECT\_XYX}) that you defined for your hypercubes have no effect}.


\section{ The flagging branch of NFLAG}
\label{.flagging.branch}

\input{../fig/nflag_flag.cap}

\input{../fig/nflag_operate.cap}

	The flagging branch is where the actual work of NFLAG is done. All
commands that change flag setting on visibility points or scan headers are to
be found here. The flagging options are shown in the upper left box in
\figref{.nflag.flag} A detailed overview of the branch is given in
\figref{.nflag.operate}.


\section{ The INSPECTion branch of NFLAG}
\label{.inspection.branch}

\input{../fig/nflag_inspect.cap}

	The purpose of the inspection branch is to provide overviews of the
flags present. Generally speaking, the bulk of the visibilities is much too
large for making useful graphic or tabular representations of the individual
flag settings. For most purposes, however, the information one needs can be
represented in the form of tables of flag counts along various cross-sections
of the visibility hypercube. Such tables is what the inspection commands
produce.

	A schematic overview of the branch is given in \figref{.nflag.inspect}.

	From the INSPECT branch, it is possible also to make a detour into the
\textref{MODE}{.mode.branch} or \textref{STATISTICS}{.statist.branch} branch.


\section{ The STATISTics branch of NFLAG}
\label{.statist.branch}

\input{../fig/nflag_statist.cap}

	The purpose of the statistics branch is to provide statistical
overviews of the visibility data. Generally speaking, the bulk of the
visibilities is much too large for making useful graphic or tabular
representations of the individual values (except with \whichref{NGIDS}). For
flagging purposes, however, much of the information one needs can be
represented in the form of tables of data statistics along various
cross-sections of the visibility hypercube. Such tables is what the statistics
commands produce.

	A schematic overview of the branch is given in \figref{.nflag.statist}.

	From the STATISTICS branch, it is possible also to make a detour into
the INSPECT branch.


\section{ NFLAGS environment paremeters: The MODE branch}
\label{.mode.branch}

%\input{../fig/nflag_mode.cap}



\section{ Selection of bad visibilities on an X11 display }
\label{.x11}

\input{../fig/nflag_gids.cap}

	For detailed point-by-point flagging of visibilities the flagging modes
of NFLAG are much too cumbersome. For this purpose one may instead use a
combination of
\bi
\item   NGIDS, to select data points with a cursor on a two-dimensional display
of a cross section through the visibility hypercube in which data values are
represented by colours;

\item   NFLAG, to do the actual flagging of the selected points.
\ei
The procedure is schematically represented in \figref{.nflag.gids}.


.c+

{\bf The user is urged to work through the examples in this section before
starting to use NFLAG. Quite a lot of valuable (and sometimes essential)
information is available in the on-line HELP text.}



%------------------------------------------- Option NFLAG subsubsection ------
\newpage
\subsection{Example: two flagging operations}
\label{.example1}

	This example has been annotated, so that it may serve as a first
introduction to the use of NFLAG option FLAG. Starting with an unflagged
SCN-file, two successive flagging operations are shown, using two different
flag types (MANUAL and SHADOW). The results of this particular example
are shown in the description of NFLAG option SHOW, later in this section.


**** Put new script here ***


{\it  The other FLAG\_OPTIONs deal with the NFLAG internal flag list:
Clearing it and copying flags to/from the two types of flag files.
This will be dealt with in a later example of GET/PUT.
Now the uv-data hypercube will be defined, to which the subsequent
(un)flagging operations will be limited:}


**** Put new script here ***


{\it In an FILLED data hypercube, all interferometers and all polarisations are
selected. Some operations will then set flags in the Scan header, rather than
in the individual uv-data. The user should be aware of this when inspecting
flags, and in subsequent unflagging operations.}


**** Put new script here ***


{\it At least one FLAG\_MODE must be specified here, e.g. FLAG.
The specified FLAG\_MODEs will be active from here onwards.
NOCORRECT indicates that no corrections will be applied to the uv-data when
they are read in.
In the following (un)flagging operations, their default flag types will be
used, since we have not used the option UFLAG to override them.}


{\it  First, some individual interferometers will be flagged, using the manual
operation IFR. Its default flag has type MANUAL. This operation has its own
interferometer selection, which can only {\em narrow} the hypercude specified
above.}


**** Put new script here ***


{\it The message per (affected) Scan is the result of the flag-mode SHOW,
which will now be turned off. The second operation that will be
shown here is the flagging of uv-data that are affected by `shadowing',
i.e. the partial blocking of the field of view of one telescope by another.
This is called a `deterministic algorithm', since it only uses known
instrumental parameters like HA, elevation and telescope position.
The operation has the default flag type SHADOW.}


**** Put new script here ***


{\it For real flagging of shadowed data, the actual diameter of 25m for WSRT
telescopes should of course be given here. This example is to demonstrate
a serendipitous alternative use of this operation, i.e. the flagging of
all baselines with a {\em projected} size of less than 200m.}


{\it  The result of these two flagging operations may now be verified
with NFLAG option SHOW. This is done, for this particular example, in
the description of option SHOW at the end of this section.}


**** Put new script here ***


%------------------------------------------- Option NFLAG subsubsection ------
\newpage
\subsection{Example: resetting all flags}
\label{.reset}

The following example is very important for users who have lost their way
in the many options and operations of NEWSTAR flagging, and want to start
with a clean slate. It also emphasizes the two questions that users should
constantly ask themselves: Which flag types are being affected, and are they
affected in the Scan header or in the individual uv-data?


**** Put new script here ***


{\it Note the use of UFLAG to select all (8) flag types, This overrides the
default flag types of any subsequent flagging operations.}


**** Put new script here ***


{\it This takes care of the flags in the individual uv-data. To make sure
that also the flags in the Scan headers are reset, another operation must
be performed:}


**** Put new script here ***


{\it Now all flags in this SCN-file have been reset.
This may be verified with NFLAG option SHOW.}



%------------------------------------------- Option NFLAG subsubsection ------
\newpage
\subsection{The internal flag list: GET and PUT}
\label{.list}

The program NFLAG uses an internal flag list to store flags temporarily
in the process of copying them from one place to another. This is
illustrated in the block diagram NFLAG\_001.fig.


The flag list can be filled in either of two ways:

\begin{enumerate}
\item
From the uv-data in the SCN-file, in one or more GET operations. The user
should keep in mind that each GET operation will add entries to the list.
the list should be explicitly CLEARed when necessary.
\item
From one of the two kinds of flag file, i.e. an FLF-file (LOAD) or an
ASCII flag file (READ). The reverse operations (UNLOAD and WRITE) are
also possible.
Thus, to store flags that are set in a SCN-file in a flag-file, they must
first be GOT into the internal list, and then UNLOADED or WRITTEN to
a flag-file.
\end{enumerate}

The PUT operation, with which flags are copied from the internal flag list
to the uv-data is very powerful, because it is possible to specify a
socalled PUT\_RANGE. This is a range in the four `dimensions' channel, HA,
ifr and polarisation. For each entry in the flag list, not only the
corresponding uv-data point will be flagged, but a 4-dimensional hypercube
around it. For example: if some automatic algorithm has detected interference
in one frequency channel, it is possible to flag all frequenct channels at
that particular HA and ifr in this way.

%------------------------------------------- Option NFLAG subsubsection ------
\newpage
\subsection{The two kinds of flag files}
\label{.file}

Flags may be stored into two different files: the FLF-file is (much) smaller
because the information is stored in a compact format. The same information
can also be stored in an ASCII file (default name FLAG.LOG), which can be
inspected and edited with a normal text editor. It is up to the user to
choose between these two kinds of file.

When printed, the ASCII file looks as follows:

\begin{verbatim}
> more FLAG.LOG
 !+ Flagging file FLAG.LOG
 !  Created by NOORDAM on 930715 at 17:06:18 at rzmws0
 !  Flags:
 !       MAN : 80  CLIP: 40  NOIS: 20  ADD : 10
 !       SHAD: 08  U3  : 04  U2  : 02  U1  : 01
 !  Types:
 !       00:  Interprete Ifr field as interferometer
 !       01:  Interprete Ifr field as baselines in m
 !  Data following an ! are seen as comments
 !  Remaining fields have format:
 !       *:          all values
 !       value:      single value
 !       val1=val2:  value range (inclusive)
 !
 !-
 !Flag  Type  Channel       Hour-angle          Ifr         Pol
 40     00    0             -88.74              *           *
 40     00    0             -88.49              *           *
 40     00    0             -88.24              *           *
 40     00    0             -87.99              *           *
 80     00    0             -10.02              *           *
 80     00    0             -9.77               *           *
 80     00    0             -9.52               *           *
 80     00    0             -9.27               *           *
 80     00    0             -9.02               *           *
 80     00    0             -8.77               *           *
\end{verbatim}

Note the two flag types, MANUAL (80) and NOISE (40), which have obviously been
set in two different flagging operations.
{\it (NB: the use of the word `type' for the second column is a bit cunfusing
here: it has no relation to the flag type!)}


%============================================================================
\newpage
\section{Interactive flagging using NGIDS}
\label{.interactive}

In the following, a complete sequence of steps will be shown:
~\\ - Making a `map' of uv-data in a WMP file, using NMAP
~\\ - Displaying the uv-data on the screen, using NGIDS
~\\ - Specifying areas of uv-data to be flagged, using mouse and NGIDS
~\\ - Writing the flags from NGIDS to an ASCII flag-file
~\\ - Printing the ASCII flag-file to inspect it
~\\ - Reading the flags into the NFLAG internal flag list
~\\ - Copying the flags to the uv-data in the SCN-file, using PUT
~\\ See the block diagram NFLAG\_001.fig.

%------------------------------------------------------------------------
\subsection{Putting the uv-data into a WMP file}
\label{.interactive.wmp}

**** Put new script here ****


%------------------------------------------------------------------------
\subsection{Displaying the WMP file with NGIDS}
\label{.interactive.display}

**** Put new script here ****

%------------------------------------------------------------------------
\subsection{Interactively specifying flags in NGIDS}
\label{.interactive.flag}

**** Put new script here ****

%------------------------------------------------------------------------
\subsection{Transferring flags with an ASCII flag file}
\label{.interactive.flag.file}

**** Put new script here ****


\begin{verbatim}
> more DEMOFLAG.LOG

 !+ Flagging file DEMOFLAG.LOG
 !  Created by NOORDAM on 930724 at 17:18:00 at rzmws0
 !  Flags:
 !       MAN : 80  CLIP: 40  NOIS: 20  ADD : 10
 !       SHAD: 08  U3  : 04  U2  : 02  U1  : 01
 !  Types:
 !       00:  Interprete Ifr field as interferometer
 !       01:  Interprete Ifr field as baselines in m
 !  Data following an ! are seen as comments
 !  Remaining fields have format:
 !       *:          all values
 !       value:      single value
 !       val1=val2:  value range (inclusive)
 !
 !-
 !Flag  Type  Channel       Hour-angle          Ifr         Pol
 80     00    *             -62.67              6D          *
 80     00    *             7.52=   38.10       9D          *
 80     00    *             -28.08=   32.59     1B=8B       *
\end{verbatim}

**** Put new script here ****


{\it This is not quite the result we expected. A GET operation produces a
separate entry for every HA-Scan. What we see here is the compact notation
produced by NGIDS. This is partly explainable: we have forgotten to
CLEAR the internal flag list of NFLAG before doing the GET operation
(remember that each GET {\em adds} entries to the list).
But we must conclude that GET has not added any entries to it in this
case, so there must be something wrong.....}


%================================================= Standard subsection =======
\newpage
\section{NFLAG option SHOW}
\label{.show}

{\bf The option SHOW} allows the user to inspect the contents of a SCN-file:
the general layout, the contents of headers at the various levels
(i.e. file header, Sector header and Scan header), and the uv-data itself.
Until july 1993, this functionality used to be part of the program NSCAN.
Its use is demonstrated in some detail in another section of this Cookbook:
`Description of the NEWSTAR SCN-file'. In this section, we will only
draw attention to the use of the SHOW option to inspect flags that are set
in the uv-data or in the Scan header.


%------------------------------------------- Option SHOW subsubsection ------
\subsection{A short tour of SHOW}
\label{.show-tour}

The NFLAG option SHOW allows the user to inspect the SCN-file at successively
deeper levels: File layout and header, Sector header, Scan header (including
header flags) and finally Scan data (including weights and flags):

**** Put new script here ****

%------------------------------------------- Option SHOW subsubsection ------
\subsection{Inspecting flags per Sector}
\label{.show-sector}

It is possible to inspect the total number of flags for each interferometer
in a Sector (i.e. many consecutive Scans). Each flagged uv-data point is
counted for `one', even if more than one of the 8 flag types have been set.

**** Put new script here ****


%------------------------------------------- Option SHOW subsubsection ------
\subsection{Inspecting flags per Scan}
\label{.show-scan}

The flag-bytes of individual uv-data can be inspected too:

**** Put new script here ****

The flag code is the sum of the codes of all the flag types that are set
for a particular uv-data sample. Thus, for some only the flag of type
MANUAL (80) is set, for others only type SHADOW(08), and a few have
them both (88).

Note that the weights of all uv-data that are flagged (any flag type) have
been made negative. This is used internally, to speed up NEWSTAR programs.

%------------------------------------------- Option SHOW subsubsection ------
\subsection{Inspecting flags in the Scan header}
\label{.show-header}


If flags are set in the Scan header, they may be inspected as follows:

**** Put new script here ****

The flag (type MANUAL, code 80) is visible in BITS.


%================================================= Special subsection ========
\newpage
\section{Ideas for improvements}
\label{.improve}

Efficient data editing is now the highest priority for NEWSTAR. The program
NFLAG, and the interactive flagging option in the program NGIDS, have been
created in a very short time, to meet the needs of important new NEWSTAR users
(notably the WHISP project and the operational groups in Westerbork and
Dwingeloo). Although some of the most important options have been implemented
(and partly tested), there is ample room for improvement.

The user is invited to add ideas and suggestions to the list below. This list
is not necessarily in order of importance, and certain things will not get done
very soon. Some ideas can be implemented rapidly: the program NFLAG has been
designed as a frame-work, in which a wide range of flagging operations can be
relatively easily provided and/or modified. Other improvements are clearly
desirable, but the required effort may not be justifiable for an interim
package (waiting for AIPS++). However, the list below will play an important
role in determining the list of priorities for the fall of 1993.

\begin{itemize}

\item
Flagging operation ARESID (comparable to RRESID at level OPERATION\_0).
The Align/Selfcal residue for each
indivudual uv-data point is used as a criterion to flag that data point.
The advantage over RRESID is that all interferometers are involved.
The disadvantage is that a good source model is required.
The most urgent application is the treatment of calibrator observations
(for which good models exist) by the operational NFRA groups.

\item
Flagging operation DCOFFSET, to detect DC offsets caused by the ageing
correlator.

\item
Automatic batch procedures (ABP) for certain flagging operations,
for use by the operational groups in Westerbork and Dwingeloo.
Ideally, they should lead to a succinct Quality Report for a particular
observation, possibly with suggestions for action if necessary (Expert
System).

\item
Easy availability of the necessary information to set limit values for
flagging criteria. For instance, the overall Redundancy noise for the
entire Sector when using the operation RRESID.

\item
The possibility of using `general' limit values for flagging criteria,
instead of explicit numeric values.
For instance: upper limit is 2*REDNS, or 3.5*SIGMA etc.

\item
More detailed inspection tools for the flags that are set in the uv-data. At
this moment there is only the possibility to show all flags (of any type) per
interferometer for a particular Sector. This is only useful to see whether or
not a certain interferometer has been flagged. Moreover, it is cumbersome to
have to leave NFLAG option FLAG, and to enter NFLAG option SHOW.

\item
A SHOW option to view the internal flag list. Perhaps this could be
gradually extended to a full set of tools to edit the flag list, comparable
to the tools for a list of source model components.

\item
More user-friendly interactive flagging with the NGIDS display.
This includes better use of available GIDS functionality to
indicate data to be flagged, and improvement of the display of vital
reference data (ifr, baseline, HA, channel, value etc) at all times.

\item
At this moment, the NFLAG flagging operations work directly on the uv-data
in the SCN-file. It might be preferable to collect the flags in the internal
flag list first, to be copied to the uv-data (PUT) or an external file after
inspection.

\item
Reading the uv-data into NGIDS directly from the SCN-file.
The present implementation requires the intermediairy steps via the program
NMAP and the WMP file. This was easier to implement, but is more cumbersome
for the user. Ideally, the user would of course like to interact directly
with the uv-data in the SCN-file, in both directions (like in AIPS TVFLAG).

\item
In some ways, it may be easier to determine which uv-data are `bad' from the
the line graphics plots produced by NPLOT.

\item
Filtering operations on frequency-spectra, comparable to the AIPS option UVLIN.
The latter simply calculates and subtracts the average from the spectrum of
each individual uv-sample, to `remove' the continuum, and thus to enhance the
line. This technique could be
refined to subtracting low-order polynomials from the spectra.
Moreover, the effects of strong sources that are far outside the primary beam
could be filtered out by removing high-frequency ripples from the spectra.

\item
\item
\item
\item
\item
\item
\end{itemize}


	NB: The concept of a flag-file, which may also be edited, is
potentially quite powerful. The information in such a flag-file
specifies flag type, and ranges for frequency channel, HA, ifr and
polarisation. Since these ranges may be wild-cards (*), the flag-files
may be used to copy flags from a calibrator observation to a real observation.

	The use of data flags by NEWSTAR programs is entirely analogous
to the use of uv-data corrections: they are applied `on-the-fly'
whenever uv-data are read  in for processing. The default is that all
8 flag types are tested for. But the user may specify that one or more
flag types are to be ignored, by means of the general (NGEN) keyword
UFLAG. This is of course analogous to the use of the NGEN keywords
APPLY and DE\_APPLY for on-the-fly corrections.

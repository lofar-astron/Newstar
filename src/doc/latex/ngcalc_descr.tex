%       JPH 940916      Make compilable 
%       JPH 950213      Fix figure reference
%       JPH 960502      In extracting, each new input observation produces a new
%                        output group.
%       JPH 960503      New sections .loops, delete, .merge, .combine   
% 
% ngcalc_descr.tex 
% 
\newcommand{\noi}{\noindent} 
\newcommand{\bi}{\begin{itemize}} 
\newcommand{\ei}{\end{itemize}} 

\chapter{The Program NGCALC} 
\tableofcontents 


\section{ General} 
\label{.general} 

        NGCALC is a program for extracting information from .SCN-file data and
then manipulating it in a great variety of ways. The data may be either
interferometer-based (visibilities) or telescope-based (e.g. corrections).

        This document is still far from complete, but it is hoped that it gives
you somethging of a handle to start exploring on your own the many
possibilities that NGCALC offers.


\section{ The .NGF file} 
\label{.ngf.file} 

        The extracted data are organised and manipulated in a special type of
file, the {\em .NGF file}. 

        The .NGF file contains a collection of {\em cuts} i.e. vectors
representing the values of some quantity as a function of either hour angle or
frequency. (You may still come across the term {\em plots} for these cuts, -
and corresponding the the .NGF file may be referred to as a {\em plot file} -,
but this unfortunate terminology is being phased out.) Each cut has a header
containing relevant parameters such as interferometer designation, bandwidth,
center frequency, hour-angle range etc.  

        The cuts are organised in an index structure similar to that for
sectors in a .SCN file and maps in the .WMP file. The order of the indices is 

\verb/  <grp>.<fld>.<chn>.<pol>.<iort>.<seq>/

For each cut, the index is derived automatically from the index of the sector
from which the data are taken. This is shown schematically in 
\figref{.ngf.scn.indices} and explained below: 

\input{ngf_scn_indices.cap} 

\bi 
\item   The \verb/<grp>/ ({\em group}\/) numbers are automatically assigned in
sequential order. A \verb/group/ generally contains all cuts produced in a
single run of NGCALC. In extracting data from a .SCN file, different input
groups and/or observations are mapped to different output groups. 
% 
\item   \verb/<fld>/ ({\em field}\/) and 
% 
\item   \verb/<chn>/ ({\em channel}\/) are those for the data in the .SCN file
from which the cut was made. 
% 
\item   \verb/<pol>/ ({\em polarisation}\/) is a fixed code: 

\begin{tabular}{lllll} 
        &for interferometer-based data: &0=XX,  &1=XY, 2=YX,    &3=YY; \\ 
        &for telescope-based data:      &0=X,   &               &1=Y. 
\end{tabular} 

\item   \verb/<iort>/ ({\em interferometer}\/ for interferometer-based data) is
a sequence number whose only function is to distinguish cuts for different
interferometers. In general, you will use this index only in specifying
\Textref{loops}{.loops} to process interferometers one by one; for {\em
selecting}\/ interferometers, the \whichref{SELECT\_IFRS parameter}{} is more
convenient 
% 
\item   \verb/<iort>/ ({\em telecsope}\/ for telecope-based data) is analogous
to that for interferometers and the remarks made above on its use apply here as
well; in particular, the index value does {\em not}\/ necessarily correpond to
the telescope number. 
% 
\item   \verb/<seq>/ ({\em sequence number}\/) is a number making it possible
to have several different kinds of cut with all preceding indices identical. 
% 
\ei 


\section{ Overview of the parameter interface } 


\input{ngcalc_interface.cap} 


\section{ General features of the parameter interface}
\label{.general.features}

\subsection{ Special use of NGF\_LOOPS} 
\label{.loops}

        In certain operations, sets of input cuts are combined to produce some
output, e.g. some statistic as in the \verb/CALC/ option, or a new cut as in
the \verb/MERGE/ option. To process such sets per interferometer, the
\verb/NGF_LOOPS/ parameter may be used to loop over the interferometers one by
one by specifying an increment for the \Textref{\verb/iort/ index}{.ngf.file},
e.g.

\verb/  NGF_LOOPS = 65, ....1/

If you do not know the number of interferometers, you may have all of them
processed by their number as 91 (the largest possible number) or higher.


\subsection{ The DELETE and COPY options}
\label{.delete}

        It is rather easy to make mistakes in defining operations that combine
cuts. To get rid of the resultant clutter in an .NGF file, NGCALC offers the
option to \verb/DELETE/ (sets of) cuts. The data are not actually deleted but
simply disconnected from the index structure which makes them invisible; be
careful: this process is {\em irrevrsible}!

        Deleted data may still occupy more disk space than you want. The
\verb/COPY/ operation may be used to create a new .NGF file from which the
deleted data along with the corresponding index structures are actually
removed.  The indices of valid data are copied unchanged, so e.g. the group
numbers in the output file may form a non-contiguous series.


\subsection{New cuts created by combining input cuts}
\label{.combine}

        Several options, such as \verb/MERGE/ and \verb/COMBINE/, combine data
from multiple input cuts to create new cuts. In such cases, NGCALC tries to
assign sensible values to the output cut's header parameters, but one cannot
rely blindly on them.

        The \verb/TRANS/ and \verb/BASE/ operations transpose data in the
three-dimensional hour-angle/baseline/frequency-channel data cube. The cuts for
the transposed data mostly use the same header parameters as the original cuts
and some educated guessing may be necessary to figure out what these parameters
mean. 


\subsection{ Inspecting the contents of an .NGF file} 
\label{.inspect} 

        The \verb/BRIEF/ and \verb/FULL/ options can be used to inspect the
composition of your file. \verb/BRIEF/ provides a quick survey, \verb/FULL/
gives details per individual cuts. Since the number of cuts in a file may be
quite large, the recommended way to explore a .NGF file is to start with 
\verb/BRIEF/ and use \verb/FULL/ only to get details about one or a few cuts. 

%script

\section{ Extracting data from a .SCN file} 
\label{.extract} 

        The first step in using NGCALC is to extract the necessary data from
one or more .SCN files and store it in an .NGF file. The number of output cuts
equals the number of input sectors times the numer of selected interferometers;
so in cases where the former is already considerable you should think carefully
about what you need. 

The mapping of the .SCN-file sector indices to .NGF-file cut indices is shown
schematically in a diagram \figref{.ngf.scn.indices}. 

%        A typical extraction run is shown below: 

%script


\section{ Merging cuts}
\label{.merge}

        The \verb/MERGE/ option enables you to merge a set of cuts into a new
one; the \verb/NGF_LOOPS/ parameter may be used to execute a whole series of
mergers in a single operation.

        In the merge algorithm, the input cut data are sorted in hour-angle
bins .25 deg wide and averaged per bin. The resulting output is stored in a
contiguous sequence of points equidistant at .25 deg, long enough to hold the
data; see \figref{.ngcalc.merge}.

\input{ngcalc_merge.cap}

        The output cuts will be indexed using the indices of the input cuts,
changing GRP to a new number and SEQ to 0. It is recommended that you check the
output using the \Textref{\verb/BRIEF/ or \verb/FULL/}{.inspect} options to
ascertain what indices NGCALC has assigned.



.c+ 
%============================================= Standard subsection ======= 
\section{Overview of NGCALC options} 
\label{.overview} 

The NGCALC keyword {\bf ACTION} can have the following responses: 

\begin{itemize} 

\item {\bf NODE:} switch data node 

\item {\bf EXTRACT:} extract information from SCN file into NGF file 

\item {\bf SHOW:} show information in NGF file 

\item {\bf BRIEF:} show one line information of plots in NGF file 

\item {\bf MERGE:} merge a number of plot sets into a single NGF file 

\item {\bf COMBINE:} combine info in NGF file(s) into new NGF file 

\item {\bf TRANS:} interchange frequency and HA axes (rough version, in which
bands are 
    translated into HA's and vv 

\item {\bf CALC:} do some calculation on an NGF file 

\item {\bf COPY:} copy info from other NGF file 

\item {\bf MONGO:} produce NGF file info into a MONGO readable file 

\item {\bf PLOT:} plot NGF file(s) 

\item {\bf DELETE:} delete NGF file(s) 

\item {\bf CVX:} convert NGF file from other machine's format to local format 

\item {\bf NVS:} update to latest NGF file format 

\item {\bf QUIT:} leave the program 
\end{itemize} 


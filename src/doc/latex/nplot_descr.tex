% 
% @(#) nplot_descr.tex  v1.2 04/08/93 JEN 
% 
% History: 
%       JPH 940916      \ref --> \textref 
%       HjV 950615      Separate parts by indenting text or by using two blank 
%                       lines
%       JPH 960429      Add NOTE. Missing \ on \textref. 
%
% NOTE:
%       Text terminated after PLUVO by a '.c+' marker because currently there 
%       is nothing relevant beond that point
%
\chapter{The Program NPLOT} 

\tableofcontents  


%============================================================================== 
\section{Overview of NPLOT options} 
\label{.options} 

The program NPLOT has the following main options: 

\begin{itemize} 
\item {\bf MAP:}        Plot map(s) of various types (from WMP-file) 
\item {\bf DATA:}       Plot uv-data from SCN-file 
\item {\bf MODEL:}      Plot uv-model from SCN-file  
        (i.e. the uv-representation of a Selfcal model,  
        calculated for the uv-coordinates of the uv-data in the SCN-file) 
\item {\bf TELESCOPE:}  Plot telescope errors 
\item {\bf RESIDUAL:}   Plot interferometer residuals 
        (i.e. the residual scatter of redundant spacing data after a 
        Redundancy solution, or the residual differences with the  
        source model after a Selfcal solution. 
\item {\bf QUIT:}       Quit the program NPLOT 
\end{itemize} 

%============================================================================== 
\section{NPLOT output} 
\label{.output} 

After selecting an NPLOT option, the {\bf PLOTTER} question is asked: 

\skeyword{PLOTTER} 
\sprompt{(QMS,QMSP,REGIS,FREGIS,EPS,EPP,PSL,PSP,...)} 
\sdefault{= PSP:} 
\suser{?} 

Produces QMS formatted files, for which currenlty no printers are available\\ 
{\tt  QMS    QMS laser printer in landscape orientation}\\ 
{\tt  QMSP   QMS laser printer in portrait orientation}\\ 
Produces output on VT terminal:\\ 
{\tt  REGIS  graphics VT terminal (e.g. VT330)}\\ 
{\tt  FREGIS (*) REGIS to file}\\ 
Produces PostScript files on disk (which may fill up quickly!):\\ 
{\tt  EPS    encapsulated PostScript for use in textprocessors etc}\\ 
{\tt  EPP    EPS in portrait mode}\\ 
{\tt  EAL    encapsulated PostScript A3 plotter landscape}\\ 
{\tt  EAP    EPS A3 in portrait mode}\\ 
Produces PostScript files, and prints them automatically (if you are lucky)\\ 
on the New PostScript laser printer in the Dwingeloo computer room:\\ 
{\tt  PSL    Postscript (do not use halftone: slow!)}\\ 
{\tt  PSP    PostScript in portrait mode}\\ 
{\tt  PAL    Postscript A3 (do not use halftone: slow!)}\\ 
{\tt  PAP    PostScript A3 in portrait mode}\\ 
Bit maps:\\ 
{\tt  BIT1   (*) bitmap for 100 dpi}\\ 
{\tt  BIT2   (*) bitmap for 200 dpi}\\ 
{\tt  BIT3   (*) bitmap for 300 bpi}\\ 
Produces a X11-plot on the Workstation (or X-terminal) screen.\\ 
{\tt  X11    X11 terminal}\\ 

{\tt (*) = not implemented yet.} 

All plot file names start with the 3-4 letters of the selected option (e.g.
EPS), followed by a unique combination of characters based on the time and the
date. All plot files have the extension .PLT. 

%============================================================================== 
\section{Plotting a map from a WMP file} 
\label{.map} 

NPLOT option MAP can be used to make {\bf contour plots} or {\bf greyscale
plots} of 2-dimensional pixel-arrays stored in a  WMP-file.  These may be maps,
antenna-patterns, residuals etc, but also gridded uv-data (see the description
of the program NMAP, and the WMP file).  They can also be displayed (and
analysed) as a color image on the X11 screen with the program NGIDS. 


*** Put new script here **** 


The plot is now made, and sent to the laser plotter automatically if that has
been specified. The program will then prompt the user for the next plot  to be
specified. 


**** Put new script here **** 


It is also possible to make `ruled-surface' plots, halftone plots and plots in
polar coordinates (gridded uv-data). 


%============================================================================== 
\section{Plotting uv-data from a SCN-file} 
\label{.data} 


%----------------------------------------------------------------------------- 
\subsection{NORMAL: interferometers vs time (HA)} 
\label{.data.normal} 

**** Put new script here **** 


The plot is now made, and sent to the laser plotter automatically if that has
been specified. The program will then prompt the user for the next plot  to be
specified. 


**** Put new script here **** 


%----------------------------------------------------------------------------- 
\subsection{PLUVO: frequency channels vs time (HA)} 
\label{.pluvo} 


**** Put new script here **** 


Etc... 

.c+ 
%============================================================================== 
\section{Plotting a uv-model from a SCN-file} 
\label{.model} 

The same as plotting \textref{uv data}{.data}. 

**** Put new script here **** 

The plot is now made, and sent to the laser plotter automatically if that has
been specified. The program will then prompt the user for the next plot  to be
specified. 

**** Put new script here **** 


%============================================================================== 
\section{Plotting telescope corrections} 
\label{.tel} 

The telescope corrections are stored in the HA-Scan headers of the SCN-file.
Plotted are the {\bf total} telescope corrections: Redundancy (REDC) plus Align
(ALGC) plus `other' (OTHC). 


%----------------------------------------------------------------------------- 
\subsection{NORMAL: telescopes vs time (HA)} 
\label{.tel.normal} 

**** Put new script here **** 

The plot is now made, and sent to the laser plotter automatically if that has
been specified. The program will then prompt the user for the next plot  to be
specified. 

**** Put new script here **** 


%----------------------------------------------------------------------------- 
\subsection{PLUVO: frequ channels vs time (HA)} 
\label{.tel.pluvo} 


**** Put new script here **** 

Etc... 

%============================================================================== 

\section{Plotting NCALIB residuals from SCN-file} 
\label{.residuals} 


The residuals are calculated by going though the SCN-file and  subtracting a
uv-model from the uv-data (Selfcal rediduals), or the average of redundant
spacings (Redundancy residuals). 

%----------------------------------------------------------------------------- 
\subsection{Redundancy residuals} 
\label{.redun.residuals} 

**** Put new script here **** 

The plot is now made, and sent to the laser plotter automatically if that has
been specified. The program will then prompt the user for the next plot  to be
specified. 

**** Put new script here **** 


%----------------------------------------------------------------------------- 
\subsection{Selfcal residuals} 
\label{.selfcal.residuals} 

**** Put new script here **** 

The plot is now made, and sent to the laser plotter automatically if that has
been specified. The program will then prompt the user for the next plot  to be
specified. 

**** Put new script here **** 





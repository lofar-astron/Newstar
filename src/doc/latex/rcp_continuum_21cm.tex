%
% @(#) rcp_continuum_21cm.tex  v1.2 04/08/93  AGB
%
% History:
%       JPH 940914      Comment out 'processing appetizer'
%
%
\chapter{Processing recipe: Continuum measurements at 21 cm} 
\tableofcontents 

Author: A.G. de Bruyn 

%=============================================== Standard subsection ======
\section{Scope of the recipe} 
\label{.scope} 

21 cm 

6 cm 

49/92 cm ........... 

%---------------------------------------------------------------------------
%  If possible, supply an idea for an "appetizer" picture, which gives an
%  idea of what can be achieved with this recipe.

%\begin{figure}[htbp]
%\vspace{10cm}
%\caption[004 processing appetizer:...]         % appears in listoffigures
%{\label{.appetizer}}
%\begin{center}\parbox{0.9\textwidth}{\it       % actual caption if long
%004 processing appetizer:...
%\\This result was obtained in the following way:
%\\Note the following features:
%}\end{center}
%\end{figure}


%=============================================== Standard subsection ======
\section{Introduction and background} 
\label{.intro} 


%=============================================== Standard subsection ======
\section{Summary of the recipe for 21cm} 
\label{.summary} 

The following is a step-by-step summary of the processing recipe. For some of
these steps, more detail is provided below. 

\begin{enumerate} 
\item {\bf Load your data:} from tape or optical disk (NSCAN, option ....). See
also Recipe ``Reading data in Dwingeloo''. 
\item {\bf Inspect the data file layout:} (NSCAN, option ...). 
\item {\bf ... :} 
\item {\bf ... :} 
\item {\bf ... :} 
\item {\bf ... :} 
\end{enumerate} 


%=============================================== Standard subsection ======
\section{More details for some of the steps} 
\label{.detail} 

%-------------------------------------------------------------------------
\subsection{Load your data} 
\label{.detail.load} 

%-------------------------------------------------------------------------
\subsection{Inspect the data file layout} 
\label{.detail.layout} 

%=======================================================================
%=======================================================================

%=============================================== Extra subsection ======
\section{Processing 6cm continuum data} 
\label{.6cm} 


%=============================================== Extra subsection ======
\section{Processing 49/92cm continuum data} 
\label{.49/92cm} 

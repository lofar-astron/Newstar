%
% @(#) rcp_linear_polarisation.tex  v1.2 04/08/93  RGS
%
\chapter{Processing recipe: Linear Polarisation}

Author: R.G. Strom 
%       JPH 951107      Break a paragraph that was too long for ndoc
%       JPH 960326      Furtehr fix for this


\tableofcontents

%=============================================== Standard subsection ======
\section{Scope of the recipe}
\label{.scope}


For obtaining polarization maps from standard continuum observations (all four
dipole combinations must be present). See also the description of the program
NCALIB, part 3.

%---------------------------------------------------------------------------

%  If possible, supply an idea for an "appetizer" picture, which gives an
%  idea of what can be achieved with this recipe.


%=============================================== Standard subsection ======
\section{Introduction and background}
\label{.intro}

In a normal WSRT continuum observation, four dipole combinations (XX, XY, etc.)
are measured for each fixed-movable (numbered-lettered telescope) baseline.
Since mid-1983, the standard configuration is for the X-dipoles in all
telescopes to be set parallel (hence the Y-dipoles are also). A polarization
code is recorded for each observation, which gives the X-dipole position angle
in the fixed and movable telescopes, in steps of $45^\circ$. The usual setting
for the X-dipoles is $90^\circ$ (i.e., directed toward increasing RA), which
gives a polarization code of 22. This results in the following definition of
the Stokes parameters: $$I={1\over2}(XX+YY)$$ $$Q={1\over2}(YY-XX)$$
$$U={1\over2}(YX-XY)$$ $$V=-{i\over2}(XY+YX)$$ (where $i=\sqrt{-1}$). In the
remainder of this description it will be assumed that the observation being
corrected has code 22, and of course that all four dipole combinations have
good data.

{\em Most continuum observations made before mid-1983 were done with the
dipoles in the movable telescopes rotated by $45^\circ$ to the fixed ones,
usually with polarization code 21 [or in any event with the digits differing by
an odd number]. For analyzing such observations, the user should consult with
the WSRT User Service in Dwingeloo. From time to time, calibration sources are
still observed in code 21 [generally indicated by `$+\times$' appended to the
source name]. In almost all line observations, only one or two of the dipole
combinations (XX, YY) are measured; they are of course unsuitable for
determining source polarization.}

The determination of the (observed) Stokes $Q$ depends, as shown by the first
two equations above, upon how well the gains of the XX and YY channels can be
ascertained (since usually, $I>>Q$). For the $U$ and $V$ determinations, the
critical parameters are the dipole setting or orthogonality, and the
ellipticity. Moreover, since $V$ is usually small (i.e., $V<<U$) its
determination will also depend upon the quality of the XY and YX gains. The
other parameter required to correct the XY and YX combinations is the $X-Y$
phase difference. The correction is usually small (since it should have been
determined and applied online in Westerbork) and has to be determined from an
observation of a polarized source, or a special `crossed' (code 21) calibration
source measurement. Finally, it may be necessary to correct for Faraday
rotation in the ionosphere (this is usually only necessary at 49 and 92 cm),
and for variations in the instrumental polarization if there is significant
emission beyond the central part of the primary beam.

%=============================================== Standard subsection ======
\section{Summary of the recipe}
\label{.summary}

The following is a step-by-step summary of the processing recipe. For some of
these steps, more detail is provided below.

\begin{enumerate}
\item {\bf Load your data:} from tape or optical disk (NSCAN, option LOAD) See
also Recipe ``Reading data in Dwingeloo''.
\item {\bf Inspect the data file layout:} (NSCAN, option SHOW)
\item {\bf Determine the instrumental gain and phase corrections for a strong
calibrator with known polarisation:} (NCALIB, option REDUN.)
\item {\bf Determine or set the instrumental polarization corrections for an
unpolarised calibrator:} (NCALIB, option POLAR.)
\item {\bf Map and inspect Stokes $Q, U, V$ for the calibrator (optional):}
(NMAP, followed by NPLOT or NGIDS.)
\item {\bf Copy (or set by hand) the instrumental polarization corrections for
the observed field:} (NCALIB, option POLAR COPY.)
\item {\bf Determine, and apply if necessary, the correction for ionospheric
Faraday rotation (optional, usually only necessary at 49/92 cm):} (NCALIB,
option SET FARADAY.)
\item {\bf Map and inspect $I, Q, U, V$ images of field:} (NMAP, followed by
NPLOT or NGIDS)
\item {\bf Primary beam correction for each Stokes parameter (may only be
necessary if source extends over large fraction of primary beam):} (NCALIB,
options BEAM\_FACTORS (for Stokes $I$) and INPOL* (where * = $Q, U, or iV$ for
the other three Stokes parameters).)
\item {\bf Map the final corrected image in all Stokes parameters:} (NMAP)
\end{enumerate}


%=============================================== Standard subsection ======
\section{More details for some of the steps}
\label{.detail}

%-------------------------------------------------------------------------
\subsection{Load your data}

        Remember that in addition to your own observation, you will probably
want to load one or more calibration source measurements. Choose an unpolarized
calibrator (3C 147 is probably the best) observed within a day of your
observation {\it and with the same instrumental settings} (load several
calibrators if you want to check on the repeatability of the solution), and a
polarized source (like 3C 286, except at 92 cm) to determine the $X-Y$ phase
difference. To locate calibration measurements, use ARCQUERY (see Appendix B).
General information on the use of external calibrators can be found in Recipe
13 (which refers to NCALIB option SET COPY), while loading data is described in
Recipe 5 (which refers to NSCAN option LOAD).

%-------------------------------------------------------------------------
\subsection{Inspect the data file layout}

        Make certain, in particular, that all four polarization combinations
(XX, XY, etc.) are present in all observations to be used, and that the
calibrators are consistent with the observation to be corrected (frequencies
and bandwidths should be the same, although a change in spacing shouldn't make
any difference). If the time between calibrator and observation is more than a
day or two, it is also advisable to check that no frontends have been changed
(consult the logbook or reduction group). Inspect data for interference (which
may be polarized -- check the XY and YX combinations in particular) and other
defects. For more information on inspecting data files, see NSCAN option SHOW.

%-------------------------------------------------------------------------
\subsection{Determine the instrumental polarization}

Before determining the instrumental polarization using a calibrator, redundancy
and selfcal solutions must be applied (use SELFCAL and ALIGN in the NCALIB
option REDUN). This will also provide a useful check of the data quality. Note
that at 92 cm (and in some cases 49 cm) there are background sources which may
have to be included in the model used for ALIGN. Having run the ALIGN solution,
NCALIB can be used to calculate the instrumental polarization (POLAR\_OPTION:
CALC) and examine the result (POLAR\_OPTION: SHOW). Under normal conditions,
the orthogonalities and positions of most dipoles should be under $1^\circ$,
and the ellipticities generally under 1\%. Large values (more than a few
degrees or percent) probably indicate an instrumental problem (bad data) and
require further investigation. Run a solution on a different calibrator as a
check. Deviant points can be changed by hand (POLAR\_OPTION: EDIT). Tables of
instrumental polarization are also generated from time to time in Westerbork,
and the values can be entered by hand (POLAR\_OPTION: SET), or used to
cross-check the solution from NCALIB.

As a check on the instrumental polarization thus generated, it may be useful to
make a map of an unpolarized calibrator if one has been observed for a few
hours or more during the same period (a shorter observation could also be used,
but the map might prove difficult to interpret). Make sure that SELFCAL and
ALIGN corrections have been successfully applied, and then look at the Q, U and
V maps (made with NMAP). Ideally, they should be zero; the residual as a
fraction of the flux density is an indication of the error which will be
present in the polarization map of your observation. (If your source is very
extended, however, the polarization error pattern generated by a point source
may be misleading).

Finally, we have to determine (or at least check) the $X-Y$ phase difference.
This is best done using a linearly polarized calibrator (strictly speaking, a
source with strong U signal). The method assumes that V is much smaller than U
(since $YX=U+iV$, a nonnegligible V affects the $X-Y$ phase), which is usually
the case. The correction can be calculated (VZERO\_OPTION: CALC, ASK, etc.) and
applied to the data in several ways. Usually, the XY and YX phase zeroes have
been determined and applied on-line to sufficient accuracy for most
polarization maps, so the correction should be a few degrees or so. However,
there have been instances where the difference was as much as $30^\circ$, so it
is advisable to check. If various frequency channels are used, the phase
difference should be calculated for each one separately, as the correction is
usually frequency dependent.

%-------------------------------------------------------------------------
\subsection{Correct the observation for instrumental polarization}

If you are happy with the instrumental parameters as applied to the
calibrator(s), the values can be copied to your observation using NCALIB (under
POLAR\_OPTION select COPY). The corrections will then be applied to the data
when making a map with NMAP. If you have run selfcal (ALIGN) on your XX and YY
polarizations using a source model based on Stokes I ($XX+YY$), remember that
solving for gain (keyword SOLVE) could remove most of the Q signal.

%-------------------------------------------------------------------------
\subsection{Correct for ionospheric Faraday rotation}

At 49 and 92 cm, Faraday rotation in the ionosphere can be considerable. The
effect is to change the position angle (p.a.) of the plane of linear
polarization (for the WSRT the p.a. will always {\it increase}, so the shift is
systematic). Since the ionosphere changes throughout the day, the amount of
rotation may vary during an observation. This has two consequences: the average
position angle will be increased, and more seriously, if the differential
rotation exceeds about one radian, there will be decorrelation of the polarized
signal and distortion of the Q and U maps.

To correct for ionospheric Faraday, the reduction group must be consulted to
generate the predicted rotation using ionosonde data and information about the
observation in question (the date, time and source position are needed, and the
correction should be done for a frequency of 1 GHz). This will produce a table
with hourly values of various parameters, including the source HA and the
amount of rotation. The HA and rotation at 1 GHz (both in degrees) have to be
entered in an ASCII file (see the description in NCALIB, option FARADAY under
SET\_OPTION), and the correction is applied using FARADAY. Note that the HAs do
not have to be hourly, or even regularly spaced: at night there may be little
change in the Faraday rotation, and a single correction might suffice.

%=============================================== Extra subsubsection ======
\subsection{Correct for primary beam instrumental polarization}
\label{.beam}

The instrumental polarization determined from a calibrator is, strictly
speaking, only correct for a source at the beam center, though at most
frequencies the variation within the half-power primary beam is small. For
polarization mapping over large fields, there is a separate correction for
off-axis instrumental polarization (NCALIB: NMODEL - BEAM).

%-------------------------------------------------------------------------
\subsection{How do I know that my data are correct? -- additional hints.}
\label{.hints}

Errors; some error patterns; effect of source extent.

Ionospheric effects and the I map.

What can be done if a polarization combination is missing?

Etc.


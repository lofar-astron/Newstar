%
% @(#) rcp_mosaic_21cm.tex  v1.2 04/08/93  TG
%
% History:
%       JPH 940914      Comment out 'processing appetizer'
%
%
\chapter{Processing recipe: Mosaicing at 21 cm}
\tableofcontents

Author: T. Ghosh

%=============================================== Standard subsection ======
\section{Scope of the recipe}
\label{.scope}

This recipe suggests a way for making a 21-cm continuum mosaic map. In this
example, a particular field was chosen where there were many previously-known
radio sources within the region to be mapped. Hence, we knew that there was
sufficient flux-density for almost all of the pointing centres to allow  us to
run {\it selfcal}. If the user is not sure of having enough SNR for this, a
modified version can be used.  Hence, consider this recipe as just a starting
point, and {\it add salt to your taste}.

%---------------------------------------------------------------------------
%  If possible, supply an idea for an "appetizer" picture, which gives an
%  idea of what can be achieved with this recipe.

%\begin{figure}[htbp]
%\vspace{10cm}
%\caption[003 processing appetizer:...]         % appears in listoffigures
%{\label{.appetizer}}
%\begin{center}\parbox{0.9\textwidth}{\it       % actual caption if long
%003 processing appetizer:...
%\\This result was obtained in the following way:
%\\Note the following features:
%}\end{center}
%\end{figure}


%=============================================== Standard subsection ======
\section{Introduction and background}
\label{.intro}

In the mosaic mode of observation, during one 12-hour period the telescopes,
along with the fringe-stopping and the delay-tracking centres, cycle through a
grid of pointings a number of times. In Fig 003.1, we used a 13 x 5 grid of
these so called, {\it pointing centres}. There were about 21 cuts (or spokes),
each containing two scans of 10-s duration for each pointing centre. The total
40-MHz bandwidth at 21 cm is usually divided into eight 5-MHz channels
(although these are not contiguous). Hence, the data file had the following set
indices (see also Ch 3.):  $$0.0.1-65.1-8.0-20$$

For a particular pointing centre, the reduction methodology is very much the
same as that in Recipe 004. {\bf However, all the maps are to be made using a
common reference point (NMAP-option REFERENCE)}. Usually, for a large number of
pointing centres, book-keeping etc. could be quite difficult. Here, all the
advantages of the Automatic Batch Processing facility (ABP, Appendix D) can be
very easily exploited. In the  following section we detail all the steps,
indicating the pre-ABP, the ABP, and the post ABP stages, Stage I, II and III
respectively, that were used to generate the map in Fig 003.1. These three
stages also coincide with the {\em calibration using a different source}, the
{\em Selfcal-Model formation loop for all the pointing centres}, and the {\em
mosaicing} stages.


%=============================================== Standard subsection ======
\section{Summary of the recipe}
\label{.summary}

{\bf STAGE I:}

\begin{enumerate}
\item {\bf Load your data:} Source data from tape or optical disk (NSCAN). See
also Recipe ``Reading data in Dwingeloo''. Also read the data from observations
of a calibration source, (e.g. 3C147 in the case of Fig 003.1) made just before
or after the {\em source field}.
\item {\bf Inspect the data file and flag bad points mentioned on the WSRT
green sheet :} (NSCAN, option SHOW etc).
\item {\bf Calculate antenna-based phase and gain correction factors for the
Calibrator data:} (NCALIB, option REDUNDANCY, suboption SELFCAL, using a model
file obtained from AGB )
\item {\bf Copy the corrections to the source data:} (NCALIB, option SET,
suboption COPY)
\end{enumerate}

{\bf STAGE II:}

This stage could be batch processed (APPENDIX D). In the following, we first
write a name for each task, which briefly describes the function of the step
too. Within the bracket, we mention the programme to be used, and comment on a
few important input parameters. Depending upon the flux density of the
strongest source in the map of a particular pointing centre, the Stage is
stopped at various {\bf Exit points}.

For each pointing centre:

\begin{enumerate}
\item {\bf Raw:} (NMAP, make a $1024 x 1024$ pixel raw map of real size
$1^\circ .2 x 1^\circ .2$ )
\item {\bf Find1:} (NMODEL, find model components down to a suitable limit e.g.
6 w.u. here)
\end{enumerate}

{\bf Exit point 1:} If the flux density of the strongest component, $S_{h}$ is
less than limit~1 (12~w.u., here) - STOP - Proceed to next pointing centre.

\begin{enumerate}
\item {\bf Upd1:} (NMODEL, Update the model list, delete sources weaker than a
certain limit after updating, e.g. 6 w.u)
\item {\bf Upd2:} (NMODEL, Second iteration of the above)
\item {\bf Self1:} (NCALIB, self-calibrate ONLY the PHASE of the data using
model, Upd2)
\item {\bf Sub1:} (NMAP, make $1^\circ .2 x 1^\circ .2$ map after subtracting
model, Upd2)
\end{enumerate}

{\bf Exit point 2:} If $S_{h} < $  limit~2 (20 w.u. here) - STOP - Proceed to
next pointing centre.

\begin{enumerate}
\item {\bf Find2:} (NMODEL, find model components from map, Sub1 down to a
suitable limit e.g. 5 w.u. here and add to the model, Upd2)
\item {\bf Upd3:} (NMODEL, update the model list, Find2 and  delete sources
weaker than a certain limit e.g. 5 w.u here)
\item {\bf Upd4:} (NMODEL, second iteration of the above)
\item {\bf Self2:} (NCALIB, self-calibrate the data using model, Upd4)
\item {\bf Del1:} (NSCAN, delete scans with selfcal-amplitude and/or phase
noise $ > 2 \sigma$)
\item {\bf Sub2:} (NMAP, make $1^\circ .2 x 1^\circ .2$  map after subtracting
model, Upd4)
\end{enumerate}

{\bf Exit point 3:} If $S_{h} < $  limit~3 (40 w.u. here) - STOP - Proceed to
next pointing centre.

\begin{enumerate}
\item {\bf Find3:} (NMODEL, find model components from map, Sub2 down to a
suitable limit e.g. 2.5 w.u. here and add to the model, Upd4)
\item {\bf Upd5:} (NMODEL, update the model list, Find3 and delete soucres
weaker than a certain limit e.g. 2.5 w.u here)
\item {\bf Upd6:} (NMODEL, second iteration of the above)
\item {\bf Self3:} (NCALIB, self-calibrate the data using model, Upd6)
\item {\bf Del2:} (NSCAN, delete scans with selfcal-amplitude and/or phase
noise $ > 2 \sigma$)
\item {\bf Sub3:} (NMAP, make $ 1^\circ .2 x 1^\circ .2 $ map after subtracting
model, Upd6)
\end{enumerate}

{\bf End of stage II}

At this point, for all the pointing centres, one should have a good model list,
and a residual map. The common model components in the overlapping region can
be checked for positional agreement. These can be cross-checked against any
source within the region whose accurate position is known from the literature.
Usually, agreements within 1 arcsec was achieved at this stage, which was
already 1/10 th of the synthesised beam.

Depending upon the quality of the maps, they can be grouped into four classes.

{\bf A.} Good maps, where noise is already at the theoretical limit, though
there may still be residual sources, and their grating rings and sidelobes.

{\bf B.} Maps containing grating rings from sources beyond the mapped region.

{\bf C.} Maps with bad scans.

{\bf D.} Maps with amplitude or/and Phase calibration problems.

For Class B maps, make larger low-resolution maps and try to include the strong
(offending) source in the model list, and then start again at the beginning of
Stage I (retaining the 'outside sources', though).

For Class C maps, search out the bad data (no easy-solution, sorry), delete it,
and start again.

For Class D maps, there may be spurious sources in the model-list. Plot them,
(NPLOT) and try to locate suspects, delete these, and continue (refer to
sec..... if this doesn't help).

{\bf STAGE III:}

Once all the maps are of Class A-quality, do the following:

\begin{enumerate}
\item {\bf Mkbeam :} (NMAP. Make beams for all the pointing centres)
\item {\bf Clean :} (NCLEAN, option UVC. Clean the entire map with the number
of components = 100, Gain = 0.1 and a Cycle-Depth value of 0.35)
\item {\bf Restore :} (NCLEAN, option UREST. Restore both the model and the
clean components)
\item {\bf Extract :} (NMAP, option FIDDLE, suboption EXTRACT. Write out a
smaller section of this map into one set index of a common map file that will
be the input for the MOSAIC combination, here, inner 700 x 700 pixels of each
pointing centre were used)
\item {\bf Mosaic :} (NMAP, option FIDDLE, suboption MOSC. This part of the
programme corrects the input maps for primary beam attenuation and then
averages them with proper weightage)
\item {\bf Ready to serve:} (NPLOT, NGIDS, write out FITS image and take it to
AIPS ..... use the garnish of your choice)
\end{enumerate}




% nmap_private_intfc.tex

\chapter{ Private parameters for program NMAP}
\tableofcontents


\section{ Overview}
\label{.overview}

	This document contains an overview of the parameter interface of the
program NMAP. The program also uses a number of public interfaces; references
to these are also \textref{listed}{.public}.


	Apart from a number of utility options, the main functions of NMAP are
the making of maps from .SCN files (option MAKE) and manipulating these maps in
various ways (option FIDDLE).

	"Roadmap"-style overviews for the entire parameter interface are shown
in \textref{below}{.diagrams}.

	The remainder of the document describes the individual parameters in
alphabetical order. This description centers on the Help texts, which have been
designed to guide the user to the proper choice at each junction, even if his
knowledge of the overall workings of the program is only superficial.



\section{ Interface diagrams}
\label{.diagrams}

	The first diagram, an synopsis of NMAP's parameter interface, does not
yet exist. Following that diagram several sub-diagrams are planned, of which
those for the FIDDLE and utility options still have to be made.

	\Figref{.nmap.make} shows the parameter interface for MAKE. Two detours
in this interface, for parameters QMAPS and QDATAS, are normally bypassed; they
are shown in \figref{.nmap.make.q}.

	Included in these diagrams are NMAP's public parameters. These are
interleaved with the private ones so they can not be diagrammed as a separate
unit. In fact, the only reason that they are public is that the NCLEAN
\textref{'data clean'}{nclean_descr.data.clean} operation uses them. This use
is indicated on the right-hand side of \figref{.nmap.make} and
\figref{.nmap.make.q}.

\input{nmap_make.cap}
\input{nmap_make_q.cap}
\input{nmap_handle.cap}

\section{ Descriptions of the individual parameters}
\label{.descriptions}

\subsection{ References to public interfaces}
\label{.public}

\input {nmap_private_keys.tef}

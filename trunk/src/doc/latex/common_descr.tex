%
% @(#) common_descr.tex  v1.2 04/08/93 JEN
%
\chapter{Common features of \NEWSTAR programs}
\tableofcontents 

%==============================================================================
\section{Running \NEWSTAR programs under DWARF}
\label{common.descr.dwarf}

All \NEWSTAR programs run under 
the Dwingeloo Westerbork Astronomical Reduction Facility (DWARF). 
This means that the programs communicate with the user
via the DWARF interface, especially the DWARF parameter interface.
DWARF also provides the user with various commands
to modify the environment in which programs will be executed.
This chapter tells you enough of DWARF to be able to run your first
\NEWSTAR programs. 

A full explanation of all the DWARF possibilities can be found in the 
``DWARF User's Guide'', written by Friso Olnon. 


%------------------------------------------------------------------------------
\subsection{DWARF environment set-up}

Before running the programs some symbols and logicals (aliases) should be known.
Ask the System manager about the proper login commands.

The current values of the DWARF environment variables can be viewed:

\scmd{dwv dwarf} \sinline{dwview (see below)}
\sline{  DWARF\$0\_STREAM (program) = 1}
\sline{  DWARF\$0\_CURNODE (program) = 0}
\sline{  DWARF\$0\_ASK (program) = NO}
\sline{  DWARF\$0\_SAVELAST (program) = NO}
\sline{  DWARF\$0\_USERLEVEL (program) = BEGINNER}
\sline{  DWARF\$0\_BELL (program) = ON}
\sline{  DWARF\$0\_MESSAGEDEVICE (program) = TERMINAL}
\sline{  DWARF\$0\_EXTENDSIZE (program) = 64}
\sline{  DWARF\$0\_IOBUFSIZE (program) = 32768}
\sline{  DWARF\$0\_TEST (program) = NO}
\sline{  DWARF\$0\_LOGLEVEL (program) = 4}
\sline{  DWARF\$0\_LOGFATAL (program) = NO}
\sline{  DWARF\$0\_IDENT (local) = 143}
\sline{  DWARF\$0\_IBMODE (local) = INTERACTIVE}

The DWARF environment variables can also be changed by typing:
\scmd{dws dwarf}\sinline{dwspecify (see below)}


%------------------------------------------------------------------------------
\subsection{Running programs}
\label{common.descr.dwarf-run}

Programs are normally run by typing: 

\scmd{exe $<$programname$>$}\sinline{(e.g. exe[cute] NSCAN)}

Note: {\tt exe} stands for execute, and can, and this is preferred, 
also be given as {\tt dwe} or {\tt dwexe}.

The program starts running, and asks questions (parameters, or `keywords') 
that will specify the action of
the program. All questions have some prompt information, and more help on the
specific question can be obtained by typing a `?' as answer. 

The type of answer (value, name, option, yes/no, ...) depends on the question.
Some answer should be a single value, some should be a list separated by commas.
In the case of a numeric list, the answer can also be of the form: start-value
[by increment] [to end-value]
There are three special answers that can always be given:

\begin{itemize}
\item	\# or $\wedge$D ($\wedge$Z) means end-of-file. 
			The general action is to restart the
                          asking of questions at a {\em higher logical level} 
			  in the
                          program. In the case of options it indicates in
                          general the QUIT option.
\item	"" means empty answer. In general this is taken to mean that no
                          answer is given. Depending on the keyword this can
                          mean to by-pass a certain action, or to go back to
                          another level of questions. If the answer is
                          essential, it will be repeated.
\item	* means `all'. If interpretable it is taken as meaning "all
                          possible values", otherwise the question is repeated,
                          or a special default value is taken.
\end{itemize}

In general (but with the exception of obvious interactive programs) all
questions for the program are asked and checked for consistency, availability of
files etc, before the program starts actually executing.

After the program is finished, a log of the program, including the questions and
their answers and all data produced by the program, may be spooled to the
lineprinter (controlled by the common \NEWSTAR keyword {\bf LOG}, see below).
On this log you will see the answers to some questions that were never asked.
The default answers to these hidden questions need normally sufficient, 
and are, therefore, not asked.
Ways to change them are given in below.


%------------------------------------------------------------------------------
\subsection{Streams}
\label{common.descr.DWARF-stream}

It is sometimes useful to run the same program in parallel, or to run a program
regularly with a specified set of answers. To be able to differentiate between
the programs, programs can be run in different streams. For all practical
purposes the program (and its parameters) have different names.
Streamnames can be any alpha-numeric string, although for practical reasons
integers may be preferred by the user:

\sline{`program-name'\$`stream'}\sinline{(e.g. dwe nscan\$5)}

The zero stream has a special meaning in specifying values across streams.


%------------------------------------------------------------------------------
\subsection{Specifying program parameters (keywords)}

In addition to answering questions asked, keyword values can be also be
specified beforehand. 
This is especially handy if you want to run programs in Batch mode,
or you want to make sure that for a set of programs the 
keyword values will be the same.

Keyword values can be specified before a program is run, by using the 
{\tt dws(pecify)} command. To answer all questions, type:

\scmd{dws `program-name'}\sinline{(e.g. dws NSCAN)}

All keywords (and their current values) will now be displayed one after the
other, and may be modified by the user.
The process can be stopped at any time by giving $\wedge$D ($\wedge$Z) as
answer. If you only want to specify a few keyword values only
(and know the name of the keyword), type: 

\scmd{dws `program-name'/nomenu}\sinline{(e.g. dws nscan/nom)}
\sline{: keyword=`value'}\sinline{(e.g. ha-range=-10,10)}
\sline{:}

The prog-name can be replaced  by {\tt `prog-name'\$`stream'} to
specify keyword values for a separate stream (the default stream is in
general \$1).
NB: The program name and/or stream can contain the wildcard $\ast$. 

Some questions are asked more than once in the program, for instance in
an interactive loop. 
To specify all answers beforehand, seperate them in {\tt dws} with a 
semi-colon (;).
However, to make really sure that all answers are given, it is much safer to
use the `Dry-run' or the `Saved-run' method (see below). 

Keywords in DWARF have several levels of default values. The decision 
process is as follows:

- is there a value for this keyword in the currect stream? If yes: use it.
\\- is there a value for this keyword in stream 0? If yes: use it.
\\- is there a value for this keyword in NGEN in this stream? If yes: use it.
\\- is there a value for this keyword in NGEN for stream 0? If yes: use it.
\\- should the program ask the keyword? If no: use program default.
\\- prompt the user for a value and use it.

Note: NGEN is a collection of special \NEWSTAR keywords, described below.

%------------------------------------------------------------------------------
\subsection{Asking all}
\label{common.descr.dwarf-ask}

All questions (including the hidden ones) will be asked if you run the program
as:

\scmd{dwe `name' /ask}\sinline{(e.g. dwe nscan/ask)}

Asking can also be enabled for all runs of all programs by setting
the global DWARF parameter ASK:

\scmd{dws dwarf/nomenu}
\sline{: ask=yes}
\sline{:}


%------------------------------------------------------------------------------
\subsection{Maintenance of keyword values}

From the above it is clear that the variety of possibilities to specify
keyword values also makes the possibility of errors quite large. It is therefore
recommended to make sure that programs are run as intended. To aid in
maintaining the specified answers, run the following regularly:

\scmd{dwv(iew) 'prog-name'[\$`stream']/extern}

shows all current values for the specified program and stream.

\scmd{dwc(lear) `prog-name'[\$`stream']}

clears all current values for the specified program and stream.



%------------------------------------------------------------------------------
\subsection{Dry run.}

A program can have a dry-run, in which all questions are asked and all checks
are done by:

\scmd{dwe `prog-name'[\$`stream']/norun}

In that case (except for obviously interactive program options) the program will
not run, but the answers to all questions will be remembered 
for subsequent runs {\em in that particular stream}.

%------------------------------------------------------------------------------
\subsection{Saved run}

All answers in any program run can be saved for later use by:

\scmd{dwe `prog-name'[\$`stream']/save}

Note: more than one /name can be given at a {\tt dwe} call.

%================================================= Special subsection ========
\section{Some common \NEWSTAR keywords}
\label{common.descr.newstar}


%------------------------------------------------------------------------------
\subsection{NGEN: General keywords}
\label{common.descr.newstar.ngen}

The following keywords (parameters) are used for most \NEWSTAR programs, but
they are usually hidden from the user (default switch /NOASK).
Their current values  can be inspected by:


\scmd{dwv ngen}
\sline{  NGEN\$1\_LOG (program) = SPOOL /NOASK}
\sline{  NGEN\$1\_RUN (program) = YES /NOASK}
\sline{  NGEN\$1\_DATAB (program) = "" /NOASK}
\sline{  NGEN\$1\_INFIX (program) = "" /NOASK}
\sline{  NGEN\$1\_APPLY (program) = * /NOASK}
\sline{  NGEN\$1\_DE\_APPLY (program) = NONE /NOASK}
\sline{  NGEN\$1\_LOOPS (program) = "" /ASK}
\sline{  NGEN\$1\_DELETE\_NODE (program) = NO /ASK}

More information about these keywords can be found in the Summary of
Common keywords in the next section.

Usually, their default values are taken for the NGEN keywords, 
but they may also be specified by the user in a number of ways,
for instance with {\tt dwspecify}:

\scmd{dws `prog-name'[\$`stream']/nomenu}
\sinline{will set the keyword for specified program and stream.}

\scmd{dws `prog-name'\$0/nomenu}
\sinline{will set the keyword for specified program and all streams.}

\scmd{dws ngen[\$`stream']/nomenu}
\sinline{will set the keyword for all programs in the stream.}

\scmd{dws ngen\$0/nomenu}
\sinline{will set the keyword for all programs and streams.}

It is also possible to give (any) keyword values as switches when running
the program:

\scmd{dwe `prog-name'[\$`stream']/keyw[=`value']}\sinline{or}
\scmd{dwe `prog-name'[\$`stream']/nokeyw}

will set the keyword for the specified program and stream

Some examples in the case of NGEN keywords:

\begin{itemize}
\item {\tt  /LOG= spool/yes/no/append}\\
	specify if program log should be
        spooled (the name will be the first
      	three letters of the program name
        (capital), followed by date, time,
        letter (capital), .LOG), made but not
        spooled (name will be PROG-NAME.LOG),
        not made, appended to PROG-NAME.LOG.
\item {\tt  /LOG X /LOG=spool}
\item {\tt  /NOLOG X /LOG=no}
\item {\tt  /RUN}\\				
	run the program
\item {\tt  /NORUN}\\				
	start the program, ask all questions,
        save all the keyword values (as if specified).
	This can be used to prepare a batch run for intance.
\item {\tt  /INFIX= 'node shorthand'}\\	
	specify a part of the node name that is always the same (see later)
\item {\tt  /INFIX X /INFIX=""}
\item {\tt  /NOINFIX X /INFIX=""}
\item {\tt  /APPLY=list-of-options}\\	
	specify the corrections to be applied to data (see later for full
        explanation). The list can contain one or more of:
	ALL, NONE, RED, ALG, OTH, EXT, POL,
        FAR, MOD, IFR, MIFR , NORED, ... NOMIFR
\item {\tt  /APPLY X /APPLY=*}\\
\item {\tt  /NOAPPLY X /APPLY=NONE}\\
\item {\tt  /DE\_APPLY='list-of-options'}\\
\item {\tt  /DE\_APPLY X /DE\_APPLY=NONE}\\
\item {\tt  /NODE\_APPLY X /DE\_APPLY=*}\\(i.e. NO DE\_APPLY)
\end{itemize}

%----------------------------------------------------------------------
\subsection{Some frequently used \NEWSTAR keywords}
\label{common.descr.NEWSTAR-other}

Some other keywords are also used by many \NEWSTAR programs,
but not hidden to the user.
They are usually concerned with the 
selection of data: {\bf SETS, POLARISATION, SELECT\_IFRS, HA\_RANGE,
AREA}. There are also some others. They have been collected into this
`common' section to emphasize their communality, and also to make it
possible to treat some of them in some more detail.
More information about these keywords can be found in the Summary of
Common keywords in the next section.

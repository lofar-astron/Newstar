
% NOTES: 
%       1. This is a "meta"-document, i.e. a document that describes LaTeX 
% commands using \verb and \verbatim. Latex2html understands verbatim 
% environments incompletely and is prone to the mistake of trying to execute 
% commands within them. To prevent this from happening, the escape sequence
",," 
% is used instead of a "\" where necessary. 
%       \verb/ndoc Cook/ preserves these through LaTeX2html and then
substitutes 
% them in the .html file. 
%       \verb/ndoc Print/ removes them before submitting the file to LaTeX. 
% 
% History 
%       JPH 940628      Original 
%       JPH 940719      Fix errors, expand/clarify at various places 
%       JPH 940915      Add caption bug; \ref bug 
%                       Revise Figures section to reflect new methods 
%                       Revise sectioning/subsectioning 
%       JPH 940916      Make compilable 
%       JPH 941027      Add escapes for sequences that were unintendedly 
%                        translated 
%       JPH 941031      Remove keyref 
%       JPH 941110      Shift position of /label in figure declaration 
%       JPH 941121      Replace "\<escape>" by ",," to make robust against 
%                        changes/variants of unix utilities. (Perhaps a more 
%                        "improbable" escape sequence must be found.) 
%                       Use rawhtml environment for script section. 
%       JPH 950213      Fix some '\verb/,,textref\' which cause 'Ouput line too 
%                        long' and 'Undefined control sequence': Replace \ by
'' 
%       HjV 950615      Use $n_l2h iso. ~jph. Correct some typo's 
%       JPH 950824      Replace \s by ,,s. Update sections on referencing 
%                        commands 
%       JPH 951016      LaTeX caption bug. - doc_sources_and_print.cap. 
%       JPH 960326      Line-too-long bug.
%       JPH 960508      Formulas-in-l2h section at end
% 
% 
\newcommand{\bi}{ \begin{itemize} } 
\newcommand{\ei}{ \end{itemize} } 
\newcommand{\noi}{ \noindent} 
\newcommand{\ndoc}{ \verb/ndoc/} 
\newcommand{\ltoh}{ \verb/latex2html/ } 
\newcommand{\tex}{ \verb/.tex/ } 


\chapter{Guide for writing and maintaining \NEWSTAR documents} 

\tableofcontents 

\section{ Summary of \NEWSTAR LaTeX commands } 

        For quick reference use the following list of special LaTeX commands
available in the \NEWSTAR documentation system: 
\\ \\ 
\textref{ascref}{.cross.ref.other} \\ 
\textref{Ascref}{.cross.ref} \\ 
\textref{caption}{.caption} \\ 
\textref{chapter}{.layout} \\ 
\textref{eqref}{.eqref} \\ 
\textref{Eqref}{.eqref} \\ 
\textref{fig}{.caption} \\ 
\textref{figref}{.figref} \\ 
\textref{Figref}{.figref} \\ 
\textref{label}{.cross.ref.latex} \\ 
\textref{ps}{.ps} \\ 
\textref{psref}{.cross.ref.other} \\ 
\textref{Psref}{.cross.ref} \\ 
\textref{ref}{.cross.ref.latex}: {\em Not to be used} \\ 
\textref{tableofcontents}{.layout} \\ 
\textref{textref}{.cross.ref.latex} \\ 
\textref{Textref}{.cross.ref} \\ 
\textref{whichref}{.cross.ref} \\ 


\section{ The \NEWSTAR documentation system} 

        The \NEWSTAR documentation system provides documentation in two forms:
An on-line hypertext system based on the public-domain program "xmosaic" and
standard LATeX documents. Both forms are derived from a single set of LaTeX
source files; for the hypertext form, the public-domain translator \ltoh is
used. 

        Broadly speaking, authors are free to use the standard LaTeX
constructs. In particular, \ltoh handles tabular material, mathematical
formulas and the inclusion of diagrams very well. Writers should be aware,
however, that every bit of math-mode text is translated into a little bitmap
that must be loaded separately for display. This considerably slows down both
the processing by \verb/ndoc Cook/ and the display by xmosaic. 

         Simple macros defined through 
\verb/,,newcommand/ can be used freely, but complicated nesting of macros
should be avoided. 

        To satisfy the specific needs of the \NEWSTAR system and circumvent
some deficiencies of \ltoh, the guidelines given below should be followed. 


\subsection{ LaTeX and ASCII documents} 

        The official format for \NEWSTAR document sources is LaTeX. It is,
however, considered more important that documentation exist at all than that it
have a finished form. {\em Users are strongly invited to submit any bits of
text that they consider to be of potential use to others.} Such contributions
will quickly be integrated with the system and the \NEWSTAR group will assume
responsibility for their further maintenance, including their eventual
conversion to or integration with the LaTeX document collection. 


\section{ Guidelines for writing LaTeX documents} 

\subsection{ General layout} 
\label{.layout} 

\bi 
\item   Preamble 

        The document compiler \ndoc automatically inserts a preamble including
the \verb/,,begin{document}/ and also appends the \verb/,,end{document}/. The
source of a document may add private preamble elements such as 
\verb/,,newcommand/ definitions. Apart from that, the first document line is
the 
\verb/,,chapter/ line: 

\item \verb/,,chapter{<text>}/ 

        The chapter is the basic documentation unit. \ndoc formats the argument 
\verb/<text>/ as a hypertext or printed document title. 


\item \verb/,,tableofcontents/ 

        This creates a standard table of the chapter's contents in the printed
document and an equivalent unnumbered table with links to the
(sub)(sub)sections in the hypertext display. 


\item   \verb/,,[sub]sub]]section{<text>}/ 

        These commands produce standard section headings, numbered in the
printed document, unnumbered in the hypertext display. 
\ei 


\section{ Cross-references} 
\label{.cross.ref} 

        In writing documents, the placeholder \verb/,,whichref{text}{}/ can be
used for references whose target is not yet known. 

        For most referencing commands, a companion with the leading character
capitalised is available; these work the same way except that the text argument
is printed in boldface. This feature is intended for use in the documentation
home-page document, \verb/hb_contents.tex/. 


\subsection{ Cross-references to text in LaTeX documents } 
\label{.cross.ref.latex} 

        Cross-references must work in both the printed and hypertext versions
of a document. For this reason, {\it the standard LaTeX \verb/,,ref/ command is
unsuitable} and the rules given below must be followed instead. 

\bi 
\item   \verb/,,label{.<label>}/ 

        A label name {\em must start in a dot} as shown. LaTeX {\em forbids the
use of underscores in labels}, even if they are escaped by a backslash; it is
recommended to use {\em dots} instead. 

\item   \verb/,,textref{<text>}{.<label>}/ 

        This is translated for the printed document as \verb/<text> (sec.
<section number>)/. In the hypertext display, \verb/<text>/ becomes a hypertext
link to \verb/<label>/. \verb/<text>/ may have a local format of its own, e.g. 
\verb/{\em <text>}/ or \verb/{\bf <text>}/. 

\item   \verb/,,textref{<text>}{<file name>.<label>}/ 

        This is the form for a reference to a label in an external file. In the
printed document it becomes \verb/<text>/ with a numbered footnote showing the
directory and name of the target file's source and the target label. 

        \verb/<file name>/ is the name of a .tex file (without the .tex
extension!) in directory \verb:\$n\_doc/latex:; the name may contain {\em
neither dots nor uppercase characters}, but underscores are allowed (and need
not to be quotes with a backslash). 

\item   \verb/,,whichref{<text>}{}/ 

         can be used as a placeholder for a \verb/,,textref/ whose target is as
yet unknown. 

\ei 

\subsection{ Cross-references to other types of documents} 
\label{.cross.ref.other} 

        The documentation system also contains ASCII documents and PostScript
documents imported from elswehere. Such documents may be referred to through
the following commands: 

\bi 
\item   \verb/,,ascref{<text>}{<file name>}/ 
\item   \verb/,,psref{<text>}{<file name>}/ 
\ei 

        \verb/<file name>/ is the name (without extension!) of a .txt file in
directory \verb:\$n\_doc/txt: or of a .ps file in directory \verb:\$n\_hlp:;
the name may contain {\em neither dots nor uppercase characters}, but
underscores are allowed (and need not to be quotes with a backslash). 

\bi 
\item   \verb/,,srcref{<text>}{<directory><file name><extension>}/ 
\ei 

        This command is used to refer to program code or WNTINC table
definitions (.dsc files). \verb/directory/ must be specified relative to 
\verb/n_src/ and the file extension must be included. 


\subsection{ Equations and equation references} 
\label{.eqref} 

        The commands \verb/,,Eqref{.<label>}/ and \verb/,,eqref{.<label>}/ are
entirely analogous to the \textref{figure references}{.figref} 
\verb/,,Figref/ and \verb/,,figref/. 


\section{ Figures and figure references} 
\label{.fig} 

        Figures with their captions reside in the directory 
\verb:\$n\_doc/fig:. A figure consists of two components: 
\bi 

\item   a \verb:<name>.cap: {\em caption file} containing a caption and a the
directive to include the figure. 

\item   the diagram proper in the form of either an \verb/xfig/ drawing 
\verb:<name>.fig:, or a binary (e.g. \verb/<name>.ps/ or \verb/<name>.gif/)
file. 
\ei 

        A figure with its caption is included in a document source file through
the command 

\verb:  ,,input{../fig/<name>.cap: \ 

In the printed document, figure and caption are included in the standard way.
In the hypertext display the word \verb/FIGURE/ is included, followed by the
caption text, with a hypertext link from FIGURE to a postscript file. \ndoc
generates that file automatically from the .fig file. 


\subsection{ Caption files} 
\label{.caption} 

        The purpose of the caption file (as opposed to spelling out the caption
text in the document source) is to enable the figure with its caption to exist
as an independent unit that can be included in more than one document and also
exist outside the context of any document. 

        The caption file must not only provide the caption, but also the title
under which the figure may be listed in the Handbook Overview. It is therefore
important to adhere strictly to the following format: 


\begin{verbatim} 
        ,,begin{figure}[htbp] 
          ,,fig{<name>} 
          ,,caption[]{<optional LaTeX commands, e.g. \it> 
        <title> 
        <optional LaTeX commands, e.g. \> <optional remainder of caption> 
        } 
          ,,label{.<label>} 
        ,,end{figure} 
\end{verbatim} 

\noi Examples can be found in the 
\verb:/\$n\_doc/fig: directory. 

        You are free to choose the label; the recommended choice, however, is
the file name with underscores replaced by dots. (Remember that underscores are
illegal in labels and that labels must start with a dot.) 

        Comment lines (but no {\em empty} lines!) may be arbitrarily added,
except that the \verb/<title>/ line must {\em immediately follow} the 
\verb/,,caption/ command. 


\subsection{ Xfig figures} 
\label{.xfig} 

        \verb/.fig/ files may be created without any restrictions with the
public-domain program \verb/xfig/. It is not necessary to produce a postscript
file with this program. \ndoc/ will take the \verb/.fig/ file and convert it
automatically. 

        Often, figures drawn on a convenient scale on a workstation screen will
not fit on a standard A4 page. \ndoc/ may be instructed to automatically reduce
its scale by including in the figure a text line 

\verb/  <file name>.fig <nn>/ % 
It is best to put this text in an inconspicuous corner and a small font (e.g.
10pt as opposed to a standard 20pt). 


\subsection{ Postscript figures} 
\label{.ps} 

        It is also possible to include Postscript figures produced in other
ways than through xfig: Use the command \verb/,,ps{<filename>}/ instead of 
\verb/,,fig/ and place the postscript file \verb/<filename>.ps/ in the
directory \verb:/\$n\_doc/bin: with a soft link to it in \verb/\$n\_hlp/. If it
can not be easily recovered, it is advisable to protect it against accidental
deletion! 


\subsection{ Figure references} 
\label{.figref} 

        For references to figures, use 

\verb/  ,,figref{.<label>}/ and \verb/  ,,Figref{.<label>}/ 

\noi These commands translate to hypertext links \verb/figure/ or \verb/Figure/
in the hypertext document, and to a standard reference \verb/figure <n>/ or 
\verb/Figure <n>/ in the printed document. 


\section{ Session transcripts} 

        Verbatim transcripts of pieces of terminal dialogue controlling program
execution are an important ingredient to documentation. Such transcripts are
easily produced through the command 

        \verb/ndoc S[cript] <file name>/ 

This starts a cshell script session in which you execute the program of which
you want a transcript. After exit from the script session, the collected script
is automatically formatted into a LaTeX file \verb/<file name>.trs/; what
remains for you to do is to cut out the irrelevant parts and perhaps add some
comments. A sample section of such a file is shown below 

\spbegin 
\svbegin \begin{verbatim} % 
,,spbegin %.+.+.+.+.+.+.+.+.+.+.+.+.+.+.+.+.+.+.+.+.+.+.+.+.+.+.+.+.+.+.+.+.+. 
,,skeyword{USE\_SCN\_NODE} ,,sprompt{(input node name)} ,,sdefault{= *:}
,,suser{?} % 
,,svbegin ,,begin{verbatim} 
 Specify the node name from which the corrections should be calculated. 
 * indicates the same as the output node name. 
\end{verbatim}\svend 
\verb/,,end{verbatim},,svend/ 
\svbegin\begin{verbatim} ,,spend
%.-.-.-.-.-.-.-.-.-.-.-.-.-.-.-.-.-.-.-.-.-.-.-.-.-.-.-.-.-.-.-.-.-.-. 
% 
,,spbegin %.+.+.+.+.+.+.+.+.+.+.+.+.+.+.+.+.+.+.+.+.+.+.+.+.+.+.+.+.+.+.+.+.+. 
,,skeyword{USE\_SCN\_NODE} ,,sprompt{(input node name)} ,,sdefault{= *:}
,,suser{*} ,,sinline{Use the same SCN file} ,,spend
%.-.-.-.-.-.-.-.-.-.-.-.-.-.-.-.-.-.-.-.-.-.-.-.-.-.-.-.-.-.-.-.-.-.-. 
% 
,,spbegin %.+.+.+.+.+.+.+.+.+.+.+.+.+.+.+.+.+.+.+.+.+.+.+.+.+.+.+.+.+.+.+.+.+. 
,,skeyword{USE\_SCN\_SETS} ,,sprompt{(Set(s) of input uv-data Sectors:
g.o.f.c.s)} ,,sdefault{= "":} ,,suser{2.0.0.0.0} ,,sinline{Sets in job 2}
,,spend %.-.-.-.-.-.-.-.-.-.-.-.-.-.-.-.-.-.-.-.-.-.-.-.-.-.-.-.-.-.-.-.-.-.-. 
% 
,,spbegin %.+.+.+.+.+.+.+.+.+.+.+.+.+.+.+.+.+.+.+.+.+.+.+.+.+.+.+.+.+.+.+.+.+. 
,,skeyword{HA\_RANGE} ,,sprompt{(DEG) (HA range)} ,,sdefault{= *:}
,,suser{,,scr} ,,sinline{Use all scans in these sets} % 
,,svbegin ,,begin{verbatim} 
   0123456789ABCD 
 0 -+++++++++++++ 
 1  -++++++++++++ 
 2   -+++++++++++ 
 3    -++++++++++ 
 4     -+++++++++ 
 5      -++++++++ 
 6       -+++++++ 
 7        -++++++ 
 8         -+++++ 
 9          -++++ 
 A           -+++ 
 B            -++ 
 C             -+ 
 D              - 
\end{verbatim}\svend 
\verb/,,end{verbatim}\svend/ 
\svbegin\begin{verbatim}
,,spend%.-.-.-.-.-.-.-.-.-.-.-.-.-.-.-.-.-.-.-.-.-.-.-.-.-.-.-.-.-.-.-.-./ 
\end{verbatim}\svend 
\spend 

        The file consists of sections delimited by \verb/,,spbegin/ and 
\verb/,,spend/, each consisting of a prompt with its reply and the ensuing
output from the computer. LaTeX will treat these sections as blocks in which a
page break can not occur. The following LaTeX commands are used: 

\bi 

\item   \verb/,,skeyword/, \verb/,,sprompt/ and \verb/,,sdefault/ are the
components of a program prompt; 

\item   \verb/,,suser/ is the user's reply (in which \verb/,,scr/ represents a
null reply (carriage return only) and \verb/,,seof/ an end-of-file
(control-D)).; 

\item   \verb/,,sinline/ is an inline comment typed in during the session (an
exclamation mark plus text following the answer to a prompt); 

\item   \verb/,,svbegin/ and \verb/,,svend/ delimit the verbatim section in
which computer output is rendered. 

\ei 

        It is necessary to check these scripts with some care. The system does
occasionally get confused by the way the parameter interface splits long input
and output lines. The errors are usually easily corrected. 

\bi 

\item   To replace parts of lengthy computer output, e.g. by ellipses; note
that, since this occurs inside \verb/verbatim/ sections, LaTeX commands such as 
\verb/,,vdots/ can not be used. A line consisting of just three dots gives a
satisfactory effect. 

\item   To add \verb/,,sinline/ comments. These must be placed either
immediately following \verb/,,suser/ commands or between an 
\verb/,,end{verbatim}/ and the adjacent \verb/,,svend/. 

\item   To reduce the size of verbatim sections that are more than 80
characters wide (e.g. sections from a log file). 

\ei 


\section {Documentation directories} 

\input{../fig/doc_sources_and_hyper.cap} 
\input{../fig/doc_sources_and_print.cap} 

        The documentation system encompasses three classes of files. The
document source files reside in subdirectories of \verb/$n_doc/; the code
source files which may also be referenced reside in subdirectories of 
\verb/$n_src/. All hypertext files reside in a tree of subdirectories rooted in 
\verb/\$n\_hlp/. The interrelations of all these files are shown in 
\figref{.doc.sources.and.hyper}. 

        The following types of source texts are used: 
\bi 
\item   Narrative LaTeX source texts: \verb:$n_doc/latex/<name>.tex: 

\item   Caption LaTeX source texts: \verb:$n_doc/fig/<name>.cap: 

\item   Standardised LaTeX files describing the private and public program
parameter interfaces: \verb:$n_doc/intfc/<name>.tex:. These files incorporate 
\verb/.tef/ files derived from the program-parameter definition files
(\verb/<name>.psc/). 

\item   ASCII texts: \verb:$n_doc/txt/<name>.txt: 

\item   Postscript files for which no source exists: \verb:$n_doc/ps/<name>.ps: 

\item   Code sources: \verb:$n_src/<directory>/<name>.<extension>: 
\ei 

        In processing, temporary files are created in the current directory;
their names either contain the string \verb/\_tmp./ or end in \verb/.tmp/. 
\ndoc normally deletes them when exiting. 

        \verb/latex2html/, invoked by \verb/ndoc Cook/, translates each file 
\verb/<name>.tex/ into a file 
\verb:\$n\_hlp/<name>/<name>.html:. A large number of 
\verb/.html/ files are therefore in subdirectories of 
\verb/\$n\_hlp/. 

        \verb/ndoc Print/ creates .ps PostScript text files in \verb/\$n\_hlp/. 
        \verb/ndoc Fig/ creates .fps PostScript diagram files in 
\verb/\$n\_hlp/. 




\section {Processing commands} 

\subsection{ Programmer's commands: Processing individual files } 

        The commands are of the form 

        \verb/ndoc <operation> [-<option>] <file>/ 

        \verb/<file>/ is the full file name including the extension, and it
must be in the current directory. Wildcards may be used. 

        The \verb/<operation>/ and \verb/-<option>/ arguments may be
abbreviated and are case-insensitive. The following operations and options are
available: 

\bi 
\item   \verb/P[rint]/ creates [a] PostScript file[s] from [a] .tex source
file[s], with the following options: 
  \bi 
  \item[] \verb/-S[yntax]/ do nothing else; 
  \item[] \verb/-V[iew]/ display the output using ghostview; 
  \item[] \verb/-P[rint]/ print the PostScript output. 
  \ei 

\verb/Print/ uses the LaTeX program. It attempts to suppress the verbose output
of this program by filtering out everything except essential error messages. If
you encounter a problem that you can not solve, submit your source file to the 
\NEWSTAR group for diagnosis. 

\item   \verb/C[ook]/ creates a Hypertext file from [a] .tex document[s]. This
operation makes use of \ltoh. This program may crash over LaTeX syntax errors
without properly analysing them. {\em It is therefore recommended to first do a 
\verb/Print -Syntax/ check.} 

\item   \verb/K[ey]/ creates Hypertext files from [a] .pin, .psc or .pef
file[s] 

\item   \verb/L[ink]/ creates the softlinks from the \verb/n_hlp/ directory to
the \verb:$n_doc/txt: and \verb:$n_src: directories. 
\ei 

        \ndoc checks the dates of its input and output files. If the output is
newer than the date, \ndoc emits a message and proceeds with the next input
file. This check is incomplete in that it does not check the status of 
\verb/,,input/ files such as figure captions. 

        There are two ways of bypassing this check: 
\bi 
\item   Setting the environment variable \verb/n_force/. 
\item   \verb/touch/ing the inpout file to set its modification time to the
present time. 
\ei 

        When working in a shadow system, \ndoc makes no difference between hard
copies and soft links to the master system: Both are unconditionally compiled. 


\subsection{ \NEWSTAR manager's commands: Maintaining the documentation system
} 

        To recompile the entire documentation collection from its sources, one
uses the command 

\verb/          ndoc All/ 

\noindent \ndoc will ask which parts of the system to recompile, the default
being 'yes' for all subsystems. In this mode of operation \ndoc bypasses the
check for an up-to-date output file. 

        One of the actions available in \verb/ndoc All/ is a systematic check
of the correspondence between source and output files. This is a very useful
action because it clears up leftovers of obsolete documents and pinpoints areas
of trouble. 




\section{ Debugging .tex files} 

        \ndoc goes a long way in filtering from the verbiage that LaTeX spews
out the essential diagnostics. It shows the essential section of LaTeX output
where the error is reported (including a line number) plus five lines of text
in the middle of which the error was found. Very often this information
suffices to pinpoint the error. If it does not, you may have to take a better
look either at your source or at the \verb/<file name>_tmp.tex/ file that is
the file LaTeX was actually processing. Note that the line number reported
refers to this latter file! 

        Specific errors that have been encountered and are not covered by this
diagnosis mechanism are described below. 


\subsection{ Temporarily disabling parts of .tex files} 

        \ndoc provides a simple facility for "commenting out" large parts of 
\tex input files: \verb/.c+/ and \verb/.c-/ can be inserted (on lines
containing nothing else) to delimit a piece of text that must be ignored. This
has proved very useful to quickly pinpoint the location where an error occurs. 


\subsection{ Specific errors} 

\subsection{ LaTeX bugs} 

        Both \verb/ndoc Print/ and \verb/ndoc Cook/ make use of LaTeX and may
therefore be affected by LaTeX bugs. For cases not covered here, refer to "The
LaTex Book", which is probably the best informed source about that program. 

\bi 
\item   Symptom: 
        {\em The word \verb/dump/ appears in the place of a Figure or Table.} 

        This is an obscure problem. One situation in which it has been observed
is when a \verb/,,figure/ environment is put at the very start of a document,
i.e. without any preceding text. (This condition will not occur for regular
documents but may occur in test situations.) It has also been seen in a case
where a \verb/tabular/ environment was nested inside a \verb/table/, in which
case the solution was to remove the \verb/table/ environment. 

\item   Symptom: 
        {\em LaTeX complains about a 'runaway argument' in a \verb/,,caption/
environment.} 

        This may happen when the optional argument in square brackets is
omitted. As of 940914, this argument is automatically provided for by 
\verb/doc_preprocess.csh/. 

\item   Symptom: 
        {\em ndoc Print runs without reporting any error, but produces a
PostScript file in which figure references appear without numbers filled in.} 

        This happens when the \verb/,,label/ directive in a \verb/figure/
declaration is placed before the \verb/,,caption/ command.! 

\item   Symptom: 
        {\em LaTeX reports: TeX capacity exceeded, sorry [parameter stack
size=60].} 

        This has been seen to happen when a \verb/,,section/ title argument is
too complicated, e.g. because it contains italicised text. 

\item   Symptom: 
        {\em LaTeX finds a \$ or \_ in a \verb/,,textref/ file argument that is
not escaped by a preceding backslash.} 

        This may happen when the expansion of a command defined through 
\verb/,,newcommand/ contains a \textref{referencing command}{.cross.ref}. Fix
the problem by including escape charaters in the \verb/,,newcommand/
definition. 

\item   Symptom: 
        {\em LaTeX complains about "Paragraph ended before \verb/,,sbox/ was
complete" in a figure caption.} 

        The \verb/,,caption/ environment appears to be sensitive to formatting.
In particular new paragraphs (i.e. blank lines) and \verb/,,indent/ commands
have been noted to produce this nasty error. In some cases it was found
necessary to put the \verb/,,label/ before the \verb/,,caption/. 
\ei 


\subsection{ Processing errors in ndoc} 

        When LaTeX reports an error to \ndoc, \ndoc displays a five-line text
section in the middle of which the error occurred. This is not from the
original input text, but from the preprocessed text as it was eventually
presented tp LaTeX. Most errors are easily recognised and traced back to the
input \tex file. 

\bi 
\item   Symptom: 
        An error occurs which is associated with lines from the .tex input that
were concatenated by ndoc. 

        This may very well be an error in the algorithm used by ndoc to merge
paragraphs into single long lines. It is not necessarily clear why LaTeX does
not accept the long line. An any rate, the recommended action is to check with
the \NEWSTAR manager. 

\item   Symptom: 
        ndoc cook stops with message 
\verb/           nawk: record `[...]' has too many fields/ 

        In preprocessing documents, ndoc formats paragraphs into single lines
to make it easier to identify LaTeX command sequences. Find the offending
paragraph in you documents and insert \verb/<newline><blank>/ at some
appropriate place. The leading paragraph will inhibit the merger of the two
parts of the paragraph, yet LaTeX will format the two parts into one paragraph.
 
\ei 


\subsection{Processing errors in \protect\verb/ndoc Cook/} 

        The program \ltoh on which \verb/ndoc Cook/ is based is rather lax on
checking LaTeX syntax but may stumble over the consequences of an error missed.
{\it It is therefore strongly recommended to first check the syntactical
correctness of the .tex file through \verb/ndoc P/}. 

\bi 
\item   Symptom: 
        \verb/.DVI file can't be opened/ 

        \ltoh runs LaTeX on a selection from the .tex file consisting of
formulas, tabular material and diagrams. This run failed to produce a 
\verb/.dvi/ file, but \ltoh gobbled up all the error information. This error
may be due to improper configuration of your system, {\em e.g.} a misplaced 
\verb/.sty/ file. It is virtually impossible to diagnose except by using a 
\textref{modified version of \ltoh}{.latex2html}. 

\item   Symptom 
\begin{verbatim} The nplot_descr.aux file was not found, so sections will not
be numbered and cross-references will be shown as icons. 
\end{verbatim} 

        Your .tex file contains a \verb/,,ref/ directive. Convert al 
\verb/,,ref/s to \verb/,,textref/s. 

\item   Symptom: 
        Commands that are normally accepted are reported as unknown. 

        This is likely to be the result of a syntax error. (For example, a
missing \verb/,,end{itemize}/ will cause \verb/,,item/ to be reported as
unknown.) Check your input file by running \verb/ndoc Print/ on it. 

\item   Symptom: 
        \verb/xmosaic/ can not find a file pointed at by a hypertext link. 

        Note the file name appearing at the bottom of the \verb/xmosaic/ window
when you point at the reference. This reference should be in either of the
forms: 

        \verb:<file name>/<file name>.html 
\indent \verb:../<file name>/<file name>.html 

If it starts in a \verb:/:, you have to replace your \ltoh by the 
\textref{modified version}{.latex2html}. 

\item   Symptom: 
        \verb/latex2html/ warns that the \verb/.aux/ file was not found but
otherwise \verb/ndoc Cook/ seems to complete normally. 

        The precise nature of this error is not understood, but it seems to
indicate that something is wrong in a cross-reference. (e.g. a LaTeX 
\verb/,,ref/ command). It has been seen to occur when a \verb/,,ref/ with a
label name containing dots is used (cf. the section on \textref{cross
references}{.cross.ref}). 
\ei 


\appendix

\section{\protect\ltoh} 
\label{.latex2html} 

        \ltoh is a public-domain perl script created by Nikos Drakos at Leeds
University, England (Email: nikos@cbl.leeds.ac.uk) and available through
anonymous ftp. It has been found to be a reliable and highly capable program.
Apart from configuring it to the local environment, two changes were found
necessary: 

\bi 
\item   All lines containing the string \verb:s/\W//g: must be commented out by
prefixing them with a \verb/#/. This makes \ltoh capable of handling relative
file references and removes an inconsistency it the processing of labels and
references. 

\item   After the line 

\indent \verb/system("$LATEX $$_images.tex");/ 

insert the lines 

\begin{verbatim} 
        if (!-e "20756_images.dvi") 
          {system("cat 20756_images.log"); die "Failure to process formulas"; } 
\end{verbatim}
\ei 

\subsection{ Formulas and the like in \protect\ltoh}
\label{.formulas}

        LaTeX allows many constructs, such as formulas, tables etc., for which
there is no HTML counterpart (yet). \ltoh converts each single occurrence of
these into a little picture file that is linked in-line into the \verb/.html/
document. The pictures are stored in \verb/.xbm/ files in the same directory as
the \verb/.html/ file.

        Since building these pictures is a very time-consuming process, \ltoh
seeks to reuse them as much as possible whenever it recompiles a \verb/.tex/
file. To this end, it maintains an administration in the file \verb/images.pl/
in the \verb/.html/ file's directory. Therefore, to insure the integrity of the
.html document, both the \verb/.xbm/ files and \verb/images.pl/ must be left in
place. As of May 1996, \verb/doc_cook.csh/ contains a section of code that
finds all in-line picture references in the renewed \verb/.html/ file, reports
any referred-to files that are missing and deletes unreferenced \verb/.xbm/
files.

        In cases a \verb/.xbm/ file is reported missing, the way to get it
rebuilt is to delete \verb/images.pl/ in order to force \ltoh to rebuild all
the picture files from scratch.  




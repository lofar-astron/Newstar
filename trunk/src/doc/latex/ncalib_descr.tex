%
% ncalib_descr.tex  v1.2 04/08/93 JEN
%       JPH 940407      Technical changes
%
\chapter{The Program NCALIB}

\tableofcontents {\bf Related chapters:}
\begin{itemize}
\item \textref{NCALIB REDUN: Redundancy, Align and Selfcal}{ncalib_redun}
\item \textref{NCALIB POLAR: Polarisation corrections}{ncalib_polar}
\end{itemize}

%============================================================================

\section{Overview of NCALIB options}
\label{.overview}

	The program NCALIB is the heart of the NEWSTAR package.  It offers a
wide range of options for the determination of all kinds of instrumental
corrections (calibration) and their application to the data (correction).
Corrections are stored in the Set headers and Scan headers of the \textref{SCN
file}{scn_descr}.  They may be applied to the uv-data whenever they are read
into memory, subject to the specifications given by means the
\textref{APPLY}{common_descr} keyword. Corrections in the SCN-file may also be
manipulated, or `imported' from other sources.


\begin{itemize}
\item \textref{\bf REDUNDANCY}{ncalib_redun}: Calculate Redundancy, Align or
Selfcal corrections. %
\item
\textref{\bf POLAR}{ncalib_polar}:
  Determine/manipulate polarisation corrections: %
  \begin{itemize}
  \item \textref{\bf CALC}{ncalib_polar.calc}:  Calculate polarisation
corrections
  \item \textref{\bf SHOW}{ncalib_polar.show}:  Show polarisation corrections
  \item \textref{\bf SET}{ncalib_polar.set}:    Set corrections manually
  \item \textref{\bf COPY}{ncalib_polar.copy}:  copy polarisation corrections
from one set to others
  \item \textref{\bf EDIT}{ncalib_polar.ccopy}: Edit polarisation corrections
  \item \textref{\bf ZERO}{ncalib_polar.zero}:  Zero polarisation corrections
  \item \textref{\bf VZERO}{ncalib_polar.vzero}:        Calculate X-Y phase
zero difference, assuming V=0
  \end{itemize} %
\item   \textref{\bf SET}{.option.set}: Set some correction data %
  \begin{itemize}
  \item \textref{\bf ZERO}{.set.zero}:          Zero selected corrections
  \item \textref{\bf MANUAL}{.set.manual}:      Copy corrections from manual
input
  \item \textref{\bf COPY}{.set.copy}:          Copy corrections from somewhere
else
  \item \textref{\bf LINE}{.set.line}:          Copy all corrections from
corresponding continuum channel
  \item \textref{\bf EXTINCT}{.set.extinct}:    Set extinction
  \item \textref{\bf REFRACT}{.set.refract}:    Set refraction
  \item \textref{\bf FARADAY}{.set.faraday}:    Set Faraday rotation
  \item \textref{\bf RENORM}{.set.renorm}:      Renormalise telescope
corrections
  \end{itemize} %
\item   \textref{\bf SHOW}{.option.show}:       Show (on printer) the average
telescope corrections (over all HA-Scans) in specified set(s). %
\item   {\bf QUIT}:     Quit program NCALIB
\end{itemize}


%============================================================================
%============================================================================

\section{Option SET: Set various corrections in the SCN-file}
\label{.option.set}

	The NCALIB option SET is accessed in the following manner:

\scmd{dwe ncalib}
\sline{\ \ \ \ \ NCALIB\$1 is started at 17-DEC-92 15:58:01}

\skeyword{OPTION}
\sprompt{(REDUNDANCY,POLAR,SET,SHOW,QUIT)}
\sdefault{ = QUIT}:
\suser{set}

\skeyword{SET\_OPTION}
\sprompt{(ZERO,MANUAL,COPY,LINE,EXTINCT,...)}
\sdefault{ = QUIT}:
\suser{...}

In the following, the various sub-options of the NCALIB option SET will be
treated in some detail.


%----------------------------------------------------------------------------

\subsection{SET ZERO: Set selected corrections to zero}
\label{.set.zero}

The uv-data stored in a SCN file are never physically modified. Corrections are
stored separately in the Scan and Set header(s), and applied to the data
whenever they are read into memory for processing. Thus, by setting some (or
all) corrections to zero, a reduction process that has gone wrong can always be
returned to a known initial state. The keyword ZERO allows the user to specify
which of the various kinds of corrections are to be set to zero.

\skeyword{SET\_OPTION}
\sprompt{(ZERO,MANUAL,COPY,LINE,EXTINCT,...)}
\sdefault{ = QUIT}:
\suser{zero}

\skeyword{ZERO}
\sprompt{(ALL,NONE,RED,ALG,OTH,...)}
\sdefault{ = NONE}:
\suser{all}

***** Insert new script here *****

NB: Going trough a large SCN file (e.g. one with many line channels) will take
some time.

%----------------------------------------------------------------------------

\subsection{SET MANUAL: Manual input of telescope corrections}
\label{.set.manual}

The telescope gain and phase corrections may be specified manually by the user.
The given values will be stored as `other corrections' (OTHC) in the Scan
headers of the specified range (Sets and HA-range).

{\it Note of the editor: It is not yet clear to me what happens to the other
telescope corrections (REDC and ALGC) in the Scan header. It seems reasonable
that they are set to zero.}

***** Insert new script here *****

%----------------------------------------------------------------------------
%----------------------------------------------------------------------------

\subsection{SET RENORM: Renormalise telescope corrections}
\label{.set.renorm}

In the Redundancy calibration process, the {\em average} telescope gain and
phase corrections over all telescopes are arbitrarily set to zero. This is a
`reasonable' assumption, unless the correction for one or more telescopes
happens to be  anomalously large. In that case, the gain and/or phase
corrections of all the other telescopes will be shifted by an `unreasonable'
amount (since the average must be zero). Therefore, it is sometimes desirable
to RENORMalise by shifting the telescope corrections by a common amount, until
the average is zero for a selection of `good' telescopes.


***** Insert new script here *****

Note that the program terminates upon completion, i.e. it does not return to
the level of the SET option.


%----------------------------------------------------------------------------

\subsection{SET LINE: Copy telescope corrections from continuum channel}
\label{.set.line}

Since Selfcal/Redundancy calibration requires a S/N of more than 2-5, it is
often possible for the continuum channel (0), but not for the individual line
channels. In those cases, the telescope gain and phase corrections that have
been found for channel 0 may be transferred to the line channels. This is a
`reasonable' thing to do, since the total telescope errors will vary much more
than the reletive errors between channels (i.e. the bandpass shape).

***** Insert new script here *****

%----------------------------------------------------------------------------
%----------------------------------------------------------------------------

\subsection{SET COPY: Copy telescope corrections from somewhere else}
\label{.set.copy}

In the Standard Calibration process, the telescope gain and phase errors
$\gerr_{i}$ and $\perr_{i}$ are calculated with the help of a strong calibrator
source, which has been observed directly before (or after) the actual
observation. In order to use these calibrator corrections to correct the
latter, they must be transferred (copied) from the Scan header(s) of the
calibrator to the Scan header(s) of the observed object.

There are two possibilities: The calibrator observation (and thus the desired
corrections) may be stored in a separate SCN-file (node), or they may be stored
as another `job' of the same SCN-file as the observed object. In the following
example, the observed object is stored as job nr 0, while the calibrator
observation is stored in the same SCN-file, as job nr 1:


***** Insert new script here *****

NB: Note that the program exits upon completion, and does not return to
SET\_OPTION.

%----------------------------------------------------------------------------
\subsection{SET CCOPY: Like COPY, but more intelligent}
\label{.set.ccopy}



%----------------------------------------------------------------------------
%----------------------------------------------------------------------------

\subsection{SET EXTINCT: Set extinction correction}
\label{.set.extinct}

The actual atmospheric extinction factor (as a function of telescope elevetion)
may differ from the default value, which is based on a standard model of the
atmosphere.


***** Insert new script here *****

%----------------------------------------------------------------------------
\subsection{SET REFRACT: Set refraction correction}
\label{.set.refract}

***** Insert new script here *****

%----------------------------------------------------------------------------
\subsection{SET FARADAY: Set Faraday rotation}
\label{.set.faraday}

Information about the ionospheric Faraday rotation during the observation may
be obtained externally, e.g. from ionosonde measurements. NFRA receives these
values routinely from meteorological stations not too far from the WSRT. The
information may be entered into the SCN-file as a function of HA. They are
stored as corrections (FARAD) in the Scan header, and will be applied routinely
to the data if specified by the keyword APPLY.

***** Insert new script here *****

%============================================================================
\newpage
\section{Option SHOW: Print average corrections (on line printer)}
\label{.option.show}

The average telescope gain and phase corrections that are stored in the SCN
file can be printed, for a specified range of sets. The numbers printed are a
{\em combination} of the several kinds of telescope corrections stored in the
SCN file (see the SCN-file description section). The desired combination may be
specified with the keywords APPLY and D\_APPLY.

The output takes the form of 14 columns of 8 numbers:
\\- X gain: as gain factor and as percentage (\%)
\\- X phase: in radians and in degrees
\\- Y gain: as gain factor and as percentage (\%)
\\- Y phase: in radians and in degrees

***** Insert new script here *****

The output will now be printed on the line printer. Note that the program
terminates upon completion, i.e. it does not return to the level of the SHOW
option.




%
% ncalib_polar.tex  v1.2 04/08/93 JEN
%       JPH 940407      Technical changes
%       R. Sault 940606 Thorough revision
%       JPH 940927      Some text improvements. Blank lines following labels.
%       JPH 951107      Break a paragraph that was too long for ndoc


\chapter{NCALIB POLAR: Polarisation corrections}
\tableofcontents



\section{ Preface }

	This document describes the handling of polarisation calibration and
correction by the NCALIB POLAR option.  This option is activated as follows:

\spbegin %.+.+.+.+.+.+.+.+.+.+.+.+.+.+.+.+.+.+.+.+.+.+.+.+.+.+.+.+.+.+.+.+.+.
\scmd{exe\ ncalib}
\svbegin \begin{verbatim}
      NCALIB$1 (v4.62) is started at 30-May-94 13:52:04
\end{verbatim}\svend
\spend %.-.-.-.-.-.-.-.-.-.-.-.-.-.-.-.-.-.-.-.-.-.-.-.-.-.-.-.-.-.-.-.-.-.-.
%
\spbegin %.+.+.+.+.+.+.+.+.+.+.+.+.+.+.+.+.+.+.+.+.+.+.+.+.+.+.+.+.+.+.+.+.+.
\skeyword{OPTION}
\sprompt{(REDUNDANCY, POLAR, SET, SHOW, QUIT)}
\sdefault{= QUIT:}
\suser{polar}
\spend %.-.-.-.-.-.-.-.-.-.-.-.-.-.-.-.-.-.-.-.-.-.-.-.-.-.-.-.-.-.-.-.-.-.-.
%
\spbegin %.+.+.+.+.+.+.+.+.+.+.+.+.+.+.+.+.+.+.+.+.+.+.+.+.+.+.+.+.+.+.+.+.+.
\skeyword{POLAR\_OPTION}
\sprompt{(CALC, SHOW, SET, COPY, EDIT...)}
\sdefault{= QUIT:}
\suser{show}
\sinline{Select option}
\spend %.-.-.-.-.-.-.-.-.-.-.-.-.-.-.-.-.-.-.-.-.-.-.-.-.-.-.-.-.-.-.-.-.-.-.



\section{WSRT polarisation calibration strategy}
\label{.strategy}

	In the following, it will be assumed that the dipoles of the WSRT
telescopes are {\em parallel} (++).  The usual calibration strategy consists of
the following steps:

\begin{itemize}

\item
	In addition to observing calibrators along with normal observations,
the WSRT at regular intervals observes calibrators to determine various
instrumental parameters which are reasonably constant. These corrections are
automatically applied to your data at the telescope.

\item
	The complex gain factors $\cGain_{ij}$ (i.e. dipole gain errors,
$\lerr_i=\log(\gerr_i)$, and phase errors, $\perr_i$ are determined by
observing a strong calibrator, for which an accurate model exists. The
determination is done separately for the X-dipoles and the Y-dipoles, using
\eqref{.equ.pol.008}.  {\em Note that any inaccuracy in the assumed value for Q
will be interpreted as a gain error.  Also note that $\cGain_{xy}$ and
$\cGain_{yx}$ will be inaccurate by the value of the ``phase-zero difference
(PZD)'' between the X and Y dipoles.}

\item
	The dipole angle errors $\derr_i$ and ellipticities $\eerr_i$ are
determined by observing a strong {\em unpolarised} calibrator. This is done by
means of NCALIB option POLAR CALC in this section, using \eqref{.equ.pol.010}.
The measured corrections are stored (as POLC) in the {\em sector header} of the
SCN-file. (I.e. they are assumed to be constant for the duration of an entire
sector rather than to change from scan to scan.).

\item
	The missing 'phase-zero difference (PZD)' between the X- and Y-dipoles
is determined with the help of a calibrator with strong U. Again,
\eqref{.equ.pol.010} is used.  It is assumed that $\Apol_{xy}$ and
$\Apol_{yx}$, which determine the `leakage' of the large I-term into the
result, are negligible in this stage, and that $V=0$.  If this assumption is
incorrect, this will translate into a spurious V in later observations.

\item
	Ionospheric Faraday rotation may vary on much shorter timescales than
the length of the observation.  This can be calibrated to a large extent by
introducing external information, obtained by
\textref{ionosonde measurements}{ncalib_descr.set.faraday}.

\item
	The WSRT instrumental polarisation is small, due to the placement of
the feed on the axis of the antenna paraboloid.  The effect can be eliminated
with the help of the multi-parameter \NEWSTAR source model (see NMODEL).

\end{itemize}

	For the best results, it may be necessary to iterate two or three
times, because one pair of dipole errors (position angle/ellipticity) may
affect the determination of the other pair (gain/phase).  However, the process
will iterate to the correct result, because, {\em in the case of parallel
dipoles}, all four types of dipole errors have distinct signatures.  This means
that one type of dipole error cannot be interpreted as another type, and still
give a consistent result.  Therefore, all four types can be determined
independently.

%=============================================================================
\section{Polarisation equations}

	Each of the 14 WSRT telescopes have two perpendicular linear dipoles, X
and Y.  In the existing front-ends, the XY-dipole unit can be rotated over an
arbitrary angle.  The dipole position angle ($\dang$) is defined from North
($\dang=0^\circ$) through East ($\dang=90^\circ$).  The complex visibility
$\cVis_{12}$ that is measured with an interferometer (consisting of two dipoles
with position angles $\dang_1$ and $\dang_2$ can be written as (see Weiler,
1973):

\begin{eqnarray}
\cVis_{12} =& \cGain_{12} &( I[\cos(\dang_1-\dang_2) -
				  \Apol_{12}\sin(\dang_1-\dang_2)] \nonumber \\
	    &  + &           Q[\cos(\dang_1+\dang_2) -
				  \Bpol_{12}\sin(\dang_1+\dang_2)] \nonumber \\
	    &  + &           U[\sin(\dang_1+\dang_2) +
				  \Bpol_{12}\cos(\dang_1+\dang_2)] \nonumber \\
	    &  - &          iV[\sin(\dang_1-\dang_2) +
				  \Apol_{12}\cos(\dang_1-\dang_2)])
\label{.equ.pol.002}
\end{eqnarray}

in which I, Q, U and V are the Fourier transforms of the corresponding Stokes
parameters of the observed source, and the $\cGain$, $\Apol$ and $\Bpol$
factors contain the four types of dipole errors: phase ($\perr$), gain
($\lerr=\log(\gerr)$), dipole angle error ($\derr$) and ellipticity ($\eerr$).
In the ideal case, they are all zero. For small values of $\derr$ and $\eerr$,
second-order terms can be ignored, and we can write:
%
\begin{eqnarray}
\Apol_{12} &=& (\derr_1 - \derr_2) - i(\eerr_1 + \eerr_2) \nonumber\\
\Bpol_{12} &=& (\derr_1 + \derr_2) - i(\eerr_1 - \eerr_2)
\label{.equ.pol.004}
\end{eqnarray}
%
\begin{eqnarray}
\cGain_{12} = \gerr_1\gerr_2\exp(-i(\perr_1-\perr_2))
	    = \exp(\lerr_1+\lerr_2-i(\perr_1-\perr_2))
\label{.equ.pol.006}
\end{eqnarray}


\subsection{Parallel dipoles (++)}

In the `normal' position (+) of the dipole unit, $\dang_x=90^\circ$ (east), and
$\dang_y=180^\circ$ (south). Usually the dipole units in all WSRT telescopes
are set `parallel' (++) to each other. In this case, the equations reduce to a
particularly simple form. Again ignoring second-order terms, we get %
\begin{eqnarray}
\cVis_{xx} &=& \cGain_{xx}(I-Q) \nonumber\\
\cVis_{yy} &=& \cGain_{yy}(I+Q)
\label{.equ.pol.008}
\end{eqnarray}
%
\begin{eqnarray}
\cVis_{xy} &=& \cGain_{xy}(-U - iV - \Apol_{xy} I) \nonumber\\
\cVis_{yx} &=& \cGain_{yx}(-U + iV + \Apol_{yx} I)
\label{.equ.pol.010}
\end{eqnarray}
%
in which $\cVis_{xx}$ is the visibility measured between the X-dipoles of
telescopes {\em i} and {\em j}, etc. After calibration, $\cGain=1$ and
$\Apol=0$, and the complex Stokes values can be calculated from the observed
visibilities:
%
\begin{eqnarray} I&=& +(\cVis_{xx} + \cVis_{yy})/2  \nonumber \\ Q&=&
-(\cVis_{xx} - \cVis_{yy})/2  \nonumber \\ U&=& -(\cVis_{xy} + \cVis_{yx})/2
\nonumber \\ iV&=& -(\cVis_{xy} -
\cVis_{yx})/2
\label{.equ.pol.014}
\end{eqnarray}

In the parallel mode (++), the two sets of dipole errors (phase/gain and
angle/ellipticity) are separable, and so can be determined. Nowadays (after
December 1983), the dipoles are almost always parallel, and in the new
frontends (after 1997), the dipoles will not be rotatable.


\subsection{X-Y Phase Zero Difference (PZD)}

In the case of parallel dipoles (++), the complex gain errors of the X-dipoles
and those of the Y-dipoles are calibrated separately; $\cVis_{xy}$ and
$\cVis_{yx}$ will usually not have enough signal for calibration, because these
dipoles are nominally perpendicular and the fraction of linear polarisation is
usually small in calibrators. Since we only calibrate phase {\em differences},
an arbitrary phase zero ($\perr_{x0}$ and $\perr_{y0}$) is assigned to the X-
and Y-dipoles. These phase zeros will cancel out for $\cGain_{xx}$ and
$\cGain_{yy}$, but not for $\cGain_{xy}$ and $\cGain_{yx}$. Thus, the latter
will be multiplied by an unknown phase factor $\pzd$, the so-called {\em XY
Phase Zero Difference (PZD)}.

\begin{eqnarray}
\cGain_{xx} &=& \gerr_x\gerr_x\exp(-i(\perr_x-\perr_x)) \nonumber \\
\cGain_{yy} &=& \gerr_y\gerr_y\exp(-i(\perr_y-\perr_y))\\
 \label{.equ.pol.016} %
\cGain_{xy} &=& \gerr_x\gerr_y\exp(-i(\perr_x-\perr_y+\pzd)) \nonumber \\
\cGain_{yx} &=& \gerr_y\gerr_z\exp(-i(\perr_y-\perr_x-\pzd))
\label{.equ.pol.018}
\end{eqnarray}

\noindent in which $\pzd=\perr_{x0}-\perr_{y0} = {\rm PZD}$. Of course,
$\cGain_{xy}$ is the complex gain of the interferometer made up of the X-dipole
of telescope $i$ and Y-dipole of telescope $j$, etc.


\subsection{Crossed dipoles ($+\times$)}

	Before 1983, polarisation measurements with the WSRT were usually
carried out with dipoles in the fixed telescopes (0-9) in the `normal'
position, and the dipoles in the movable telescopes (A-D) rotated by $45^\circ$
($\times$), with $\dang_x=45^\circ$ and $\dang_y=135^\circ$.  The 40 standard
(fixed-movable) interferometers were said to have `crossed' ($+\times$)
dipoles.  In this case, the visibility equations take the following form:
%
\begin{eqnarray}
\cVis_{xx} &=& \ \ \cGain_{xx}(I(1-\Apol_{xx}) -
	Q(1+\Bpol_{xx}) + U(1-\Bpol_{xx}) + iV(1-\Apol_{xx})) \nonumber \\
\cVis_{yy} &=& \ \ \cGain_{yy}(I(1+\Apol_{yy}) +
	Q(1+\Bpol_{yy}) - U(1-\Bpol_{yy}) + iV(1-\Apol_{yy})) \nonumber \\
\cVis_{xy} &=& -\cGain_{xy}(I(1-\Apol_{xy}) +
	Q(1-\Bpol_{xy}) + U(1+\Bpol_{xy} - iV(1+\Apol_{xy})) \nonumber \\
\cVis_{yx} &=& \ \ \cGain_{yx}(I(1-\Apol_{yx}) -
	Q(1-\Bpol_{yx}) - U(1+\Bpol_{yx}) - iV(1+\Apol_{yx}))
\label{.equ.pol.022}
\end{eqnarray}
%
	With crossed dipoles, the signal-to-noise ratio of the xy/yx terms are
similar to that of the xx/yy terms.  This makes it possible to make a
redundancy calibration solution for all 28 dipoles simultaneously, {\em but
only if we may assume that $V=0$}.  In principle, this would open the way to
continuous polarisation calibration on the object itself, provided it has
enough flux.

	However, crossed dipoles introduce a number of new problems that make
them less attractive:
%
\begin{itemize}

\item The two sets of dipole errors (phase/gain and angle/ellipticity) are no
longer completely separable.  This means that the phase solution will influence
the angle error solution, and the gain solution will influence the ellipticity
solution.

\item For part of the telescopes, the four-petal clover-leaf pattern of
instrumental polarisation will be rotated by $45^\circ$ with respect to the sky
and with respect to the four legs that support the focus box. This complicates
any calibration schemes for instrumental polarisation.

\end{itemize}

	These problems, combined with the fact that the phase-zero problem is
not really avoided (since redundancy solution is only possible if it has to be
assumed that $V=0$) have led to the practice to use only parallel dipoles.

	{\it (Annotation by JPH 940624: The reader is cautioned that the
arguments advanced here against the use of crossed dipoles are a matter of
continuing controversy.  It can easily be shown from first principles that the
separability of phase/gain and dipole errors is irrelevant. Furthermore, the
parallel-dipole configuration is inherently asymmetric with respect to Stokes
{\em Q} and {\em U} and crossed dipoles are clearly superior in this respect;
how important this is is not clear.)}


\subsection{Faraday rotation}

When radiation passes through a charged medium (like the ionosphere), the place
of linear polarisation will be rotated by the angle $\farang$:
%
\begin{eqnarray} Q &=& Q_{obs}\cos(2\farang) + U_{obs}\sin(2\farang) \nonumber
\\ U &=& U_{obs}\cos(2\farang) + Q_{obs}\sin(2\farang)
\label{.equ.pol.024}
\end{eqnarray}
%
Note that $I$, $V$ and $P=\sqrt{Q^2+U^2}$ are independent of $\farang$. The
Faraday effect is strongly frequency-dependent: $\farang \propto \nu^{-2}$.


\section{POLAR SHOW: Show corrections}
\label{.show}

The 56 correction factors (2 for each of the 28 dipoles) for angle errors and
ellipticities that are stored in the {\em set header} (as POLC, see SCN-file
description) can be viewed in a digestible form:

\spbegin %.+.+.+.+.+.+.+.+.+.+.+.+.+.+.+.+.+.+.+.+.+.+.+.+.+.+.+.+.+.+.+.+.+.
\skeyword{POLAR\_OPTION}
\sprompt{(CALC, SHOW, SET, COPY, EDIT...)}
\sdefault{= QUIT:}
\suser{show}
\spend %.-.-.-.-.-.-.-.-.-.-.-.-.-.-.-.-.-.-.-.-.-.-.-.-.-.-.-.-.-.-.-.-.-.-.
%
\spbegin %.+.+.+.+.+.+.+.+.+.+.+.+.+.+.+.+.+.+.+.+.+.+.+.+.+.+.+.+.+.+.+.+.+.
\skeyword{SCN\_NODE}
\sprompt{(input/output 'node' name)}
\sdefault{= "":}
\suser{psr1937.jan93}
\spend %.-.-.-.-.-.-.-.-.-.-.-.-.-.-.-.-.-.-.-.-.-.-.-.-.-.-.-.-.-.-.-.-.-.-.
%
\spbegin %.+.+.+.+.+.+.+.+.+.+.+.+.+.+.+.+.+.+.+.+.+.+.+.+.+.+.+.+.+.+.+.+.+.
\skeyword{SCN\_SETS}
\sprompt{(Set(s) to do:  g.o.f.c.s )}
\sdefault{= "":}
\suser{2.0.0.0.0}
\sinline{For example} %
\svbegin \begin{verbatim}
 Sector: 2.0.0.0.0

	Position          Ellipticity    Rotation Orthog.
       X(%)     Y(%)     X(%)     Y(%)    (deg)    (deg)

 0    -1.13    -0.06    -0.37     0.54    -0.34     0.61
 1    -0.16     0.13     0.07     0.04    -0.01     0.17
 2     1.25     0.82     1.77    -2.56     0.59    -0.24
 3     0.49    -0.02     0.11     1.19     0.13    -0.29
 4    -0.72    -0.14    -0.03    -0.46    -0.25     0.33
 5    -0.09    -0.41     0.47     0.15    -0.14    -0.19
 6    -0.04    -0.21    -0.93     0.52    -0.07    -0.10
 7    -1.30    -1.42     0.45    -0.21    -0.78    -0.07
 8     2.51     1.77     0.86     0.99     1.23    -0.42
 9     0.80    -0.23    -0.69    -0.39     0.16    -0.59
 A    -0.03    -0.73     1.61    -0.08    -0.22    -0.40
 B    -0.13    -0.57    -0.57     0.30    -0.20    -0.26
 C     0.00     0.00     0.00     0.00     0.00     0.00
 D    -0.05    -0.31    -1.23     1.50    -0.10    -0.15
\end{verbatim}\svend
\spend %.-.-.-.-.-.-.-.-.-.-.-.-.-.-.-.-.-.-.-.-.-.-.-.-.-.-.-.-.-.-.-.-.-.-.

The first two columns (position) give the dipole angle error, expressed as the
percentage of I that will corrupt Q, U and V. This is equivalent to expressing
the dipole misalignment errors as percentages of one radian. In the last two
columns, these same numbers are re-interpreted as a position error (rotation)
of the entire XY-dipole assembly in degrees, and a deviation from the nominal
orthogonality between the X and Y dipole. This is useful, since the entire
XY-dipole assembly can be rotated as a whole for each WSRT telescope.

The ellipticities are also given as percentages of the corrupting I, or as
percentages of one radian.

%=============================================================================

\section{POLAR CALC: Calculate corrections}
\label{.calc}

The dipole angle {\em errors} $\derr_i$ and ellipticities $\eerr_i$ are
calculated using \eqref{.equ.pol.010}. A strong calibrator is observed, which
is known to be unpolarised  ($U=0$ and $V=0$). It is assumed that gain and
phase have already been calibrated by other means (e.g. Selfcal): $\cGain_{xy}
= \cGain_{yx} = 1$. Thus,
\eqref{.equ.pol.010} reduces to:
%
\begin{equation}
  \cVis_{12}/I = \Apol_{12} = (\derr_{1}-\derr_{2})-i(\eerr_{1}+\eerr_{2})
\end{equation}
%
with $I=(\cVis_{11}+\cVis_{22})/2$.

The system of linear equations (one for each $\cVis_{xy}$ and $\cVis_{yx}$) can
be solved in a manner that is entirely analogous to the
\textref{redundancy solution}{ncalib.redun} for gain and phase errors. The
separate solutions for the real and imaginary parts now give the $\derr_i$ and
$\eerr_i$ respectively.

Since the S/N of the $\cVis_{xy}$ and $\cVis_{yx}$ will be small, the
least-squares solution will be more accurate if more data (Sets and HA-range)
are used. However, it must of course be assumed that the $\derr_i$ and
$\eerr_i$ values are the same for all these data.  This is a fairly safe
assumption, since the causes for these dipole errors are `mechanical', and vary
only slowly in time. If the estimated values are to be useful for correction
subsequent observations, they must at least be constant for the duration of the
calibrator observation.

\spbegin %.+.+.+.+.+.+.+.+.+.+.+.+.+.+.+.+.+.+.+.+.+.+.+.+.+.+.+.+.+.+.+.+.+.
\skeyword{POLAR\_OPTION}
\sprompt{(CALC, SHOW, SET, COPY, EDIT...)}
\sdefault{= QUIT:}
\suser{calc}
\spend %.-.-.-.-.-.-.-.-.-.-.-.-.-.-.-.-.-.-.-.-.-.-.-.-.-.-.-.-.-.-.-.-.-.-.
%
\spbegin %.+.+.+.+.+.+.+.+.+.+.+.+.+.+.+.+.+.+.+.+.+.+.+.+.+.+.+.+.+.+.+.+.+.
\skeyword{SCN\_NODE}
\sprompt{(input/output 'node' name)}
\sdefault{= "":}
\suser{psr1937.jan93}
\spend %.-.-.-.-.-.-.-.-.-.-.-.-.-.-.-.-.-.-.-.-.-.-.-.-.-.-.-.-.-.-.-.-.-.-.
%
\spbegin %.+.+.+.+.+.+.+.+.+.+.+.+.+.+.+.+.+.+.+.+.+.+.+.+.+.+.+.+.+.+.+.+.+.
\skeyword{SCN\_LOOPS}
\sprompt{(niter, Setincr(g.o.f.c.s) ....)}
\sdefault{= "":}
\suser{\scr}
\spend %.-.-.-.-.-.-.-.-.-.-.-.-.-.-.-.-.-.-.-.-.-.-.-.-.-.-.-.-.-.-.-.-.-.-.
%
\spbegin %.+.+.+.+.+.+.+.+.+.+.+.+.+.+.+.+.+.+.+.+.+.+.+.+.+.+.+.+.+.+.+.+.+.
\skeyword{SCN\_SETS}
\sprompt{(Set(s) to do:  g.o.f.c.s )}
\sdefault{= "":}
\suser{2.0.0.0.0}
\spend %.-.-.-.-.-.-.-.-.-.-.-.-.-.-.-.-.-.-.-.-.-.-.-.-.-.-.-.-.-.-.-.-.-.-.
%
\spbegin %.+.+.+.+.+.+.+.+.+.+.+.+.+.+.+.+.+.+.+.+.+.+.+.+.+.+.+.+.+.+.+.+.+.
\skeyword{HA\_RANGE}
\sprompt{(DEG) (HA range)}
\sdefault{= *:}
\suser{\scr} %
\svbegin \begin{verbatim}
 All cross interferometers pre-selected
\end{verbatim}\svend
\spend %.-.-.-.-.-.-.-.-.-.-.-.-.-.-.-.-.-.-.-.-.-.-.-.-.-.-.-.-.-.-.-.-.-.-.
%
\spbegin %.+.+.+.+.+.+.+.+.+.+.+.+.+.+.+.+.+.+.+.+.+.+.+.+.+.+.+.+.+.+.+.+.+.
\skeyword{SELECT\_IFRS}
\sprompt{(Select/de-select ifrs)}
\sdefault{= "":}
\suser{p} %
\svbegin \begin{verbatim}
   0123456789ABCD
 0 -+++++++++++++
 1  -++++++++++++
 2   -+++++++++++
 3    -++++++++++
 4     -+++++++++
 5      -++++++++
 6       -+++++++
 7        -++++++
 8         -+++++
 9          -++++
 A           -+++
 B            -++
 C             -+
 D              -
\end{verbatim}\svend
\spend %.-.-.-.-.-.-.-.-.-.-.-.-.-.-.-.-.-.-.-.-.-.-.-.-.-.-.-.-.-.-.-.-.-.-.
%
\spbegin %.+.+.+.+.+.+.+.+.+.+.+.+.+.+.+.+.+.+.+.+.+.+.+.+.+.+.+.+.+.+.+.+.+.
\skeyword{SELECT\_IFRS}
\sprompt{(Select/de-select ifrs)}
\sdefault{= "":}
\suser{\scr}
\spend %.-.-.-.-.-.-.-.-.-.-.-.-.-.-.-.-.-.-.-.-.-.-.-.-.-.-.-.-.-.-.-.-.-.-.
%
\spbegin %.+.+.+.+.+.+.+.+.+.+.+.+.+.+.+.+.+.+.+.+.+.+.+.+.+.+.+.+.+.+.+.+.+.
\skeyword{BASEL\_CHECK}
\sprompt{(M) (Baseline deviation allowed)}
\sdefault{= 0.5 M:}
\suser{\scr}
\spend %.-.-.-.-.-.-.-.-.-.-.-.-.-.-.-.-.-.-.-.-.-.-.-.-.-.-.-.-.-.-.-.-.-.-.

The only difference with a `normal' (gain/phase) Redundancy solution is that
these solutions are for all 28 dipoles simultaneously. Therefore, the necessary
constraint equations have 28 coefficients. The words `gain' and `phase' refer
to the separate Real and Imaginary solutions. The succession of dipoles is
0X,0Y,1X,1Y,...etc.

The constraint equations arbitrarily set the {\em average ellipticity} and the
{\em average dipole angle errors} over the array to zero. Without further
information, this may be the most reasonable value. But if it is wrong it could
affect the observations that are calibrated with the results.

The result of CALC looks very much like the output of SHOW (see above), except
that the estimated accuracy (mean error) of the numbers is give in brackets.

%\input ncalib_polar_calc2.scr
\spbegin %.+.+.+.+.+.+.+.+.+.+.+.+.+.+.+.+.+.+.+.+.+.+.+.+.+.+.+.+.+.+.+.+.+.
\svbegin \begin{verbatim}
 Sector: 2.0.0.0

	 Position                     Ellipticity           Rotation Orthog.
	 X(%)           Y(%)           X(%)           Y(%)        (deg)    (deg)

 0    -1.13(0.06)    -0.06(0.06)    -0.37(0.06)     0.54(0.06)    -0.34     0.61
 1    -0.16(0.06)     0.13(0.06)     0.07(0.06)     0.04(0.06)    -0.01     0.17
 2     1.25(0.06)     0.82(0.06)     1.77(0.06)    -2.56(0.06)     0.59    -0.24
 3     0.49(0.06)    -0.02(0.06)     0.11(0.06)     1.19(0.06)     0.13    -0.29
 4    -0.72(0.06)    -0.14(0.06)    -0.03(0.06)    -0.46(0.06)    -0.25     0.33
 5    -0.09(0.06)    -0.41(0.06)     0.47(0.06)     0.15(0.06)    -0.14    -0.19
 6    -0.04(0.06)    -0.21(0.06)    -0.93(0.06)     0.52(0.06)    -0.07    -0.10
 7    -1.30(0.06)    -1.42(0.06)     0.45(0.06)    -0.21(0.06)    -0.78    -0.07
 8     2.51(0.06)     1.77(0.06)     0.86(0.06)     0.99(0.06)     1.23    -0.42
 9     0.80(0.07)    -0.23(0.07)    -0.69(0.07)    -0.39(0.07)     0.16    -0.59
 A    -0.03(0.04)    -0.73(0.04)     1.61(0.04)    -0.08(0.04)    -0.22    -0.40
 B    -0.13(0.05)    -0.57(0.05)    -0.57(0.05)     0.30(0.05)    -0.20    -0.26
 C     0.00(0.72)     0.00(0.72)     0.00(0.72)     0.00(0.72)     0.00     0.00
 D    -0.05( 0.2)    -0.31( 0.2)    -1.23( 0.2)     1.50( 0.2)    -0.10    -0.15
\end{verbatim}\svend
\spend %.-.-.-.-.-.-.-.-.-.-.-.-.-.-.-.-.-.-.-.-.-.-.-.-.-.-.-.-.-.-.-.-.-.-.

The dipole corrections estimated by CALC will be {\em added} to the corrections
that were applied to the data when they were read in.

%=============================================================================
\section{POLAR SET: Set corrections manually}
\label{.set}

The user may specify values for the dipole angle errors $\derr_i$ and
ellipticity $\eerr_i$ manually. The numbers given by the user are converted to
internal format, and stored (as POLC) in the headers of the given range of
Sets. The default values are zero.

\spbegin %.+.+.+.+.+.+.+.+.+.+.+.+.+.+.+.+.+.+.+.+.+.+.+.+.+.+.+.+.+.+.+.+.+.
\skeyword{POLAR\_OPTION}
\sprompt{(CALC, SHOW, SET, COPY, EDIT...)}
\sdefault{= QUIT:}
\suser{set}
\spend %.-.-.-.-.-.-.-.-.-.-.-.-.-.-.-.-.-.-.-.-.-.-.-.-.-.-.-.-.-.-.-.-.-.-.
%
\spbegin %.+.+.+.+.+.+.+.+.+.+.+.+.+.+.+.+.+.+.+.+.+.+.+.+.+.+.+.+.+.+.+.+.+.
\skeyword{SCN\_NODE}
\sprompt{(input/output 'node' name)}
\sdefault{= "":}
\suser{psr1937.jan93}
\spend %.-.-.-.-.-.-.-.-.-.-.-.-.-.-.-.-.-.-.-.-.-.-.-.-.-.-.-.-.-.-.-.-.-.-.
%
\spbegin %.+.+.+.+.+.+.+.+.+.+.+.+.+.+.+.+.+.+.+.+.+.+.+.+.+.+.+.+.+.+.+.+.+.
\skeyword{SCN\_SETS}
\sprompt{(Set(s) to do:  g.o.f.c.s )}
\sdefault{= "":}
\suser{2.0.0.0.0}
\sinline{For example}
\spend %.-.-.-.-.-.-.-.-.-.-.-.-.-.-.-.-.-.-.-.-.-.-.-.-.-.-.-.-.-.-.-.-.-.-.
%
\spbegin %.+.+.+.+.+.+.+.+.+.+.+.+.+.+.+.+.+.+.+.+.+.+.+.+.+.+.+.+.+.+.+.+.+.
\skeyword{POL\_ROTAN}
\sprompt{(dipole position)}
\sdefault{= 0, 0, 0, 0, 0, 0, 0, 0, 0, 0, 0, 0, 0, 0:}
\suser{\scr}
\spend %.-.-.-.-.-.-.-.-.-.-.-.-.-.-.-.-.-.-.-.-.-.-.-.-.-.-.-.-.-.-.-.-.-.-.
%
\spbegin %.+.+.+.+.+.+.+.+.+.+.+.+.+.+.+.+.+.+.+.+.+.+.+.+.+.+.+.+.+.+.+.+.+.
\skeyword{POL\_ORTHOG}
\sprompt{(dipole orthogonality)}
\sdefault{= 0, 0, 0, 0, 0, 0, 0, 0, 0, 0, 0, 0, 0, 0:}
\suser{\scr}
\spend %.-.-.-.-.-.-.-.-.-.-.-.-.-.-.-.-.-.-.-.-.-.-.-.-.-.-.-.-.-.-.-.-.-.-.
%
\spbegin %.+.+.+.+.+.+.+.+.+.+.+.+.+.+.+.+.+.+.+.+.+.+.+.+.+.+.+.+.+.+.+.+.+.
\skeyword{POL\_X\_ELLIPS}
\sprompt{(X ellipticity)}
\sdefault{= 0, 0, 0, 0, 0, 0, 0, 0, 0, 0, 0, 0, 0, 0:}
\suser{\scr}
\spend %.-.-.-.-.-.-.-.-.-.-.-.-.-.-.-.-.-.-.-.-.-.-.-.-.-.-.-.-.-.-.-.-.-.-.
%
\spbegin %.+.+.+.+.+.+.+.+.+.+.+.+.+.+.+.+.+.+.+.+.+.+.+.+.+.+.+.+.+.+.+.+.+.
\skeyword{POL\_Y\_ELLIPS}
\sprompt{(Y ellipticity)}
\sdefault{= 0, 0, 0, 0, 0, 0, 0, 0, 0, 0, 0, 0, 0, 0:}
\suser{\scr} %
\svbegin \begin{verbatim}
 Sector: 2.0.0.0.0
\end{verbatim}\svend
\spend %.-.-.-.-.-.-.-.-.-.-.-.-.-.-.-.-.-.-.-.-.-.-.-.-.-.-.-.-.-.-.-.-.-.-.

%=============================================================================
\section{POLAR EDIT: Edit corrections}
\label{.edit}

This is similar to POLAR SET above, except that the default values are the
corrections (POLC) that are already stored in the Set headers. Thus, the
existing POLC corrections in each Set header of the given range can be edited
separately:


\spbegin %.+.+.+.+.+.+.+.+.+.+.+.+.+.+.+.+.+.+.+.+.+.+.+.+.+.+.+.+.+.+.+.+.+.
\skeyword{POLAR\_OPTION}
\sprompt{(CALC, SHOW, SET, COPY, EDIT...)}
\sdefault{= QUIT:}
\suser{edit}
\spend %.-.-.-.-.-.-.-.-.-.-.-.-.-.-.-.-.-.-.-.-.-.-.-.-.-.-.-.-.-.-.-.-.-.-.
%
\spbegin %.+.+.+.+.+.+.+.+.+.+.+.+.+.+.+.+.+.+.+.+.+.+.+.+.+.+.+.+.+.+.+.+.+.
\skeyword{SCN\_NODE}
\sprompt{(input/output 'node' name)}
\sdefault{= "":}
\suser{psr1937.jan93}
\spend %.-.-.-.-.-.-.-.-.-.-.-.-.-.-.-.-.-.-.-.-.-.-.-.-.-.-.-.-.-.-.-.-.-.-.
%
\spbegin %.+.+.+.+.+.+.+.+.+.+.+.+.+.+.+.+.+.+.+.+.+.+.+.+.+.+.+.+.+.+.+.+.+.
\skeyword{SCN\_SETS}
\sprompt{(Set(s) to do:  g.o.f.c.s )}
\sdefault{= "":}
\suser{2.0.0.0.0}
\sinline{For example} %
\svbegin \begin{verbatim}
 Sector: 2.0.0.0.0
\end{verbatim}\svend
\spend %.-.-.-.-.-.-.-.-.-.-.-.-.-.-.-.-.-.-.-.-.-.-.-.-.-.-.-.-.-.-.-.-.-.-.
%
\spbegin %.+.+.+.+.+.+.+.+.+.+.+.+.+.+.+.+.+.+.+.+.+.+.+.+.+.+.+.+.+.+.+.+.+.
\skeyword{POL\_ROTAN}
\sprompt{(dipole position)}
\sdefault{= 0, 0, 0, 0, 0, 0, 0, 0, 0, 0, 0, 0, 0, 0:}
\suser{\scr}
\spend %.-.-.-.-.-.-.-.-.-.-.-.-.-.-.-.-.-.-.-.-.-.-.-.-.-.-.-.-.-.-.-.-.-.-.
%
\spbegin %.+.+.+.+.+.+.+.+.+.+.+.+.+.+.+.+.+.+.+.+.+.+.+.+.+.+.+.+.+.+.+.+.+.
\skeyword{POL\_ORTHOG}
\sprompt{(dipole orthogonality)}
\sdefault{= 0, 0, 0, 0, 0, 0, 0, 0, 0, 0, 0, 0, 0, 0:}
\suser{\scr}
\spend %.-.-.-.-.-.-.-.-.-.-.-.-.-.-.-.-.-.-.-.-.-.-.-.-.-.-.-.-.-.-.-.-.-.-.
%
\spbegin %.+.+.+.+.+.+.+.+.+.+.+.+.+.+.+.+.+.+.+.+.+.+.+.+.+.+.+.+.+.+.+.+.+.
\skeyword{POL\_X\_ELLIPS}
\sprompt{(X ellipticity)}
\sdefault{= 0, 0, 0, 0, 0, 0, 0, 0, 0, 0, 0, 0, 0, 0:}
\suser{\scr}
\spend %.-.-.-.-.-.-.-.-.-.-.-.-.-.-.-.-.-.-.-.-.-.-.-.-.-.-.-.-.-.-.-.-.-.-.
%
\spbegin %.+.+.+.+.+.+.+.+.+.+.+.+.+.+.+.+.+.+.+.+.+.+.+.+.+.+.+.+.+.+.+.+.+.
\skeyword{POL\_Y\_ELLIPS}
\sprompt{(Y ellipticity)}
\sdefault{= 0, 0, 0, 0, 0, 0, 0, 0, 0, 0, 0, 0, 0, 0:}
\suser{\scr}
\spend %.-.-.-.-.-.-.-.-.-.-.-.-.-.-.-.-.-.-.-.-.-.-.-.-.-.-.-.-.-.-.-.-.-.-.


%=============================================================================
\section{POLAR ZERO: Zero corrections}
\label{.zero}

For the specified range of sets, the POLC corrections in the set header are set
to zero:

\spbegin %.+.+.+.+.+.+.+.+.+.+.+.+.+.+.+.+.+.+.+.+.+.+.+.+.+.+.+.+.+.+.+.+.+.
\skeyword{POLAR\_OPTION}
\sprompt{(CALC, SHOW, SET, COPY, EDIT...)}
\sdefault{= QUIT:}
\suser{zero}
\spend %.-.-.-.-.-.-.-.-.-.-.-.-.-.-.-.-.-.-.-.-.-.-.-.-.-.-.-.-.-.-.-.-.-.-.
%
\spbegin %.+.+.+.+.+.+.+.+.+.+.+.+.+.+.+.+.+.+.+.+.+.+.+.+.+.+.+.+.+.+.+.+.+.
\skeyword{SCN\_NODE}
\sprompt{(input/output 'node' name)}
\sdefault{= "":}
\suser{psr1937.jan93}
\spend %.-.-.-.-.-.-.-.-.-.-.-.-.-.-.-.-.-.-.-.-.-.-.-.-.-.-.-.-.-.-.-.-.-.-.
%
\spbegin %.+.+.+.+.+.+.+.+.+.+.+.+.+.+.+.+.+.+.+.+.+.+.+.+.+.+.+.+.+.+.+.+.+.
\skeyword{SCN\_SETS}
\sprompt{(Set(s) to do:  g.o.f.c.s )}
\sdefault{= "":}
\suser{2.0.0.0.0} %
\svbegin \begin{verbatim}
 Sector: 2.0.0.0.0
\end{verbatim}\svend
\spend %.-.-.-.-.-.-.-.-.-.-.-.-.-.-.-.-.-.-.-.-.-.-.-.-.-.-.-.-.-.-.-.-.-.-.
%
\spbegin %.+.+.+.+.+.+.+.+.+.+.+.+.+.+.+.+.+.+.+.+.+.+.+.+.+.+.+.+.+.+.+.+.+.
\skeyword{POLAR\_OPTION}
\sprompt{(CALC, SHOW, SET, COPY, EDIT...)}
\sdefault{= QUIT:}
\suser{\scr}
\spend %.-.-.-.-.-.-.-.-.-.-.-.-.-.-.-.-.-.-.-.-.-.-.-.-.-.-.-.-.-.-.-.-.-.-.


%=============================================================================
\section{POLAR COPY: Copy corrections from somewhere else}
\label{.copy}

The polarisation corrections $\derr_i$ and $\eerr_i$  are calculated with the
help of a strong, unpolarised calibrator source, using the option POLAR CALC
(see above). In order to use these corrections to correct a real observation,
they must be transferred (copied) from the Set header of the calibrator to the
Set header(s) of the observed object.

There are two possibilities: The calibrator observation (and thus the desired
corrections) may be stored in a separate SCN-file (node), or they may be stored
in another `group' in the same SCN-file as the observed object. Below, an
example is given for both situations:

If the calibrator observation is stored in a separate SCN-file:

\spbegin %.+.+.+.+.+.+.+.+.+.+.+.+.+.+.+.+.+.+.+.+.+.+.+.+.+.+.+.+.+.+.+.+.+.
\skeyword{POLAR\_OPTION}
\sprompt{(CALC, SHOW, SET, COPY, EDIT...)}
\sdefault{= QUIT:}
\suser{copy}
\spend %.-.-.-.-.-.-.-.-.-.-.-.-.-.-.-.-.-.-.-.-.-.-.-.-.-.-.-.-.-.-.-.-.-.-.
%
\spbegin %.+.+.+.+.+.+.+.+.+.+.+.+.+.+.+.+.+.+.+.+.+.+.+.+.+.+.+.+.+.+.+.+.+.
\skeyword{SCN\_NODE}
\sprompt{(input/output 'node' name)}
\sdefault{= "":}
\suser{psr1937.jan93}
\spend %.-.-.-.-.-.-.-.-.-.-.-.-.-.-.-.-.-.-.-.-.-.-.-.-.-.-.-.-.-.-.-.-.-.-.
%
\spbegin %.+.+.+.+.+.+.+.+.+.+.+.+.+.+.+.+.+.+.+.+.+.+.+.+.+.+.+.+.+.+.+.+.+.
\skeyword{SCN\_LOOPS}
\sprompt{(niter, Setincr(g.o.f.c.s) ....)}
\sdefault{= "":}
\suser{\scr}
\spend %.-.-.-.-.-.-.-.-.-.-.-.-.-.-.-.-.-.-.-.-.-.-.-.-.-.-.-.-.-.-.-.-.-.-.
%
\spbegin %.+.+.+.+.+.+.+.+.+.+.+.+.+.+.+.+.+.+.+.+.+.+.+.+.+.+.+.+.+.+.+.+.+.
\skeyword{SCN\_SETS}
\sprompt{(Set(s) to do:  g.o.f.c.s )}
\sdefault{= "":}
\suser{0}
\spend %.-.-.-.-.-.-.-.-.-.-.-.-.-.-.-.-.-.-.-.-.-.-.-.-.-.-.-.-.-.-.-.-.-.-.
%
\spbegin %.+.+.+.+.+.+.+.+.+.+.+.+.+.+.+.+.+.+.+.+.+.+.+.+.+.+.+.+.+.+.+.+.+.
\skeyword{USE\_SCN\_NODE}
\sprompt{(input node name)}
\sdefault{= *:}
\suser{?} %
\svbegin \begin{verbatim}
 Specify the node name from which the corrections should be calculated.
 * indicates the same as the output node name.
\end{verbatim}\svend
\spend %.-.-.-.-.-.-.-.-.-.-.-.-.-.-.-.-.-.-.-.-.-.-.-.-.-.-.-.-.-.-.-.-.-.-.
%
\spbegin %.+.+.+.+.+.+.+.+.+.+.+.+.+.+.+.+.+.+.+.+.+.+.+.+.+.+.+.+.+.+.+.+.+.
\skeyword{USE\_SCN\_NODE}
\sprompt{(input node name)}
\sdefault{= *:}
\suser{3c48}
\spend %.-.-.-.-.-.-.-.-.-.-.-.-.-.-.-.-.-.-.-.-.-.-.-.-.-.-.-.-.-.-.-.-.-.-.
%
\spbegin %.+.+.+.+.+.+.+.+.+.+.+.+.+.+.+.+.+.+.+.+.+.+.+.+.+.+.+.+.+.+.+.+.+.
\skeyword{USE\_SCN\_SETS}
\sprompt{(Set(s) of input uv-data Sectors:  g.o.f.c.s)}
\sdefault{= "":}
\suser{2.0.0.0.0} %
\svbegin \begin{verbatim}
 Sector: 0.0.0.0.0
 Sector: 0.0.0.4.0
\end{verbatim}\svend
\spend %.-.-.-.-.-.-.-.-.-.-.-.-.-.-.-.-.-.-.-.-.-.-.-.-.-.-.-.-.-.-.-.-.-.-.

If the calibrator observation is stored in the same SCN-file (e.g. as group 1,
while the observed object is stored as group 0), the process runs as follows:

\spbegin %.+.+.+.+.+.+.+.+.+.+.+.+.+.+.+.+.+.+.+.+.+.+.+.+.+.+.+.+.+.+.+.+.+.
\skeyword{POLAR\_OPTION}
\sprompt{(CALC, SHOW, SET, COPY, EDIT...)}
\sdefault{= QUIT:}
\suser{copy}
\spend %.-.-.-.-.-.-.-.-.-.-.-.-.-.-.-.-.-.-.-.-.-.-.-.-.-.-.-.-.-.-.-.-.-.-.
%
\spbegin %.+.+.+.+.+.+.+.+.+.+.+.+.+.+.+.+.+.+.+.+.+.+.+.+.+.+.+.+.+.+.+.+.+.
\skeyword{SCN\_NODE}
\sprompt{(input/output 'node' name)}
\sdefault{= "":}
\suser{psr1937.jan93}
\spend %.-.-.-.-.-.-.-.-.-.-.-.-.-.-.-.-.-.-.-.-.-.-.-.-.-.-.-.-.-.-.-.-.-.-.
%
\spbegin %.+.+.+.+.+.+.+.+.+.+.+.+.+.+.+.+.+.+.+.+.+.+.+.+.+.+.+.+.+.+.+.+.+.
\skeyword{SCN\_LOOPS}
\sprompt{(niter, Setincr(g.o.f.c.s) ....)}
\sdefault{= "":}
\suser{\scr}
\spend %.-.-.-.-.-.-.-.-.-.-.-.-.-.-.-.-.-.-.-.-.-.-.-.-.-.-.-.-.-.-.-.-.-.-.
%
\spbegin %.+.+.+.+.+.+.+.+.+.+.+.+.+.+.+.+.+.+.+.+.+.+.+.+.+.+.+.+.+.+.+.+.+.
\skeyword{SCN\_SETS}
\sprompt{(Set(s) to do:  g.o.f.c.s )}
\sdefault{= "":}
\suser{0}
\spend %.-.-.-.-.-.-.-.-.-.-.-.-.-.-.-.-.-.-.-.-.-.-.-.-.-.-.-.-.-.-.-.-.-.-.
%
\spbegin %.+.+.+.+.+.+.+.+.+.+.+.+.+.+.+.+.+.+.+.+.+.+.+.+.+.+.+.+.+.+.+.+.+.
\skeyword{USE\_SCN\_NODE}
\sprompt{(input node name)}
\sdefault{= *:}
\suser{\scr}
\spend %.-.-.-.-.-.-.-.-.-.-.-.-.-.-.-.-.-.-.-.-.-.-.-.-.-.-.-.-.-.-.-.-.-.-.
%
\spbegin %.+.+.+.+.+.+.+.+.+.+.+.+.+.+.+.+.+.+.+.+.+.+.+.+.+.+.+.+.+.+.+.+.+.
\skeyword{USE\_SCN\_SETS}
\sprompt{(Set(s) of input uv-data Sectors:  g.o.f.c.s)}
\sdefault{= "":}
\suser{1.0.0.0.0} %
\svbegin \begin{verbatim}
 Sector: 0.0.0.0.0
 Sector: 0.0.0.4.0
\end{verbatim}\svend
\spend %.-.-.-.-.-.-.-.-.-.-.-.-.-.-.-.-.-.-.-.-.-.-.-.-.-.-.-.-.-.-.-.-.-.-.


%=============================================================================
\section{POLAR VZERO: X-Y Phase Zero Difference, assuming V=0}
\label{.vzero}

The VZERO option deals with the determination and manipulation of the {\em
Phase Zero Difference (PZD)} between the X and Y dipoles. It is invoked in the
following way:

\spbegin %.+.+.+.+.+.+.+.+.+.+.+.+.+.+.+.+.+.+.+.+.+.+.+.+.+.+.+.+.+.+.+.+.+.
\skeyword{POLAR\_OPTION}
\sprompt{(CALC, SHOW, SET, COPY, EDIT...)}
\sdefault{= QUIT:}
\suser{vzero} %
\spend %.-.-.-.-.-.-.-.-.-.-.-.-.-.-.-.-.-.-.-.-.-.-.-.-.-.-.-.-.-.-.-.-.-.-.


%------------------------------------------------------------------------------
\subsection{POLAR VZERO CALC: Calculate and show}
\label{.vzero.calc}

An Phase Zero Difference (PZD) between the X-dipoles an the Y-dipoles affects
the phase of the complex gain factors $\cGain_{xy}$ and $\cGain_{yx}$ (but not
$\cGain_{xx}$ and $\cGain_{yy}$). The PZD is determined with the help of a
strong calibrator source, which must have a relatively large U-component of
linear polarisation, and an accurately known amount of circular polarisation
(Stokes V). The latter is important, since a wrong value for V will be
incorrectly interpreted as a PZD. Since the V is difficult to measure
accurately, it is safer to use a calibrator that `should not' have any circular
polarisation: $V=0$.

The algorithm uses \eqref{.equ.pol.010} and \eqref{.equ.pol.018}. It is assumed
that the `leakage' of the strong I-term is eliminated by means
\textref{POLAR CALC}{.calc}: $\Apol_{xy} = \Apol_{yx} = 0$.  It is also assumed
that the gain and phase errors have also been well-calibrated: $\lerr_i =
\perr_i = 0$. Thus, \eqref{.equ.pol.010} reduce to:
\begin{eqnarray}
  \cVis_{xy} & = & -U\exp (-i\pzd) \nonumber \\
  \cVis_{yx} & = & -U\exp (+i\pzd)
\end{eqnarray}

The calculation of the PZD angle ($\pzd$) is now quite straightforward. Because
of the low S/N of the `cross-terms' $\cVis_{xy}$, as many data (Sets, HA-range)
should be used in the estimation as possible. However, the PZD may change with
time.

\spbegin %.+.+.+.+.+.+.+.+.+.+.+.+.+.+.+.+.+.+.+.+.+.+.+.+.+.+.+.+.+.+.+.+.+.
\skeyword{VZERO\_OPTION}
\sprompt{(CALC, APPLY, ASK, MANUAL, SCAN...)}
\sdefault{= QUIT:}
\suser{calc}
\spend %.-.-.-.-.-.-.-.-.-.-.-.-.-.-.-.-.-.-.-.-.-.-.-.-.-.-.-.-.-.-.-.-.-.-.
%
\spbegin %.+.+.+.+.+.+.+.+.+.+.+.+.+.+.+.+.+.+.+.+.+.+.+.+.+.+.+.+.+.+.+.+.+.
\skeyword{SCN\_NODE}
\sprompt{(input/output 'node' name)}
\sdefault{= "PSR1937.JAN93":}
\suser{\scr}
\spend %.-.-.-.-.-.-.-.-.-.-.-.-.-.-.-.-.-.-.-.-.-.-.-.-.-.-.-.-.-.-.-.-.-.-.
%
\spbegin %.+.+.+.+.+.+.+.+.+.+.+.+.+.+.+.+.+.+.+.+.+.+.+.+.+.+.+.+.+.+.+.+.+.
\skeyword{SCN\_LOOPS}
\sprompt{(niter, Setincr(g.o.f.c.s) ....)}
\sdefault{= "":}
\suser{\scr}
\spend %.-.-.-.-.-.-.-.-.-.-.-.-.-.-.-.-.-.-.-.-.-.-.-.-.-.-.-.-.-.-.-.-.-.-.
%
\spbegin %.+.+.+.+.+.+.+.+.+.+.+.+.+.+.+.+.+.+.+.+.+.+.+.+.+.+.+.+.+.+.+.+.+.
\skeyword{SCN\_SETS}
\sprompt{(Set(s) to do:  g.o.f.c.s )}
\sdefault{= "":}
\suser{0}
\spend %.-.-.-.-.-.-.-.-.-.-.-.-.-.-.-.-.-.-.-.-.-.-.-.-.-.-.-.-.-.-.-.-.-.-.
%
\spbegin %.+.+.+.+.+.+.+.+.+.+.+.+.+.+.+.+.+.+.+.+.+.+.+.+.+.+.+.+.+.+.+.+.+.
\skeyword{HA\_RANGE}
\sprompt{(DEG) (HA range)}
\sdefault{= *:}
\suser{\scr} %
\svbegin \begin{verbatim}
 All cross interferometers pre-selected
\end{verbatim}\svend
\spend %.-.-.-.-.-.-.-.-.-.-.-.-.-.-.-.-.-.-.-.-.-.-.-.-.-.-.-.-.-.-.-.-.-.-.
%
\spbegin %.+.+.+.+.+.+.+.+.+.+.+.+.+.+.+.+.+.+.+.+.+.+.+.+.+.+.+.+.+.+.+.+.+.
\skeyword{SELECT\_IFRS}
\sprompt{(Select/de-select ifrs)}
\sdefault{= "":}
\suser{\scr} %
\svbegin \begin{verbatim}
 Sector: 0.0.0.0
 Sector: 0.0.0.4
\end{verbatim}\svend
\spend %.-.-.-.-.-.-.-.-.-.-.-.-.-.-.-.-.-.-.-.-.-.-.-.-.-.-.-.-.-.-.-.-.-.-.

The resulting PZD angle is given as follows:

%\input ncalib_polar_vzero3.scr
\spbegin %.+.+.+.+.+.+.+.+.+.+.+.+.+.+.+.+.+.+.+.+.+.+.+.+.+.+.+.+.+.+.+.+.+.
\svbegin \begin{verbatim}
 A complex angle of 0.54+0.12I(0.01+0.01I) or 12.44 degrees
\end{verbatim}\svend
\spend %.-.-.-.-.-.-.-.-.-.-.-.-.-.-.-.-.-.-.-.-.-.-.-.-.-.-.-.-.-.-.-.-.-.-.

This message is somewhat confusing as the interesting information in the
``complex angle'' has been truncated in the printing (the relative sizes of the
real and imaginary parts). The angle given in degrees is somewhat more
informative.



%------------------------------------------------------------------------------
\subsection{POLAR VZERO APPLY: Calculate, show and apply}
\label{.vzero.apply}

The PZD angle is `applied' by adjusting the phase in the `other corrections'
(OTHC) in the Scan headers. This is done for the entire specified range (Sets
and HA-range).

\spbegin %.+.+.+.+.+.+.+.+.+.+.+.+.+.+.+.+.+.+.+.+.+.+.+.+.+.+.+.+.+.+.+.+.+.
\skeyword{VZERO\_OPTION}
\sprompt{(CALC, APPLY, ASK, MANUAL, SCAN...)}
\sdefault{= QUIT:}
\suser{apply}
\spend %.-.-.-.-.-.-.-.-.-.-.-.-.-.-.-.-.-.-.-.-.-.-.-.-.-.-.-.-.-.-.-.-.-.-.
%
\spbegin %.+.+.+.+.+.+.+.+.+.+.+.+.+.+.+.+.+.+.+.+.+.+.+.+.+.+.+.+.+.+.+.+.+.
\skeyword{SCN\_NODE}
\sprompt{(input/output 'node' name)}
\sdefault{= "":}
\suser{psr1937.jan93}
\spend %.-.-.-.-.-.-.-.-.-.-.-.-.-.-.-.-.-.-.-.-.-.-.-.-.-.-.-.-.-.-.-.-.-.-.
%
\spbegin %.+.+.+.+.+.+.+.+.+.+.+.+.+.+.+.+.+.+.+.+.+.+.+.+.+.+.+.+.+.+.+.+.+.
\skeyword{SCN\_LOOPS}
\sprompt{(niter, Setincr(g.o.f.c.s) ....)}
\sdefault{= "":}
\suser{\scr}
\spend %.-.-.-.-.-.-.-.-.-.-.-.-.-.-.-.-.-.-.-.-.-.-.-.-.-.-.-.-.-.-.-.-.-.-.
%
\spbegin %.+.+.+.+.+.+.+.+.+.+.+.+.+.+.+.+.+.+.+.+.+.+.+.+.+.+.+.+.+.+.+.+.+.
\skeyword{SCN\_SETS}
\sprompt{(Set(s) to do:  g.o.f.c.s )}
\sdefault{= "":}
\suser{0}
\spend %.-.-.-.-.-.-.-.-.-.-.-.-.-.-.-.-.-.-.-.-.-.-.-.-.-.-.-.-.-.-.-.-.-.-.
%
\spbegin %.+.+.+.+.+.+.+.+.+.+.+.+.+.+.+.+.+.+.+.+.+.+.+.+.+.+.+.+.+.+.+.+.+.
\skeyword{HA\_RANGE}
\sprompt{(DEG) (HA range)}
\sdefault{= *:}
\suser{\scr} %
\svbegin \begin{verbatim}
 All cross interferometers pre-selected
\end{verbatim}\svend
\spend %.-.-.-.-.-.-.-.-.-.-.-.-.-.-.-.-.-.-.-.-.-.-.-.-.-.-.-.-.-.-.-.-.-.-.
%
\spbegin %.+.+.+.+.+.+.+.+.+.+.+.+.+.+.+.+.+.+.+.+.+.+.+.+.+.+.+.+.+.+.+.+.+.
\skeyword{SELECT\_IFRS}
\sprompt{(Select/de-select ifrs)}
\sdefault{= "":}
\suser{\scr} %
\svbegin \begin{verbatim}
 Sector: 0.0.0.0
 Sector: 0.0.0.4
 A complex angle of 0.55+0.01I(0.01+0.01I) or 1.00 degrees
\end{verbatim}\svend
\spend %.-.-.-.-.-.-.-.-.-.-.-.-.-.-.-.-.-.-.-.-.-.-.-.-.-.-.-.-.-.-.-.-.-.-.

%------------------------------------------------------------------------------
\subsection{POLAR VZERO ASK: Calculate, check with user, and apply}
\label{.vzero.ask}

The user may influence the result by giving another PZD, which will be
`applied' for the entire specified range (Sets and HA-range). The default is
the calculated value.

\spbegin %.+.+.+.+.+.+.+.+.+.+.+.+.+.+.+.+.+.+.+.+.+.+.+.+.+.+.+.+.+.+.+.+.+.
\skeyword{VZERO\_OPTION}
\sprompt{(CALC, APPLY, ASK, MANUAL, SCAN...)}
\sdefault{= QUIT:}
\suser{ask}
\spend %.-.-.-.-.-.-.-.-.-.-.-.-.-.-.-.-.-.-.-.-.-.-.-.-.-.-.-.-.-.-.-.-.-.-.
%
\spbegin %.+.+.+.+.+.+.+.+.+.+.+.+.+.+.+.+.+.+.+.+.+.+.+.+.+.+.+.+.+.+.+.+.+.
\skeyword{SCN\_NODE}
\sprompt{(input/output 'node' name)}
\sdefault{= "PSR1937.JAN93":}
\suser{\scr}
\spend %.-.-.-.-.-.-.-.-.-.-.-.-.-.-.-.-.-.-.-.-.-.-.-.-.-.-.-.-.-.-.-.-.-.-.
%
\spbegin %.+.+.+.+.+.+.+.+.+.+.+.+.+.+.+.+.+.+.+.+.+.+.+.+.+.+.+.+.+.+.+.+.+.
\skeyword{SCN\_LOOPS}
\sprompt{(niter, Setincr(g.o.f.c.s) ....)}
\sdefault{= "":}
\suser{\scr}
\spend %.-.-.-.-.-.-.-.-.-.-.-.-.-.-.-.-.-.-.-.-.-.-.-.-.-.-.-.-.-.-.-.-.-.-.
%
\spbegin %.+.+.+.+.+.+.+.+.+.+.+.+.+.+.+.+.+.+.+.+.+.+.+.+.+.+.+.+.+.+.+.+.+.
\skeyword{SCN\_SETS}
\sprompt{(Set(s) to do:  g.o.f.c.s )}
\sdefault{= 0:}
\suser{\scr}
\spend %.-.-.-.-.-.-.-.-.-.-.-.-.-.-.-.-.-.-.-.-.-.-.-.-.-.-.-.-.-.-.-.-.-.-.
%
\spbegin %.+.+.+.+.+.+.+.+.+.+.+.+.+.+.+.+.+.+.+.+.+.+.+.+.+.+.+.+.+.+.+.+.+.
\skeyword{HA\_RANGE}
\sprompt{(DEG) (HA range)}
\sdefault{= *:}
\suser{\scr} %
\svbegin \begin{verbatim}
   0123456789ABCD
 0 -+++++++++++++
 1  -++++++++++++
 2   -+++++++++++
 3    -++++++++++
 4     -+++++++++
 5      -++++++++
 6       -+++++++
 7        -++++++
 8         -+++++
 9          -++++
 A           -+++
 B            -++
 C             -+
 D              -
\end{verbatim}\svend
\spend %.-.-.-.-.-.-.-.-.-.-.-.-.-.-.-.-.-.-.-.-.-.-.-.-.-.-.-.-.-.-.-.-.-.-.
%
\spbegin %.+.+.+.+.+.+.+.+.+.+.+.+.+.+.+.+.+.+.+.+.+.+.+.+.+.+.+.+.+.+.+.+.+.
\skeyword{SELECT\_IFRS}
\sprompt{(Select/de-select ifrs)}
\sdefault{= "":}
\suser{\scr} %
\svbegin \begin{verbatim}
 Sector: 0.0.0.0
 Sector: 0.0.0.4
 A complex angle of 0.55-0.00I(0.01+0.01I) or -0.19 degrees
\end{verbatim}\svend
\spend %.-.-.-.-.-.-.-.-.-.-.-.-.-.-.-.-.-.-.-.-.-.-.-.-.-.-.-.-.-.-.-.-.-.-.
%
\spbegin %.+.+.+.+.+.+.+.+.+.+.+.+.+.+.+.+.+.+.+.+.+.+.+.+.+.+.+.+.+.+.+.+.+.
\skeyword{VZERO\_PHASE}
\sprompt{(X-Y difference)}
\sdefault{= -0.1857903:}
\suser{\scr}
\sinline{Use the calculated value}
\spend %.-.-.-.-.-.-.-.-.-.-.-.-.-.-.-.-.-.-.-.-.-.-.-.-.-.-.-.-.-.-.-.-.-.-.

%------------------------------------------------------------------------------
\subsection{POLAR VZERO MANUAL: Ask for and apply}
\label{.vzero.manual}

The PZD angle given by the user is `applied' to the entire specified range
(Sets and HA-range). The default value is $PZD =0^\circ$.

\spbegin %.+.+.+.+.+.+.+.+.+.+.+.+.+.+.+.+.+.+.+.+.+.+.+.+.+.+.+.+.+.+.+.+.+.
\skeyword{VZERO\_OPTION}
\sprompt{(CALC, APPLY, ASK, MANUAL, SCAN...)}
\sdefault{= QUIT:}
\suser{manual}
\spend %.-.-.-.-.-.-.-.-.-.-.-.-.-.-.-.-.-.-.-.-.-.-.-.-.-.-.-.-.-.-.-.-.-.-.
%
\spbegin %.+.+.+.+.+.+.+.+.+.+.+.+.+.+.+.+.+.+.+.+.+.+.+.+.+.+.+.+.+.+.+.+.+.
\skeyword{SCN\_NODE}
\sprompt{(input/output 'node' name)}
\sdefault{= "PSR1937.JAN93":}
\suser{\scr}
\spend %.-.-.-.-.-.-.-.-.-.-.-.-.-.-.-.-.-.-.-.-.-.-.-.-.-.-.-.-.-.-.-.-.-.-.
%
\spbegin %.+.+.+.+.+.+.+.+.+.+.+.+.+.+.+.+.+.+.+.+.+.+.+.+.+.+.+.+.+.+.+.+.+.
\skeyword{SCN\_LOOPS}
\sprompt{(niter, Setincr(g.o.f.c.s) ....)}
\sdefault{= "":}
\suser{\scr}
\spend %.-.-.-.-.-.-.-.-.-.-.-.-.-.-.-.-.-.-.-.-.-.-.-.-.-.-.-.-.-.-.-.-.-.-.
%
\spbegin %.+.+.+.+.+.+.+.+.+.+.+.+.+.+.+.+.+.+.+.+.+.+.+.+.+.+.+.+.+.+.+.+.+.
\skeyword{SCN\_SETS}
\sprompt{(Set(s) to do:  g.o.f.c.s )}
\sdefault{= 0:}
\suser{\scr}
\spend %.-.-.-.-.-.-.-.-.-.-.-.-.-.-.-.-.-.-.-.-.-.-.-.-.-.-.-.-.-.-.-.-.-.-.
%
\spbegin %.+.+.+.+.+.+.+.+.+.+.+.+.+.+.+.+.+.+.+.+.+.+.+.+.+.+.+.+.+.+.+.+.+.
\skeyword{HA\_RANGE}
\sprompt{(DEG) (HA range)}
\sdefault{= *:}
\suser{\scr}
\spend %.-.-.-.-.-.-.-.-.-.-.-.-.-.-.-.-.-.-.-.-.-.-.-.-.-.-.-.-.-.-.-.-.-.-.
%
\spbegin %.+.+.+.+.+.+.+.+.+.+.+.+.+.+.+.+.+.+.+.+.+.+.+.+.+.+.+.+.+.+.+.+.+.
\skeyword{VZERO\_PHASE}
\sprompt{(X-Y difference)}
\sdefault{= 0:}
\suser{-1}
\sinline{User specifies a PZD = -1 degrees}
\spend %.-.-.-.-.-.-.-.-.-.-.-.-.-.-.-.-.-.-.-.-.-.-.-.-.-.-.-.-.-.-.-.-.-.-.

%------------------------------------------------------------------------------
\subsection{POLAR VZERO SCAN: Calculate and apply on a per-scan basis}
\label{.vzero.scan}

The change of the PZD angle as a function of time can be studied by estimating
it for each Scan (HA) separately. The S/N of the estimation will necessarily be
low.

\spbegin %.+.+.+.+.+.+.+.+.+.+.+.+.+.+.+.+.+.+.+.+.+.+.+.+.+.+.+.+.+.+.+.+.+.
\skeyword{VZERO\_OPTION}
\sprompt{(CALC, APPLY, ASK, MANUAL, SCAN...)}
\sdefault{= QUIT:}
\suser{scan}
\spend %.-.-.-.-.-.-.-.-.-.-.-.-.-.-.-.-.-.-.-.-.-.-.-.-.-.-.-.-.-.-.-.-.-.-.
%
\spbegin %.+.+.+.+.+.+.+.+.+.+.+.+.+.+.+.+.+.+.+.+.+.+.+.+.+.+.+.+.+.+.+.+.+.
\skeyword{SCN\_NODE}
\sprompt{(input/output 'node' name)}
\sdefault{= "PSR1937.JAN93":}
\suser{\scr}
\spend %.-.-.-.-.-.-.-.-.-.-.-.-.-.-.-.-.-.-.-.-.-.-.-.-.-.-.-.-.-.-.-.-.-.-.
%
\spbegin %.+.+.+.+.+.+.+.+.+.+.+.+.+.+.+.+.+.+.+.+.+.+.+.+.+.+.+.+.+.+.+.+.+.
\skeyword{SCN\_LOOPS}
\sprompt{(niter, Setincr(g.o.f.c.s) ....)}
\sdefault{= "":}
\suser{\scr}
\spend %.-.-.-.-.-.-.-.-.-.-.-.-.-.-.-.-.-.-.-.-.-.-.-.-.-.-.-.-.-.-.-.-.-.-.
%
\spbegin %.+.+.+.+.+.+.+.+.+.+.+.+.+.+.+.+.+.+.+.+.+.+.+.+.+.+.+.+.+.+.+.+.+.
\skeyword{SCN\_SETS}
\sprompt{(Set(s) to do:  g.o.f.c.s )}
\sdefault{= 0:}
\suser{\scr}
\spend %.-.-.-.-.-.-.-.-.-.-.-.-.-.-.-.-.-.-.-.-.-.-.-.-.-.-.-.-.-.-.-.-.-.-.
%
\spbegin %.+.+.+.+.+.+.+.+.+.+.+.+.+.+.+.+.+.+.+.+.+.+.+.+.+.+.+.+.+.+.+.+.+.
\skeyword{HA\_RANGE}
\sprompt{(DEG) (HA range)}
\sdefault{= *:}
\suser{\scr} %
\svbegin \begin{verbatim}
   0123456789ABCD
 0 -+++++++++++++
 1  -++++++++++++
 2   -+++++++++++
 3    -++++++++++
 4     -+++++++++
 5      -++++++++
 6       -+++++++
 7        -++++++
 8         -+++++
 9          -++++
 A           -+++
 B            -++
 C             -+
 D              -
\end{verbatim}\svend
\spend %.-.-.-.-.-.-.-.-.-.-.-.-.-.-.-.-.-.-.-.-.-.-.-.-.-.-.-.-.-.-.-.-.-.-.
%
\spbegin %.+.+.+.+.+.+.+.+.+.+.+.+.+.+.+.+.+.+.+.+.+.+.+.+.+.+.+.+.+.+.+.+.+.
\skeyword{SELECT\_IFRS}
\sprompt{(Select/de-select ifrs)}
\sdefault{= "":}
\suser{\scr}
\spend %.-.-.-.-.-.-.-.-.-.-.-.-.-.-.-.-.-.-.-.-.-.-.-.-.-.-.-.-.-.-.-.-.-.-.

%------------------------------------------------------------------------------
\subsection{POLAR VZERO COPY: Calculate from input and apply to output}
\label{.vzero.copy}

The PZD angle is calculated with the help of a strong, unpolarised calibrator
source. In order to use it to correct a real observation, it must be
transferred (copied) from the Scan header(s) of the calibrator to the Scan
header(s) of the observed object.

There are two possibilities: The calibrator observation (and thus the desired
corrections) may be stored in a separate SCN-file (node), or it may be stored
in another group in the same SCN-file as the observed object. NB: In the latter
case, it could even be the same observation!  In both cases, the PZD is
calculated from a range of data (Sets and HA-range) in the INPUT\_NODE, and
`applied' (i.e. stored as `other corrections' (OTHC)) to the Scan Headers in
the SCAN\_NODE. In the following example, the calibrator is stored in the same
SCN-file as group 1, while the observed object is stored as group 0):

\spbegin %.+.+.+.+.+.+.+.+.+.+.+.+.+.+.+.+.+.+.+.+.+.+.+.+.+.+.+.+.+.+.+.+.+.
\skeyword{VZERO\_OPTION}
\sprompt{(CALC, APPLY, ASK, MANUAL, SCAN...)}
\sdefault{= QUIT:}
\suser{copy}
\spend %.-.-.-.-.-.-.-.-.-.-.-.-.-.-.-.-.-.-.-.-.-.-.-.-.-.-.-.-.-.-.-.-.-.-.
%
\spbegin %.+.+.+.+.+.+.+.+.+.+.+.+.+.+.+.+.+.+.+.+.+.+.+.+.+.+.+.+.+.+.+.+.+.
\skeyword{SCN\_NODE}
\sprompt{(input/output 'node' name)}
\sdefault{= ""}
\suser{psr1937.jan93}
\sinline{SCN-file}
\spend %.-.-.-.-.-.-.-.-.-.-.-.-.-.-.-.-.-.-.-.-.-.-.-.-.-.-.-.-.-.-.-.-.-.-.
%
\spbegin %.+.+.+.+.+.+.+.+.+.+.+.+.+.+.+.+.+.+.+.+.+.+.+.+.+.+.+.+.+.+.+.+.+.
\skeyword{SCN\_LOOPS}
\sprompt{(niter, Setincr(g.o.f.c.s) ....)}
\sdefault{= "":}
\suser{\scr}
\spend %.-.-.-.-.-.-.-.-.-.-.-.-.-.-.-.-.-.-.-.-.-.-.-.-.-.-.-.-.-.-.-.-.-.-.
%
\spbegin %.+.+.+.+.+.+.+.+.+.+.+.+.+.+.+.+.+.+.+.+.+.+.+.+.+.+.+.+.+.+.+.+.+.
\skeyword{SCN\_SETS}
\sprompt{(Set(s) to do:  g.o.f.c.s )}
\sdefault{= 0:}
\suser{0.*.*.*.*}
\sinline{Apply the PZD to all sets of job 0}
\spend %.-.-.-.-.-.-.-.-.-.-.-.-.-.-.-.-.-.-.-.-.-.-.-.-.-.-.-.-.-.-.-.-.-.-.
%
\spbegin %.+.+.+.+.+.+.+.+.+.+.+.+.+.+.+.+.+.+.+.+.+.+.+.+.+.+.+.+.+.+.+.+.+.
\skeyword{USE\_SCN\_NODE}
\sprompt{(input node name)}
\sdefault{= *:}
\suser{?} %
\svbegin \begin{verbatim}
 Specify the node name from which the corrections should be calculated.
 * indicates the same as the output node name.
\end{verbatim}\svend
\spend %.-.-.-.-.-.-.-.-.-.-.-.-.-.-.-.-.-.-.-.-.-.-.-.-.-.-.-.-.-.-.-.-.-.-.
%
\spbegin %.+.+.+.+.+.+.+.+.+.+.+.+.+.+.+.+.+.+.+.+.+.+.+.+.+.+.+.+.+.+.+.+.+.
\skeyword{USE\_SCN\_NODE}
\sprompt{(input node name)}
\sdefault{= *:}
\suser{*}
\sinline{Use the same SCN file}
\spend %.-.-.-.-.-.-.-.-.-.-.-.-.-.-.-.-.-.-.-.-.-.-.-.-.-.-.-.-.-.-.-.-.-.-.
%
\spbegin %.+.+.+.+.+.+.+.+.+.+.+.+.+.+.+.+.+.+.+.+.+.+.+.+.+.+.+.+.+.+.+.+.+.
\skeyword{USE\_SCN\_SETS}
\sprompt{(Set(s) of input uv-data Sectors:  g.o.f.c.s)}
\sdefault{= "":}
\suser{2.0.0.0.0}
\sinline{Sets in job 2}
\spend %.-.-.-.-.-.-.-.-.-.-.-.-.-.-.-.-.-.-.-.-.-.-.-.-.-.-.-.-.-.-.-.-.-.-.
%
\spbegin %.+.+.+.+.+.+.+.+.+.+.+.+.+.+.+.+.+.+.+.+.+.+.+.+.+.+.+.+.+.+.+.+.+.
\skeyword{HA\_RANGE}
\sprompt{(DEG) (HA range)}
\sdefault{= *:}
\suser{\scr}
\sinline{Use all scans in these sets} %
\svbegin \begin{verbatim}
   0123456789ABCD
 0 -+++++++++++++
 1  -++++++++++++
 2   -+++++++++++
 3    -++++++++++
 4     -+++++++++
 5      -++++++++
 6       -+++++++
 7        -++++++
 8         -+++++
 9          -++++
 A           -+++
 B            -++
 C             -+
 D              -
\end{verbatim}\svend
\spend %.-.-.-.-.-.-.-.-.-.-.-.-.-.-.-.-.-.-.-.-.-.-.-.-.-.-.-.-.-.-.-.-.-.-.
%
\spbegin %.+.+.+.+.+.+.+.+.+.+.+.+.+.+.+.+.+.+.+.+.+.+.+.+.+.+.+.+.+.+.+.+.+.
\skeyword{SELECT\_IFRS}
\sprompt{(Select/de-select ifrs)}
\sdefault{= "":}
\suser{\scr}
\sinline{Use all interferometers} %
\svbegin \begin{verbatim}
 Sector: 0.0.0.0
 Sector: 0.0.0.4
 A complex angle of 0.55-0.00I(0.01+0.01I) or -0.89 degrees
\end{verbatim}\svend
\spend %.-.-.-.-.-.-.-.-.-.-.-.-.-.-.-.-.-.-.-.-.-.-.-.-.-.-.-.-.-.-.-.-.-.-.

Warning: Remember that the PZD may partly be an {\em artifact} of the separate
phase/gain calibration of the X-dipoles and the Y-dipoles. This may lead to a
different PZD angle for the calibrator and the subsequent observation. The
difference may be several degrees, which is much greater than the accuracy that
is required for the measurement of very small percentages of circular
polarisation. However, the measured V may be more accurate if the observed
object is extended, which will often be the case. But the truth of the matter
is that, even though the WSRT may offer the best conditions for very accurate
polarisation measurements, the PZD problem is essentially unsolved.


%       JPH 940916      Make compilable
% nclean_descr.tex
%
%       JPH 9404..      Original
%       JPH 940729      Note on corrugations and SDI Clean
%                       Note on residual map for UV Clean (bug 67)
%       JPH 940816      \textref to keyword help files --> \keyref
%       JPH 941028      \keyref --> \textref
%
%
%
\chapter{The program NCLEAN}

\tableofcontents

\section{ Overview}
\label{.overview}

	NCLEAN is a relatively simple implementation of a number of Clean
algorithms.  According to its writer (WNB), more elaborate and versatile
implementations exist in AIPS and GIPSY.

	NCLEAN makes extensive use of the
\textref{\em source model}{nmodel_descr} facilities. A source model is a list
of source components.  The user has considerable latitude in manipulating such
models; for example, NCLEAN allows one to restore source components that were
inserted in the model by mechanisms other than Clean and are not on grid
points.  Since it is left to the user to decide what is "correct" and what not,
such facilities should be used only by experts and with caution.

\input{../fig/nclean_interface.cap}

	\Figref{.nclean.interface} shows an overview of the options available
in NCLEAN.  The program's basic Clean function is implemented in several forms
to be discussed below.  In addition, there is a separate option to restore a
map from a source model in a
\textref{\em MDL file}{nmodel_descr}, and an option to produce a histogram of
the intensities of a collection of map areas.

	The components found are stored in a source model in core. NCLEAN will
ask the user for an MDL file to store the list; if he specifies none, the
\textref{\em NMODEL HANDLE}{nmodel_descr} process will be activated to process
the list.


\section{ NCLEAN versus NMODEL FIND}
\label{.nclean.versus}

	The Clean algorithm is fundamentally limited by its basic paradigm,
{\em viz.} attempting to represent a continuous brightness contribution as a
superposition of point sources on map-grid points. This approach is inadequate
for small sources if dynamic ranges in excess of, say, 1000 are aimed for.

	On the other hand, Clean is the only mechanism available to construct
models for extended sources that cannot be adequately represented by a model
with at most a few parameters. Note, however, that Clean has difficulty
guessing the missing information for extended sources which results in the
notorious problem of \whichref{\em corrugations}{}. Modifications to Clean's
search algorithm have proved succesful in suppressing these corrugations
(whichref{"SDI Clean"}{}), but they are not available in NCLEAN.

	As an alternative, Newstar has the option of FINDing sources with the
program \textref{NMODEL}{nmodel_descr}.  This method is superior for sources of
small extent, since it does not confine their positions to be on grid points
and desribes their shapes in terms of an ellipse with three parameters (axes
and orientation).  In doing this, FIND can also take the effects of bandwidth
smearing and finite integration time into account.

	FIND is therefore the recommended way of building models of small
sources.  Cleaning must be used for sources that do not fit the shape
restrictions on FIND sources.  Obviously, Clean and FIND may have to be used in
combination, in such a way that all source components in the model take the
form that best fits their real appearance.


\section{ Cleaning methods} %
\subsection{ Beam Clean = H\"ogbom Clean}
\label{.beam.clean}

	H\"ogbom's original \whichref{\bf Clean}{} method is known in Newstar
as {\bf Beam Clean}.  It consists of repeatedly finding the highest point in
(part of) a map (the {\em search area}) and subtracting a source component at
that position from the entire map or some part of it (the {\em clean area}).
The subtracted component consists of the antenna pattern shifted and multiplied
by the peak intensity of the source and a {\em gain factor}.  This factor is
usually chosen somewhat smaller then 1 to account for the fact that part of the
intensity found may be due to sidelobes from other sources.

	The subtraction may be problematic in cases where the shifted antenna
pattern does not cover the entire clean area.  Moreover, the edges of both the
map and the antenna pattern are contaminated by aliasing effects.  For this
reason cleaning is best restricted to an area somewhat less than the central
half of the map in both dimensions.

	In the NEWSTAR version, the search area coincides with the clean area.
It is defined as the union of at most 32 rectangles specified by the user.
Since subtraction occurs only in the search area, sidelobes in the remainder of
the map are not removed.  It is therefore not advisable to successively clean
different areas; nor is it necessary, since NCLEAN allows you to specify all
relevant areas simultaneously.

	H\"ogbom Clean is controlled by the parameters

\begin{itemize}
\item   \textref{\bf LOOP\_GAIN}{nclean_private_keys.loop.gain}: the gain
factor;

\item   \textref{\bf COMPON\_LIMIT}{nclean_private_keys.compon.limit}: the
number of source components at which the process will stop;

\item   \textref{\bf CLEAN\_LIMIT}{nclean_private_keys.clean.limit}: the
absolute maximum intensity in the residual map at which the process will stop;

\item   \textref{\bf AREA}{nclean_private_keys.area} (prompt repeats until a
null answer is given): the map areas in which source components are looked for
and subtracted. %
\end{itemize}

\label{.prussian.hat}
	An additional parameter,
\textref{\bf PRUSSIAN\_HAT}{nclean_private_keys.prussian.hat}, allows one to
stick a delta-function peak on the centre of the antenna pattern, which in
certain cases gives better results (Cornwell ...).


\subsection{ UV Clean = Clark Clean }
\label{.uv.clean}

	{\bf UV Clean}, more hgenerally known as \whichref{\bf Clark Clean}{}
speeds up the cleaning of extended sources which must be represented by a large
number of closely spaced point-source components. Sources are found in the same
way as for H\"ogbom Clean.  But rather than subtracting them one by one in the
entire search area, Clark Clean does only a provisional subtraction in a small
{\em patch} around the source position.  Proper subtraction is done later by
jointly convolving the sources collected with the antenna pattern and
subtracting the result.

	The convolution is implemented through Fourier transformations and a
multiplication in the visibility plane, which is considerably faster than a
direct convolution in the map domain.  However, this method suffers from
aliasing errors similar to those for Beam Clean and is therefore subject to the
same limitations.

	In UV Clean {\em minor cycles}, in each of which a single source
component is located and provisionally subtracted, alternate with {\em major
cycles} in which the newly found sources are properly removed. The number of
minor cycles between successive major cycles is controlled by an algorithm that
estimates the buildup of deviations in the residual map due to the provisional
nature of component subtractions.  The user has a few parameters to control
this algorithm.

	UV Clean is controlled by the same parameters as H\"ogbom Clean. In
addition, the number of minor cycles to a major one is controlled by the
parameters:

%
\begin{itemize}

\label{.cycle.depth}
\item   \textref{\bf CYCLE\_DEPTH}{nclean_private_keys.cycle.depth}: The lower
limit to components to be found in minor cycles relative to the present
absolute residual-map maximum.

\label{.grating.factor}
\item \textref{\bf GRATING\_FACTOR}{nclean_private_keys.grating.factor}: A
multiplier to be applied to the estimate of the maximum error buildup in the
minor cycles.  The default of 1 is supposed to guarantee that grating responses
will not erroneously be mistaken for source components.  A lower value allows a
larger number of minor cycles per major cycle, but should be used only to the
extent that there is no danger of mistakes. The \textref{\bf COMPON}{.compon}
clean option can be used to explore this effect prior to starting serious work.

\end{itemize}

	In map-making, a taper rising toward the map edges may have been
applied to the final map to componsate for the effect of the gridding
convolution in the UV plane.  The parameter
\textref{\bf DECONVOLUTION}{nclean_private_keys.deconvolution} controls if the
effect of this must be taken into account in cleaning.


\subsection{ Data clean = Cotton-Schwab Clean}
\label{.data.clean}

	"Data" Clean is the somewhat unfortunate name given to the
"Cotton-Schwab" modification of \textref{Clark Clean}{.uv.clean}) in which the
model subtraction is done in the \textref{SCN file(s)}{scn_descr} from which
the map was originally made.  In this way aliasing effects are completely
avoided, albeit at the expense of again much more processing.

	Since a Newstar map does not include a detailed bookkeeping of the SCN
files and control parameters with which it was made, the user has to respecify
them as part of the process.  This is not as bad as it may sound, because no
harm will be done if the new map is made with different specifications provided
only that a corresponding antenna pattern is generated along with it.  The user
may actually take advantage of this, e.g.  by specifying a smaller map area if
he has found that that is all he needs.

	Since the antenna beam plays no role in source subtraction, the
\textref{\bf PRUSSIAN\_HAT}{.prussian.hat} parameter is not used in Data Clean.

	An important practical point to note is that {\em the residual map
overwrites the input map}. This is discussed further
\textref{below}{.residual.maps}.


\subsection{"Component" clean: A quick-and dirty Clean}
\label{.compon}

	Another Clean option is available in NCLEAN, in which only the
minor-cycle part of a Clark or "Data" Clean is performed.  Since the
minor-cycle source subtraction is of a sloppy nature, the residual map should
{\em not} be used as a starting point for further investigations. However, the
source model can be used, e.g.  as a starting point for
\textref{Selfcal}{ncalib.redun} or \textref{NMODEL UPDATE}{nmodel_descr}.

	Another possible use is to quickly explore the effect of the {\bf
CYCLE\_DEPTH} and {\bf GRATING\_FACTOR}
\textref{parameters}{.cycle.depth} before embarking on a lengthy Clark or Data
Clean.


\section{ Residual maps}
\label{.residual.maps}

	Residual maps have the same size as their parent map. For H\"ogbom and
Clark Clean, they are stored in the .WMP file from which the input map was
taken, under the index

	\verb/<group>.<field>.<channel>.<polarisation>.1.<sequence_number>/

where \verb/<group>.../  are the same as those of the input file, the type 1
indicates a map and \verb/<sequence_number>/ is the first new number available.
Thus, derivatives of a map will form a contiguous sequence in the order in
which they were created.

	For Data Clean the residual map is written in place of the original.
The idea behind this is that, unlike for the other Clean modes, one may
backtrack from a Data Clean operation without recourse to the input map: What
one does instead is to discard the newly found model components and then
regenerate the input map with the previous model.

	In whichever way one cleans, backtracking implies that one must keep
track of which residual maps are rejected. There is (as yet?) no mechanism
available to remove them from a .WMP file except for selectively
\textref{copying}{nmap_descr.copy} maps to a new .WMP file.



\section{Missing items to be added}
\label{.missing}

\begin{itemize}
\item   Litterature references
\item   Beam-patch calculation and other details of minor-cycle clean
\item   RESTORE details
\end{itemize}




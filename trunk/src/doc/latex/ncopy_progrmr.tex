\chapter{NCOPY programmmer's documentation}

{\em Contributed by JPH, 940208. \\
Updates:
	JPH 941031      Clarify handling of APPLY/DEAPPLY masks before and
after changes of October 1994.
\\      JPH 941121      prefix underscores with slashes
}
\newcommand{\bi}{ \begin{itemize} }
\newcommand{\ei}{ \end{itemize} }

\tableofcontents


\section{ Overview}

	NCOPY Phase 2 is still a relatively straigthforward program. The
structure follows standard Newstar practice, with a main dispatching routine
NCOPY, initialisation and user-parameter input routines NCOINI and NCODAT, and
routines NCO$<$xxx$>$ that do the actual work. Some parameter input may be done
by the latter.

	There are currently two main options:
\bi
\item   COPY: To selectively copy (parts of) sectors.

\item   OVERVIEW: To display an overview of all sectors, one sector per line.
\ei

\section{ NCODAT }

	NCODAT is basically a clone of NSCDAT and follows the same kind of
logic. It is just as dull as all the other $<$xxx$>$DAT routines.

	Note however, that it calls WNDDA0 rather than WNDDAB as do the other
routines. As a consequence, the APPLY and DE\_APPLY masks (COMMON variables CAP
and CDAP) were left at their initial values of 0.

	This was changed in the release of October 1994. NCOCPY now prompts for
these masks with defaults NONE/ASK. These defaults reproduce the old behaviour,
but the user now has the option to "freeze" certain corrections "in" in the
visibilities. Since WNDDA0 does not allow dynamic setting of defaults, WNDPAP
is called for this purpose. (This option was requested by A.G. de Bruijn as a
tool to handle non-isoplanatic errors, such as may occur with a strong source
on the flank of the primary beam being affected by pointing errors to which the
field centre is essentially insensitive.)


\section{ NCOOVV: Sectors overview }

	NCOOVV accesses all sectors specified by the user in the standard way
through repeated calls to NSCSTG. For each sector it formats a line of salient
data.

	Special care had to be taken to make these lines fit on a standard
80-character screen: Since WNCTXT prefixes a blank, and terminals intend to
insert a line feed when a line is exactly 80 characters long, the actually
available line length is only 78 characters. Terminal width is set to this
value.

	Additional information may be displayed in the log, for which the width
is set to 131. Currently, pointers are displayed for diagnostic use.

	To make the display of hierarchical sector indices legible, a
3-character fixed-format length is used for each index, and leading indices
that are the same as in the nprevious line are suppressed.


\section{ NCOCPY: Copy sectors }

.       For a number of variables and data structures, NCOCPY has an input and
an output copy, which are distinguished by the prefixes I and O: IPLN $vs.$
OPLN, etc. This is in anticipation of their I and O values being different, as
some already may be, line ISCN and OSCN. Pairs that are currently identical are
linked by EQUIVALENCEs, $e.g.$ INIFR and ONIFR. "O" variables that are not
equivalenced must be explicitly assigned.

.       The basic logic of NCOPCPY is straightforward. NSCSTG is repeatedly
called to read sectors. Each sector's HA range is checked against the user's
range and first and last scan numbers to be copied, SCNFST and SCNLST, are
established. The condition that the output need no more polarisations than
present in the input is also checked.

.       ISTH is then copied to OSTH and written tentatively, with as many
changes as possible already made; its address is saved for a later rewrite.
Next, the IFR table is copied and the file address of the scans block
established.

.       The WSRT headers are copied in an internal subroutine at label 2020.
This routine consists of calls to:
\bi
\item   NSCSCR: read SCH; read raw uncorrected scan visibilities into
4-polarisation c/s and weight arrays, using STH and SCH dimension parameters
such as PLN and NIFR.

\item   NSCSDW: write scan from these arrays, again using dimension parameters.
\ei
.       Modifications to the scan are made in between.

.       The model lists and visibilities are copied in an internal subroutine
at label 2010.

.       Finally, the newly made sector is linked into the index structure. The
code is patterned after that of NSCREG.


\subsection{ Output sector numbering }

	Output sector indices are copied from the input except for the first
one, the group index. This index is automatically assigned the first available
number for each new group being processed, that is: For every new COPY request
and for every new input group within a COPY request. This method of assignment
is the simplest way to avoid collisions in output indices; alternatives should
only be considered if users ask for them.


\subsection{ STH modifications }


	NCOCPY is responsible for the following OSTH fields that differ from
their ISTH peers:
\bi
\item   HAB, SCN: reflect the user's HA selection.

\item   PLN: the number of polarisations in the output.

\item   NIFR: number of output interferometers (currently identical to input).

\item   SCNL: scan length (depends on NIFR, PLN).

\item   REDNS, ALGNS, OTHNS: zero the Y components if only XX data present.

\item   pointers to other data structures.
\ei


\subsection{ SCH modifications }

	NCOCPY is responsible for the following OSCH fields that differ from
their ISCH peers:
\bi
\item   MAX: maximum of C/S values REDNS, ALGNS, OTHNS.

\item   OTHNS: zero the Y components if only XX data present.
\ei

\subsection{ NCOCPB: Make unique copy of a data structure }

	Several input data structures may be linked by more than one pointer,
and each link might give rise to a separate copy. For copying such structures,
NCOCPB must be used. It maintains a checklist of input file addresses of blocks
it has already copied, with the corresponding output file addresses. The
calling program is responsible for allocating a buffer for this list and
initialising it through a special call to NCOCPB. Details of this and other
variant calls are to be found in NCOCPB's header.

	Since there can(?) be no duplications of data structures between
groups, the checklist is reinitialised for every new input group.

	The blocks that may have multiple references are:
\bi
\item   FDW for all sectors in a mosaic: 1
\item   OHW for all sectors in the same field: 128 fields
\item   SHW for all sectors at the same frequency: 8 frequencies.
\item   MDH for all sectors in the same field: 128 fields
\ei


\section{ Tests }

.       The following tests were made to check NCOPY:
\bi
\item   Copy sectors from the same input SCN file with various selection
parameters. Check outputs with OVERVIEW and compare samples of headers and data
with NCOPY SHOW.

\item   Copy one 4-polarisation input sector to three output sectors, selecting
XYX, XY and X polarisations. Make 4 XX and 3 YY maps from the input amd output
sectors and compare the map statistics through
.
	{\sf sdiff -s -w 80 $<$log1$>$ $<$log2$>\ |$ grep '-$<>$'}
\ei

\section{ Modification history }

\subsection{Copying flags (9401..-940207) }

	Upon request of Ger de Bruijn, all input flags are also copied to the
output. To this end, the calls to NSCSCR and NSCSCW have been changed to calls
to NSCSCF and NSCSFW, respectively. These have an extra parameter in which the
flag settings from the input are transferred to the output.

	The new entry points exist in the new version of NSCSCR and NSCSCW
which are still being tested in JPH's shadow system.

	If desired, it is very simple to add code to suppress the copying of
flags.

	Test:
	Use a 720-scan single-polarisation file. NFLAG, flag\_option DETERM,
ops\_determ PBAS, pbas\_limits=0,200. This results in 18% SHAD flags being set
in a 270-scan single-polarisation file. flag\_option INSPECT, ops\_inspect IFR
gives an overview per interferometer.
	Copy the file with NCOPY and test the copy with NFLAG, flag\_option
INSPECT, ops\_inspect COUNT, select\_flag ALL, sub\_cube NO. This gives the
same result.


\section{ Bug history }

\subsection{ Checklist overflow (de Bruijn, 940207) }

Corrections:

\bi
\item   Do not use NCOCPB for STH addresses. Multiple references to STH can
only occur due to a NSCAN REGROUP operation which is very rarely used. Multiple
copies of an STH can not be prevented between COPY operations. The user is
already responsible for avoiding those and now has the added responsibility of
avoiding REGROUP duplications.

\item   Reinitialise checklist for each new input group.

\item   Souble checklist size to handle the largest predicted mosaic observation
\ei

Test:
	Copy a simple observation A to B, then B to C, and verify that
overviews of B and C are identical.










%
% @(#) ngids_descr.tex  v1.2 04/08/93 JEN
%
%       JPH 940727      Add sectioning commands. Add "Killing" section
%
%
\chapter{A few remarks on the program NGIDS {\it (Very incomplete)}}
\tableofcontents

\section{ How to run NGIDS}
\subsection{ Starting up}

	Maps created by NMAP can be loaded into GIDS using the program NGIDS.
It creates a window for GIDS if it does not exist yet. {\bf NOTE that GIDS only
works on a COLOUR display!!!}

	If NGIDS fails to start, you should check for the existence of the file
{\em .gids\_sockets} in your login directory. This file should NOT exist; so
delete it and retry if it does exist.


\subsection{ Exiting}

	NGIDS can be stopped in the usual way by the answer \#. This does not
remove the GIDS window, so you can still play with the images. Stopping GIDS
should be done via its menu option ETCETERA, QUIT, YES. Note that NGIDS should
be stopped before GIDS, otherwise you'll get a broken pipe error and maybe a
coredump.


\subsection{ Killing GIDS or NGIDS when it runs out of control}

	An incorrect termination of either GIDS or NGIDS may result in one of
the programs devouring CPU time in vainly polling for the other one. When you
suspect this, use {\bf ps} to find the process id of the culprit and then kill
it with {\bf kill $<$process-id$>$}.


\section{ Image planes}

	NGIDS records images which can be played back using the ETCETERA,
RECORDING, LOOP menu option in GIDS. NGIDS starts recording in "image plane" 1.
The maximum number of planes, the number of used planes and the currently
active plane are shown in a little pane at the top left corner of the GIDS
window. You should restart NGIDS if you want to start recording at 1 again. The
memory of the X-terminal is used for the "image planes". The NFRA HP
X-terminals have enough memory to record up to 28 512x512 maps.

	The last image loaded is also stored in image plane 0. This is a
special plane since it contains the entire image, even if it is not completely
visible. You can zoom and scroll it using the GIDS menu and/or the mouse. For
this the buttons on the mouse have the functions: \\ -    left   : zoom in \\ -
   middle : scroll \\ -    right  : zoom out \\ When clicking one of these
buttons the pixel the cursor is pointing to will be put into the middle of the
display.


\subsection{ Image sizes}

	By default GIDS is started with a window of 512x512. You can resize it
in the standard way using the mouse. When shrinking you should keep the menu
visible. Resizing affects the maximum number of images that can be recorded. It
also clears all image planes.

	NB: If the screen window is too small for the image, the program will
automatically "zoom out", i.e. to make the image smaller by taking one out of
every 2,3,4,etc pixels (in two dimensions). Note that this is not the same as
averaging these pixels. When "zooming out" however, all pixels are displayed.


\section{ Acknowledgement}

	We thank the GIPSY group in Groningen and in particular Dr Kor Begeman
for the GIDS software and the advice and help they gave us. In the future they
will write a GIDS user manual, but until then you have to find your own way in
GIDS.

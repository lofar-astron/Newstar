%
% @(#) nmap_descr.tex  v1.2 04/08/93 JEN
%
\chapter{The Program NMAP}
\tableofcontents

\section{ Introduction}
\label{.intro}

	The \NEWSTAR program NMAP makes maps (images) from the uv-data in a
SCN-file, and stores them in a WMP-file. The input data may be combined in many
different ways, to produce maps that range from simple to quite exotic.  As a
by-product, antenna patterns may be calculated (and stored in the same .WMP
file), for use in \textref{NCLEAN}{nclean_descr}.

	The program NMAP also allows the user to manipulate maps in a .WMP file,
and to convert them to other formats.

	A .WMP file may contain various kinds of maps that are {\em related in
some way}, i.e. different frequency channels, polarisations, pointing centres
(mosaicking), antenna patterns, CLEAN residuals, and even gridded uv-data. Each
{\em image} in a .WMP file consists of a 2-dimensional array of pixel values,
withj a header containing descriptive information. A 'hypercube' of images
in a .WMP file can have maps of different sizes. For more information about the
structure of the .WMP file, see the \textref{.WMP file}{wmp_descr} description.

%==============================================================================
\section{Overview of NMAP options}
\label{.options}

The program NMAP offers the following main options:

\begin{itemize}
\item {\bf  MAKE:}      Make map(s) and/or antenna patterns
\item {\bf  SHOW:}      Show/edit map (header) data.
			See section on WMP File Description.
\item {\bf  FIDDLE:}    Perform all kinds of operations on maps.
\item {\bf  W16FITS:}   Write FITS tape/disk with 16 bits data.
\item {\bf  W32FITS:}   Write FITS tape/disk with 32 bits data.
\item {\bf  FROM\_OLD:} Convert from old (R-series) format to WMP format.
\item {\bf  TO\_OLD:}   Convert from WMP format to old (R-series) format.
\item {\bf  CVX:}       Convert a WMP-file from other machine's format
			to local machine's.
\item {\bf  NVS:}       Convert a WMP file to newest version.
			This option should be run if indicated by the program).
\item {\bf  QUIT:}      Quit the program NMAP.
\end{itemize}



%==============================================================================
\section{Option MAKE: Making maps}
\label{.make}

	An overview of the user interface for the MAKE option is shown in
\figref{.nmap.make} and its companion \figref{.nmap.make.q}.

\input{../fig/nmap_make.cap}


%-----------------------------------------------------------------------------
\subsection{Types of output images}
\label{.make.output}

	The program NMAP produces (multiple) images of the following types:

\begin{itemize}

\item {\bf  MAP:} Normal map of the uv-data or the uv-model (the
uv-representation of the Selfcal model) from the SCN-file.
Various linear combinations of the four measured polarisations can be specified
with the keyword {\bf MAP\_POLAR}, to produce:

  \begin{itemize}
  \item {\bf  XX, YY, XY or YX-maps:} Use XX, YY, XY or YX data only
  \item {\bf  I-map:}   (XX+YY)/2
  \item {\bf  Q-map:}   (-XX+YY)/2
  \item {\bf  U-map:}   (-XY+YX)/2
  \item {\bf  V-map:}   (XY+YX)*i/2
  \item {\bf  L-map:}   XX or YY or (XX+YY)/2 if both present
  \item {\bf  *I-map:}  any of the above, but multiplied with $\sqrt{-1}$
  \end{itemize}

NB: Note that parallel dipoles (++) are assumed here.
Observations with `crossed' dipoles (+x) require linear combinations of
all four polarisations. This can be done in the map plane (see NMAP
option {\bf FIDDLE\_OPTION}).

\item {\bf  AP:}        Antenna pattern (Replace uv-data by 1's)
\item {\bf  COS:}       Assume input sines to be zero
\item {\bf  SIN:}       Assume input cosines to be zero
\item {\bf  AMPL:}      Assume input phases to be zero
\item {\bf  PHASE:}     Assume input amplitudes to be one
\end{itemize}

It is also possible to store {\bf uv-data} in a WMP file, for display purposes.
The uv-data is `gridded' (convolved onto a rectangular grid).

\begin{itemize}
\item {\bf  COVER:}     Gridded uv-coverage (data replaced by 1's)
\item {\bf  REAL:}      Real part of the gridded uv-data
\item {\bf  IMAG:}      Imaginary part of the gridded uv-data
\item {\bf  AMPL:}      Amplitude of the gridded uv-data
\item {\bf  PHASE:}     Phases of the gridded uv-data
\end{itemize}


%----------------------------------------------------------------------------
\subsection{Operations on the input data}
\label{.make.input}

\input{../fig/nmap_make_q.cap}

\begin{itemize}
\item {\bf Data selection}: The uv-data that go into a map may be
selected in various ways:
  \begin{itemize}
  \item One or more SCN-files.
  \item Sets within a SCN-file (e.g. frequ channels, pointing centres).
  \item HA-Scans within each Set (HA-range).
  \item Polarisations (XX,YY,XY,YX).
  \item Individual iterferometers
  \item An area in the uv-plane
  \item Clip-level
  \end{itemize}

\item {\bf Data correction}:
Various correction factors are stored in the Set and Scan headers of the
SCN-file. They may be applied to the uv-data at the moment that they are
read from disk for processing. Application of correction is controlled
by the NGEN keywords {\bf APPLY} and {\bf DE\_APPLY}
(see the section `Common Features of \NEWSTAR Programs' in this Cookbook).

\item {\bf Data conversion}:
The uv-data may be converted in various ways:
  \begin{itemize}
  \item Subtraction of a source model.
  \item Combination of different polarisations (XX,YY,XY,YX).
  \item Conversion to amplitudes or phases.
  \end{itemize}

\item {\bf Data weighting:}
  \begin{itemize}
  \item {  NATURAL:}  Take each individual measured point separately,
			without weighting for the UV track covered by it.
  \item {  STANDARD:} Weight each observed point with the track length covered
			on the UV plane,
			and average redundant baselines on a per set basis.
  {\it (default)}
  \item {  FULL:}     Weight each point according to the actual UV point
			density. In this case care is taken of all local UV
			plane density enhancements,
			but it necessitates an extra pass through the data.
  \end{itemize}

\item {\bf Data tapering:}
  \begin{itemize}
  \item {  GAUSS:}      $\exp(-baseline^{2})$
  {\it (default)}
  \item {  LINEAR:}     max(0,1-baseline/taper\_value)
  \item {  NATURAL:}    no taper
  \item {  OVERR:}      $1/baseline$
  \item {  RGAUSS:}     $\exp(-baseline^{2})/baseline$
  \end{itemize}

\item {\bf Data convolution:} (onto a rectangular grid in the uv-plane)
  \begin{itemize}
  \item {  GAUSS:}      Gaussian type with $4\times 4$ grid points
  \item {  P4ROL:}      Prolate spheroidal function with $4\times 4$ grid points
  \item {  P6ROL:}      Prolate spheroidal function with $6\times 6$ grid points
  \item {  EXPSINC:}    $sinc\times\exp$ on $6\times 6$ grid points
  {\it (default)}
  \item {  BOX:}        A square box
  \end{itemize}
\end{itemize}

At this point, the data are either stored in the WMP file as gridded uv-data
(usually for display purposes), or Fouries transformed into an image.

%==============================================================================

\section{Option MAKE: example}
\label{.make.example}

The following is an example of making a `normal' map (and its antenna pattern),
using the program defaults. This is usually sufficient to get a satisfactory
result. Experienced users may experiment with some of the more advanced
options.

**** Put new script here ****


%------------------------------------------------------------------------------

\subsection{QMAPS: Hidden map options}
\label{.make.qmaps}

The more advanced map-making options are hidden behind the NMAP keyword {\bf
QMAPS}. If skipped, the (context-sensitive) default values will give a
satisfactory result in most cases.
Their values will be printed in the NMAP log-file.
For more information on each of these
keywords, see the on-line Help text (type `?'), which is also printed in the
`Summary of NMAP keywords' in this Cookbook.

**** Put new script here ****


%------------------------------------------------------------------------------
\subsection{QDATAS: Hidden map options}
\label{.make.qdatas}

The more advanced data-selection options are hidden behind the NMAP keyword
{\bf QDATA}. If skipped, the (context-sensitive) default values will give a
satisfactory result in most cases.
Their values will be printed in the NMAP log-file.
For more information on each of these
keywords, see the on-line Help text (type `?'), which is also printed in the
`Summary of NMAP keywords' in this Cookbook.

\sprompt{QDATAS}
\sprompt{(More data handling details?)}
\sdefault{ = NO:}
\suser{y}

(To be added later).

%==============================================================================

\section{Option FIDDLE: Operations on maps}
\label{.fiddle}

\input{../fig/nmap_handle.cap}

The NMAP keyword {\bf FIDDLE\_OPTION}
offers the user a wide range of possibilities to
perform oparations on maps in WMP files:

All relevant data are copied from the first (or only) input map.
If 2 maps are required (ADD,AVER,POL,ANGLE) all pairs of SETS\_1 and
SETS\_2 will produce an output map.
SUM will average all SETS\_1 maps.
MOSCOM will produce a single output map from all specified input maps.
The others will produce an output for each SETS\_1.
BEAM, DEBEAM and FACTOR will overwrite the input maps.
F is a specified multiplication factor.

\begin{itemize}
\item {\bf  ADD:}  $F1\times map1 + F2\times map2$
\item {\bf  AVER:}  $(F1\times map1 + F2\times map2)/(abs(F1)+abs(F2))$
\item {\bf  SUM:}  Various averages of maps.
In all cases the summation produces
a weighted average map over all SETS\_1, the weight depending on the type:
  \begin{itemize}
  \item {\bf  SUM:}  weight(i)= 1
  \item {\bf  NSUM:}  weight(i)= normalisation factor of map
  \item {\bf  BSUM:}  weight(i)= bandwidth
  \item {\bf  BNSUM:}  weight(i)= bandwidth $\times$ normalisation factor
  \item {\bf  FSUM:}  weight(i)= factor given by the user.
		Up to 8 factors can be given,
		which will be used in a cyclic fashion if more needed
  \item {\bf  NSSUM:}  weight(i)= $1/(map noise^{2})$
  \item {\bf  QUIT:}  finished
  \end{itemize}
\item {\bf  POL:}  $\sqrt{map1^{2} + map2^{2}}$,
		unless $< F1$, then 0
\item {\bf  ANGLE:}  $0.5\times \arctan{(map1/map2)}$ (radians),
		unless $POL < F1$, then 0
\item {\bf  EXTRACT:}  extract an area of maps
\label{.copy}
\item {\bf  COPY:}  copy maps
\item {\bf  BEAM:}  correct maps for primary beam
\item {\bf  DEBEAM:}  de-correct maps for primary beam
\item {\bf  FACTOR:}  $F1\times map$
\item {\bf  MOSCOM:}  combine all specified maps
		(referenced to same mosaic position) into one output map.
		The noise of the individual maps may be used as weight.
\item {\bf  QUIT:}  quit the fiddle option.
\end{itemize}



%==============================================================================

\section{Option FITS: Conversion to FITS format}
\label{.fits}

There are two separate NMAP options ({\bf W16FITS} and {\bf W32FITS}.


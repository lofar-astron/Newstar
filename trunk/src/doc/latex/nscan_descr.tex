%
% @(#) nscan_descr.tex  v1.2 04/08/93 JEN
%
% History
%       JPH 9401003     Remove \newpage, shorten %===== markers t0 80 chars
%
%
\chapter{The Program NSCAN} 
\tableofcontents 

%===============================================================================

\section{Overview of NCSAN options} 
\label{nscan.descr.options} 

The program NSCAN allows the user to interact with the uv-data file (SCN-file). 
The structure of this file is explained in detail in a separate section of this
cookbook. The program NSCAN offers the following main options: 

\begin{itemize} 
\item {\bf LOAD:} 
Load WSRT uv-data (in WSRT circle format) from tape, disk, DAT or optical disk
into a SCN-file. 
The input may be from multiple tapes and/or labels. 
The user may select data, and change the integration time. 

\item {\bf DUMP:} 
Dump WSRT data (in WSRT circle format) from tape or DAT or optical disk to a
disk file (in WSRT circle format). 

\item {\bf FROM\_OLD:} 
Convert an old (R-series format) SCN-file into a \NEWSTAR SCN-file. 

\item {\bf TO\_OLD:} 
Convert a \NEWSTAR SCN-file into an old (R-series format) SCN-file. 

\item {\bf SHOW:} 
Show/edit the contents of a SCN-file: layout, header information 
(incl corrections), uv-data and uv-model. 
This is demonstrated in the section `Description of the SCN-file' in this
Cookbook. 

\item {\bf DELETE:} 
Delete (or un-delete) uv-data in a SCN-file, according to certain selection
criteria. Actually, the data is only disabled (flagged) reversibly by making
the attached weight-factor negative. 

\item {\bf COPY:} (not yet available) 
Copy selected Sets from a SCN file to a new (secondary) SCN-file. The uv-data
may be physically modified (e.g. corrections, model subtraction, change of
integration time) in this process. 
This option will probably be implemented in a separate program NCOPY. 

\item {\bf REGROUP:} 
Select and reorganise the data in a SCN-file. 
Make a new group directory entry (job tree) with specified Sets in it. 

\item {\bf UVFITS:} 
Convert a SCN-file into a UVFITS file (tape/disk) for further image analysis in
AIPS. 
It is recommended to do all WSRT uv-data processing first in \NEWSTAR, since
AIPS uv-data processing is rather VLA-oriented, and does not do justice to WSRT
data. 

\item {\bf PFITS:} 
Print a summary of a UVFITS (AIPS) tape/disk file, showing all keywords and a
limited set of data. 

\item {\bf CVX:} 
Convert a SCN file from other machine's format to local machine's. 

\item {\bf NVS:} 
Convert a SCN file to newest version. 
This should be run if SCN file made before the dates: 
 \\          910417: add MJD to set header. 
 \\          900907: add precession rotation angle. 
 \\          900220: add polarisation corrections. 
 \\          920828: recalculate MJD for observations aborted at Wbork. 

\item {\bf WERR:} 
Correct mosaic tape errors. This only concerns mosaic data taken in 1991. 

\item {\bf QUIT:} 
Exit the program NSCAN 
\end{itemize} 


%===============================================================================

\section{Option LOAD} 
\label{nscan.descr.load} 

{\bf An example:} 
Let us assume we have a mosaicking observations at 4 settings of ABCD of 60
fields, 64 line channels. There will thus be 12 spokes per 12 hour per field. 
These data are on 4 tapes: 

\vspace{-0.8cm}                                 % reduce vertical gap
\begin{tabbing} 
+++++\=++++++++\=++++++++\=+++++\=+++++\=+++++\=+++++\=+++++\=+++++\= \kill
%tabs
 \\ \> Tape 1: \> label 1: \> 6 hours, spacing 9A is 36 m 
 \\ \>         \> label 2: \> 6 hours, 36 m 
 \\ \> Tape 2: \> label 1: \> 12 hours for fields 30-59, 72 m 
 \\ \>        \> label 2: \> 12 hours for fields 0-29, 72 m 
 \\ \> Tape 3: \> label 1: \> 12 hours all fields for 48 m 
 \\ \> Tape 4: \> label 1: \> 12 hours all fields for 90 m 
\end{tabbing} 
\vspace{-0.4cm}                                 % reduce vertical gap

The Set numbers will be unknown, but the indices generated are (if they are
read in in order of tape and label): 

\vspace{-0.8cm}                                 % reduce vertical gap
\begin{tabbing} 
+++++\=++++++++\=++++++++\=+++++\=+++++\=+++++\=+++++\=+++++\=+++++\= \kill
%tabs
 \\ \> Tape 1: \> {\tt 0.0.0-59.0-64.0-5, 0.1.0-59.0-64.0-5 } 
	\>\>\>\>\>Note continuum channel 0 
 \\ \> Tape 2: \> {\tt 1.0.30-59.0-64.0-11, 1.1.0-29.0-64.0-11 } 
 \\ \> Tape 3: \> {\tt 2.0.0-59.0-64.0-11 } 
 \\ \> Tape 4: \> {\tt 3.0.0-59.0-64.0-11 } 
\end{tabbing} 
\vspace{-0.4cm}                                 % reduce vertical gap

If a map is wanted of field 31 using spacings 36 m and 72 m averaging all odd
channels between 17 and 25, the data could be specified as: 

\begin{verbatim} 
    0..31.17-25:2,1..31.17-25:2
or  0-1.*.31.17-25:2 
\end{verbatim} 


%===============================================================================
\section{Option DELETE} 
\label{nscan.descr.delete} 

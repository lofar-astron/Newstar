% History:
%       JPH 940914      Replace & by / in \verb (latex2html mishandles &)




\chapter{ Dwingeloo Plate Measuring Machine} 


A \NEWSTAR Guide to Determining Positions Using the Dwingeloo Plate Measuring
Machine. 

\indent{\it     Original by James Albinson, July 1983 \\ 
\indent         Version 3, updated by Richard Strom, October 1992} 


{\it For Problems, questions, complaints, suggestions on use and operation: See
Richard Strom, alternatively Ger der Bruyn. \\ In case of mechanical/electrical
failures: See Jean Casse}. 



\section{Introduction} 
\label{.intro} 

This document is based on previous versions by Seth Shostak and Bob Hanisch,
and gives a simple procedure for measuring the positions of objects from the
PSS prints or KPNO plates.  The general procedure is as follows: 
\\(1) identify standard stars from an overlay of the region of interest, 
\\(2) measure the positions of a number of standard stars and the desired
objects, 
\\(3) solve for the plate constants and the positions of the desired objects. A
guide to these procedures is given below. 


\section{Logging in} 
\label{.login} 

Turn on the terminal next to the measuring machine and the yellow Interloop
interface to which it should be connected. (If the interface is off, there will
be no response from the terminal. If the terminal is disconnected from the
interface, you will have to ask for help to connect it up.) If you get the
server prompt (`Server $>$'), connect to the RZMVX4 computer (`c' should be
enough) and log in under `measure', password: \verb/meas$$92/ 

You now need to type the following command: \verb/assign user1: qd2:/ 

You're now ready to run the OVERLAY program. 


\section{Running the OVERLAY program} 
\label{.run} 

Type: \verb/exe overlay/ 

The program will ask for a right ascension and declination; these refer to the
field center you're interested in.  For example, if you want to find star
positions around a galaxy, type the position of the galaxy. 

An example: 
\\ RA\_MAPCENTRE (RA of map\_centre in h,m,s): 4,2,35.000 
\\ DEC\_MAPCENTRE (DEC of map\_centre in d,m,s): 69,40,42.00 
\\ For negative declinations simply prefix the declination degrees with a minus
sign. 

OVERLAY will produce some output on the lineprinter and an overlay plot on the
QMS printer/plotter. It may be desirable to make the plot on a transparent
sheet; in that case special Xerox transparent overhead sheets can be loaded in
the QMS printer (check with other users first, and only load as many sheets as
you need). Take the plot and identify your standard stars. If the plot was made
on paper, you can circle the stellar positions on a plastic sheet using a
marking pen that can be obtained from the secretaries. It is also helpful to
label the stars with the same numbers that appear on the OVERLAY plot, as you
will have to enter these numbers when measuring the positions. On the other
hand, you may not want to write too much stuff on the plastic sheet as this may
obliterate objects that you want to measure. Since it is not possible to
measure any more than 30 standard stars, be sure that you select standard stars
with numbers between 1 and 30. (OVERLAY numbers the stars in order of
increasing distance from the field center.) 

Special note for very northerly fields (dec $> 70^\circ$): For very northerly
fields it is more convenient to make the overlays with a Schmidt projection
rather than a linear projection (which is the default).  To do this type in the
command \verb/@schmidt/. This will produce plots with the Schmidt projection
for all subsequent runs of OVERLAY.  To go back to a linear projection, simply
type the command \verb/@linear/. 

Special note for measuring several fields: You should complete your
measurements for each field before proceeding to the next field. The reason for
this is that the program which solves for the plate constants and object
positions uses the standard star positions from the most recent run of OVERLAY. 



\section{Measuring positions} 
\label{.pos} 

Preliminaries. Turn on the plate illumination light. The switch is on top of
the green box behind the measuring machine. Turn on the Television camera and
monitor screen. The switch for this is the leftmost of the three rotary
switches on the front of the monitor. Make sure that the lowest of the 7
channel knobs on the right-hand edge of the monitor is selected. After a few
seconds the monitor should show an image. The brightness and contrast knobs are
the other two rotary knobs on the front of the monitor. Adjust to suit. The
focus knob is below and to the right of the eyepiece, facing the front of the
machine. Adjust as necessary. 

Place the ``sandwich" of PSS print and plastic overlay between the glass plates
of the measuring machine.  The orientation is not important.  Adjust the
measuring machine so that the axes are set at the center of their travel
(x-axis at 10.0, y-axis at 9.0). Then release the gray clamps, center the
microscope over your field, and retighten the clamps. 

Turn on the interface (white rocker switch on right) and make sure the mode
switch is set to `TERMINAL'.  Turn on the two Sony Magnescale digital readouts
and the illuminating lamp, if you haven't already (switch is on the little
green box). Adjust the lamp for good illumination, and use the knurled knob to
focus the image through the microscope (if no TV).  Push the `START/READ'
switch on the interface. 

Start up the measuring machine program by typing: \verb/exe measure/ 

The program asks you if you want a new file. To start measuring a new field,
answer this question with `y'. To continue measuring on a field still in
position on the measuring machine, answer with `n'. 

Set the mode switch on the interface chassis to `TRANSMIT'. Measure your
standard stars (in any order) as follows: 
\begin{enumerate} 
\item Center the standard star in the crosshairs. 
\item Set the thumbswitches on the interface to the number of the standard star
from the OVERLAY plot (i.e., a number between 01 and 30). 
\item Push the `START/READ' switch.  This sends the x-y coordinates of the star
to the computer, and you should also see them dispayed on the terminal. 
\end{enumerate} 

Repeat steps 1-3 for each standard star you want to measure (10 to 15 stars are
usually sufficient).  Since there is no backlash in the digital readout, it is
not necessary to repeat your measurements or to approach each star from the
same direction. 

If you have a large number of radio sources to identify and would like to have
approximate x-y positions around which to look for optical identifications,
skip from here to \textref{section}{.diff}. 

Measure the ``unknown'' objects (stars, galaxies, or whatever) in the same
manner, but set the thumbswitches to numbers between 31 and 98, incrementing
the count by one for each object you measure. You may, of course, measure the
objects in any order you find convenient. The x-y positions will continue to be
displayed on the terminal, but the object number will be the thumbswitch
setting minus 30. 

When you have finished all the positions you want, set the thumbswitches to 99
and push the `START/READ' switch. This terminates the program. {\bf Be sure to
set the mode switch back to `TERMINAL'.} 


\section{Solving for plate constants and positions} 
\label{.solve} 

Start up the program by typing: \verb/exe solve/ 

The program asks you for some information about the plate: 
\\ - PLATE\_TYPE: Type a carriage return if you used a PSS print, or the
appropriate code for a KPNO plate. 
\\ - FIELDNAME: Type a name (up to 8 characters) of this field. 
\\ - TRANSFORM\_CODE: Type `1' (x-y $\rightarrow$ alpha,delta) 
\\ - PLATE\_CENTER (hms,dms): Type the right ascension and declination of the
field center in the format shown. These coordinates can be found in the little
box in the upper left corner of each PSS print; put a comma between hms \/ dms. 

The program now goes on to solve for the plate constants. You have the
opportunity to delete certain stars from the solution of plate constants if you
have reason to believe that their measured positions are in error.  Assuming
that they are all okay, however, just type `return' in response to this
question. 

- CORRECT\_STAND? Type either 'Y' or `return'. 

The program now solves for the coordinates of the unknown objects. When the
program finishes, you have the option to send the results to be printed on the
lineprinter: 

- OUTPUT? The default is Y (yes), so type a `return'.  Finally, you may have
the positions you've measured added to the standard star positions as secondary
standards: 

- SECONDARY\_STAND? Typically the response is `no' (the default is yes).  You
may now pick up your output downstairs from the lineprinter. 

If your plate solution looks dubious, i.e., one or more stars have large
errors, you can rerun SOLVE and delete these stars from the solution.  Run
SOLVE again, and answer the questions concerning the type of plate and field
name again.  Hereafter you want to use the same data as before.  SOLVE
remembers all of the information from the previous run, and uses this
information if you simply answer all of the questions with `return'.  So,
answer the questions with `return's until you are asked: 

- EXCLUDE\_STAND: Type in the numbers in the specified format, and then
continue as before to produce a new plate solution. 


\section{A different approach to making source identifications} 
\label{.diff} 


If you have a large number of radio sources you want to try to identify, you
could spend an eternity measuring candidate objects unless you know where on
the plate to look.  It is possible to use SOLVE to get x-y positions for given
radio source positions (in right ascension and declination epoch 1950), and
these x-y positions can then be used as guides in measuring candidate objects. 

To make use of this option, you must measure some standard star positions in
the field of interest as \textref{previously described}{.pos}. Rather than
measuring any unknown objects at this time, however, terminate MEASURE (set
thumbswitches to 99 and push the `START/READ' switch) and run SOLVE (see 
\textref{section}{.solve}). You will need to run SOLVE twice, first to check
that your standard stars have been measured accurately (check the rms fit and
the individual errors; if one or two stars are bad delete them from the fit the
next time around), and second to convert given alphas and deltas (i.e. your
radio source positions) to x-y coordinates. 

Now run SOLVE a second time.  When the program asks you for the type of
transformation (TRANSFORM\_CODE), type a `2', i.e., transform (alpha,delta)
$\rightarrow$ (x,y). The program will then ask you to enter the coordinates of
your source positions. You should type all of the alphas and deltas in the
given format, (put a blank between alpha \/ delta as opposed to a comma), and
terminate your input with a blank line (i.e. an extra `return'). The rest of
the program is then run in the same way as described previously in section 
\whichref{}{}. 

The output from SOLVE will now contain a list of x-y coordinates for your
source positions.  You can use these to set the measuring machine to the
appropriate position, and then measure candidate objects in the vicinity.  To
do this, run MEASURE as described in steps III.a-d. However, when you are
asked: 

- NEWFILE? type a `no' (i.e., use the Old file).  The program then asks you how
many objects were measured in the previous run of the program (it should be
able to keep count itself, but this is a very stupid program): 

- OLD\_OBJECTS? Type in the number of objects previously measured.  This is
{\it not} the highest numbered standard star, but rather simply the total
number of objects measured.  At this point you can measure the positions of
candidate objects just as in step III.f.  It is not necessary to measure any
standard star positions again (i.e., skip step ...), as your unknown object x-y
positions are simply being appended to your previous measurements. 

When you have finished, please do not forget to logout; just type `lo' on the
terminal. Turn off all the bits of the measuring machine and put the green
cover over the camera and table. Put any PSS prints carefully back in their
folders, and replace them at the right place in the cabinet. 



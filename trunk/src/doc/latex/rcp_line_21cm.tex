%
% @(#) rcp_line_21cm.tex  v1.2 04/08/93  AGB
%
% History:
%       JPH 940914      Comment out 'processing appetizer'
%
%
\chapter{Processing recipe: 21cm line observations} 
\tableofcontents 

Author: A.G. de Bruyn 


%=============================================== Standard subsection ======
\section{Scope of the recipe} 
\label{.scope} 

This recipe deals with the reduction of a 21cm line dataset. It describes the
calibration of the (complex) passbands using one or more calibrators, the
flagging of bad data and the production of an image cube. It shows you how to
display this image cube in a movie-like fashion using GIDS. The recipe also
includes a description of how to visually identify groups of bad datapoints
(e.g. bad interferometers or bad channels) and flag/clip those data. If the
field is littered with bright continuum sources at the edge of the field, or
when you are working with a relatively wide band, the subtraction of the
continuum in the image plane will produce undesirable chromatic effects (the
continuum emission itself will go away but not their frequency- scaled sidelobe
and grating-lobe responses). A simple procedure to remove the continuum in such
cases will be given. 

%---------------------------------------------------------------------------
%  If possible, supply an idea for an "appetizer" picture, which gives an
%  idea of what can be achieved with this recipe.

%\begin{figure}[htbp]
%\vspace{10cm}
%\caption[001 processing appetizer:...]         % appears in listoffigures
%{\label{.appetizer}}
%\begin{center}\parbox{0.9\textwidth}{\it       % actual caption if long
%001 processing appetizer:...
%\\This result was obtained in the following way:
%\\Note the following features:
%}\end{center}
%\end{figure}


%=============================================== Standard subsection ======
\newpage 
\section{Introduction and background} 
\label{.intro} 

This recipe describes the series of steps to go through when you want to
calibrate a WSRT 21cm line observation. It assumes that the S/N in the data is
rather low and that selfcalibration is not required. It also assumes that you
do not have to clean the final images. 


%=============================================== Standard subsection ======
\section{Summary of the recipe} 
\label{.summary} 

The following is a step-by-step summary of the processing recipe. For some of
these steps, more detail is provided below. 

\begin{enumerate} 
\item {\bf Load your data:} from tape or optical disk (NSCAN). See also Recipe
005. 
\item {\bf Inspect the data file layout:} (NSCAN) 
\item {\bf Determine complex passbands:} from a calibrator (NCALIB) 
\item {\bf Inspect the calibration quality:} (NPLOT) 
\item {\bf Transfer the calibration corrections:} (NCALIB) 
\item {\bf Make a preliminary cube of images:} (NMAP) 
\item {\bf Inspect the image cube:} (NGIDS) 
\item {\bf Subtract the continuum:} (NMAP) 
\item {\bf Make a cube of (calibrated) visibilities:} (NMAP) 
\item {\bf Inspect the UV-data cube:} (NGIDS) 
\item {\bf Plot your images:} (NPLOT) 
\item {\bf Transfer your images to GIPSY or AIPS:} for further analysis. 
\item {\bf Dealing with remaining imperfections:} If there are bad groups of
datapoints or if the continuum has not been removed satisfactorily the
following additional steps may be required: 
  \begin{itemize} 
  \item {\bf Flag bad groups of data:} (NSCAN) 
  \item {\bf Make a new image cube:} (NMAP) 
  \item {\bf Find and Update source parameters:} 
  of discrete continuum sources(NMODEL) 
  \item {\bf Make a new image cube:} 
  with discrete sources subtracted (NMAP) 
  \end{itemize} Then proceed with step 7 and iterate if necessary. 
\end{enumerate} 




%=============================================== Standard subsection ======
\section{More details for some of the steps} 
\label{.detail} 


%-------------------------------------------------------------------------
\subsection{Load your data} 
\label{.detail.load} 

Log in on the computer on which you want to read in the data. At present this
is either the microVAX (RZMVX4) or the Alliant. You load data with the program
NSCAN, option LOAD. Before you load your data you must know with what time
resolution you want to read in the data, and which channels you wish to
process. If your object is confined to a small region (say less than 10 arcmin)
it may be sufficient to use 120 seconds averaging. If you wish to synthesize
bigger fields I recommend using 60 seconds averaging, in order to avoid too
much tangential smearing. If the data were Hanning tapered in Westerbork
reading in every other channel may speed up the processing by a factor of two.
Hence for a 32 channel observation you could answer at the channel question: "2
to 28 by 2". Reading in the continuum channel (channel 0) is generally not
necessary; it contains continuum data with contaminating line emission. It is
also advised to read in the calibrator data into the same scanfile. If you do
this in the second jobstep the data will be assigned the next highest 'group
index'. 

%-------------------------------------------------------------------------
\subsection{Inspect the file layout and flag/clip bad points} 
\label{.detail.inspdata} 

To find out how the data in the file is organized, and how you can access
certain subsets of it, it is suggested to move 'up and down' within the program
NSCAN. There are five levels of actions to choose, all of which work on
different parts of the dataset. The layout of the file, or the function of the
five-digit 'index' that points to the various observations, channels etc. is
particularly important to memorize. To avoid that obviously bad points (e.g.
correlator spikes) ruin the calibration solution you may wish to flag
('delete') any points with a value higher than a certain limit.  Usually the
occurrence of correlator spikes is noted on the information you receive from
Westerbork/Dwingeloo. Flagging is done on a point-by-point basis using the
amplitude of the visibility. Be careful when you use this option if you have no
information on the amplitude range which is normal. Preliminary clipping of
calibrator data is generally not necessary. During program execution it will
inform you at which hourangles interferometers have been clipped.  The cutoff
value for the object data can of course be much lower. When visibility samples
have been clipped they will be ignored in all programs. You can also undo any
clipping; you do this by first typing "undelete" at the question where the type
of flagging criterion is selected (HA, AN, RN, IFR, CLIP ...). 

%-------------------------------------------------------------------------
\subsection{Determine the instrumental gains/phases across the passband} 
\label{.detail.passband} 

Run the (self-)calibration program NCALIB (option redundancy) for the
calibrator source(s). In order to do this you need a 'model' for the calibrator
field containing the fluxes and positions of the calibrator source and any
(strong) surrounding sources (see Recipe 013: Using external calibrators).
Depending on observing frequency and source there could be anywhere between one
and many hundreds of sources in the model file. Upon execution the complex
gains are determined, and stored, per telescope/polarisation for each time
interval and channel. 

%-------------------------------------------------------------------------
\subsection{Plot the calibration results} 
\label{.detail.plotcal} 

In order to decide which calibrator(s) to use, or which part of the calibration
observation, you must inspect the quality of the output from the calibartor
NCALIB run. Both the printed LOG from the previous step and/or a plot of the
telescope complex gains and the residuals from the (self-) calibration solution
are helpful to make up your mind. The program NPLOT (options 'telescope' and
'residuals') does the plotting. In general it suffices to plot one channel,
somehwere in the middle of the band, to judge the quality of the gains (called
amplitude) and phases for the various telescopes. Phase and gain slopes across
the passband are usually the same, to first order, for all telescopes. Phase
drifts at 21cm are typically a few degrees of phase over a 12 hour period; gain
drifts 
should generally be less than a few %. Gain/phase offsets however may
occasionally be larger. Consult the printout which contains summary information
for each channel. If there are no instrumental problems the gain and phase
errors should be equal to the thermal noise values (see Appendix ....). 


%-------------------------------------------------------------------------
\subsection{Transfer the gains/phases from the calibrator(s)} 
\label{.detail.transcal} 

The corrections from the calibrator(s) are averaged over all available (non-
flagged) hourangle scans and stored in each scan of the object scanfile. If
your calibrator observations have equal length, and no scans have been flagged,
the resulting value is just the mean of the individual calibrator averages.
There is no provision to weight for the S/N ratio or the amplitude of the
calibrator. This means that the transferred gains/phases are representative for
a point in time halfway the time of the calibrator observations. This would be
a good assumption for gain/phase drifts that go linear with time. For galactic
21cm line observations you will have calibrators observed at frequencies lower
and higher (usually by about 1 MHz) than the frequency of your object. Usually
these are scheduled before and after the time of observation of your object but
this is not essential. If the calibrator observations have equal length then
the averaged gain/phase would be appropriate for a frequency halfway the
calibrator frequencies as if there is a linear gain/phase drift with frequency.
If your object frequency is not halfway in frequency between the frequencies of
the calibrators and you want to store gains/phases in your object file that
refer to the frequency of your object, on the assumption that the gains/phase
vary linearly with frequency, than you may play around with the length of the
respective calibrator observations. This can be done using the NSCAN program
(option DELETE, suboption HA (delete). 

%-------------------------------------------------------------------------
\subsection{Make a cube of images} 
\label{.detail.makecube} 

Assuming that you do not want to selfcalibrate the object data (which for line
observations is usually the case) you can now proceed to make an image cube.
When making images you want to delete any obviously bad baselines. Because you
do not yet know whether such bad baselines exist you might as well proceed with
making your first series of images. It is possible to exclude visibilities that
have an amplitude higher than a certain value. This clipping is not permanent,
but only for the execution of NMAP. If the images look good, then all
participating baselines are probably good. In an observations with 32 frequency
channels, channel 1,29,30 and 31 are usually of poor quality so one would make
maps of channels 2 through 28 only. Note that the 3d digit in the index of the
.WMP cube will always start counting at 0 and continues through 26. That is,
channel 2 is really designated 0.0.0 and channel 28 is 0.0.26. 


%-------------------------------------------------------------------------
\subsection{Visual inspection of images} 
\label{.detail.visinsp} 

Making final, good quality, images is generally an iterative procedure where
you work in both the image plane and the UV-plane. This procedure is fastest if
you can load the cubes directly into the memory of your display (e.g. X-
terminal or workstation). GIDS, the Gipsy Image Display System, is a very
useful program to do this. Currently it only runs on the Sun If you want to
analyse the images in detail, using programs in GIPSY or AIPS) you will have to
make FITS-images first. This can also be done using NMAP (option w16fits or
w32fits) but is not shown here. Before you can run GIDS (via dwe NGIDS) you
have to switch to an X-terminal (in Dwingeloo we have four HP X-terminals) or
one of the two colour displays on the SUN (IPX's). 

%-------------------------------------------------------------------------
\subsection{Subtract the continuum form the line channels} 
\label{.detail.subcont} 

After inspecting the cube of line images you can decide which channels to use
for the continuum. Use the option FIDDLE (suboption SUM) to create an averaged
continuum image. Then use the option ADD to add the continuum channels with
weight -1 to each line image to create continuum-free line channels. 

%-------------------------------------------------------------------------
\subsection{Make a cube of the corrected visibilities} 
\label{.detail.cubecorr} 

If the continuum-free line images look fine you are finished. However, there
may be problems at some low level caused by interference, correlator errors or
telescope/interferometer errors. Problems at certain position angles in the
images correspond to certain hourangles and could be due to short-lived
interference. An error pattern in the images, which is centered on the field
centre (fringe-stopping centre) and not on strong continuum sources, shows that
we are dealing with an additive, rather than multiplicative, error suggests
that we are dealing with (DC-) offsets in the backend. To find out which
baselines are causing the damage you can make a cube of the (calibrated)
visibilities before they are gridded and transformed. You should use the option
BASHA instead of UV in NMAP to generate such a cube. Answers to the questions
on FFT and map size should be the defaults that are suggested. The best way to
pick up low-level problems in the data (e.g. small offsets) is to create
outputs for the real (COS) and imaginary (SIN) data. Working with the
amplitudes and phases is generally only useful if you want to identify problems
of a multiplicative nature and when the signal is well above the noise per
visibility point. 


%-------------------------------------------------------------------------
\subsection{Visual inspection of (calibrated) visibilities} 
\label{.detail.inspcalvis} 

The cube of UV-data has a different size than that of the images. With samples
of 240 sec there are about 180 points in the horizontal direction and 152
points in the vertical direction (4 x 38). By choosing an 18m baseline
increment when making the cube you will get the various baselines well
separated and makes it easier to count on the screen which baselines are bad.
The shortest baseline (9A=72m) is at the bottom of the 'image'. You can now
proceed to make a new series of images where baseline 5B should be thrown out.
Then you can define the 'continuum' channels and run NMAP to calculate an
average continuum channel. This is not shown here because you may want to do
this all in GIPSY or AIPS using FITS images. For the purpose of the
demonstration this was done in Dwingeloo and the final 'continuum corrected'
cube shows images of channels 7 though 23. These images were 512 x 512 in size
and covered an area of 0.6 degrees. 


%-------------------------------------------------------------------------
\subsection{Make contour plots of images} 
\label{.detail.contour} 

Use the program NPLOT, option MAP to make contour plots of images. There is a
variety of output devices. In Dwingeloo you could first send a plot to a
(graphics-) terminal  (option REGIS) to check the contourvalues that you would
like to plot. When you are satisfied you can, by using the 'loop' option, send
a series of plots to the QMS plotter in either portrait (QMSP) or landscape
(QMS) mode. The former gives you, for square plotareas, somewhat more space to
plot in, if at least you want to restrict yourself to a single A4 sheet.
Dealing with the size question will take some experience. The default size of
1,1 will give you a plot of at most 5 inch. This is the case if the number of
grid points is a power of two (32, 64, 128 etc). When using a size greater than
1,1 on a power-of-two image area in landscape mode the plot will not fit on one
sheet and you have to use scissors and tape. On the portrait mode you can
actually increase the size to 1.4,1.4 before it needs more than a single A4
sheet. 


%-------------------------------------------------------------------------
\subsection{Transfer images to AIPS or GIPSY} 
\label{.detail.transcube} 

You can use the FITS write option in NMAP to generate 2 byte or 4 byte integer
images in FITS format for further analysis in AIPS or GIPSY. 





%==========================================================================
\section{Dealing with remaining imperfections} 
\label{.remain} 

If the inspection of the continuum subtracted image cube shows imperfections
due to bad data or poor continuum subtraction you have to continue the
reduction a bit further. 


%-------------------------------------------------------------------------
\subsection{Further flagging of bad data.} 
\label{.remain.flag} 

If the inspection of the visibility cube using NGIDS has indicated that there
are errors in certain hour-angles or baselines you can flag these using the
delete option in NSCAN. If there are only bad baselines you can also decide to
exclude these baselines when preparing the specifications for a new NMAP. That
is, you do not have to actively delete them. 

%-------------------------------------------------------------------------
\subsection{Make a new image cube.} 
\label{.remain.newcube} 

If the new image cube shows, after subtracting some averaged continuum image,
residual (chromatic) grating rings due to strong sources at the edge of the
field you need to subtract these sources in the original UV-data. 

%-------------------------------------------------------------------------
\subsection{Finding and updating discrete source parameters.} 
\label{.remain.sourcepar} 

In order to determine the flux densities and positions of these sources you
search the continuum image using the FIND option in NMODEL. To improve the
estimates of flux and coordinates you can use the UPDATE option in NMODEL. A
single update, using only the UV-data with coninuum emission, usually suffices.
 The update algorithm will only work for discrete (point) sources. If there are
extended sources with residual grating lobes you need to use NCLEAN (option
Beam) to decompose that source into a number of delta- functions. The final
list of components (updated discrete sources and clean- components) can be
merged into one model file. 

%-------------------------------------------------------------------------
\subsection{Make a new image cube with sources subtracted.} 
\label{.remain.subtr} 

Now you proceed with step 6, but subtract the list of sources found in the
previous step. 

%=======================================================================
%=======================================================================

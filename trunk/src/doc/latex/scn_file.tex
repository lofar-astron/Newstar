%scn_file.tex 
% 
% JPH  941125 
% HjV  950619   Correct some typo's
% JPH  960513   References to sch.dsc, sth.dsc 
% 
% 
\newcommand{\bi}{ \begin{itemize} } 
\newcommand{\ei}{ \end{itemize} } 
\newcommand{\bn}{ \begin{enumerate} } 
\newcommand{\en}{ \end{enumerate} } 
\newcommand{\eg}{ {\em e.g. } } 

\title{ The .SCN file } 
\maketitle 

\chapter{ Visibilities and associated data: The .SCN-file } 

{\par \em Contributed by Johan Hamaker, November 1994 \centering \par} 

\tableofcontents 

\section{ Overview of .SCN file contents} 
\label{.scn.file} 

\input{ scn_summary.tef } 

\section{ The visibility hypercube and its dimensions } 
\label{.hypercube} 

\input{scn_hierarchy.cap} 

        Visibilities in a synthesis observation are functions of a series of
coordinate parameters. One may conceptually arrange them in a {\em hypercube}
whose coordinate axes correspond to these parameters; alternatively, this
hypercube may be visualised as a hierarchy in which each level corresponds to
one of the parameters. 

        The organisational model used in the .SCN file is a hyprid, in which
the lowest levels form a compact array, which is embedded in a hierarchical
{\em index structure} for the remaining levels. The order of the parameters in
this arrangement is in principle quite arbitratry. \NEWSTAR's choice was
motivated by considerations of efficiency in both storage space and sorting for
the most important processes. The order is, from the bottom upwards: 

\bi 
\item   {\em Polarisation:} Each telescope contains 2 orthogonal dipoles named
{\em X} and {\em Y}. Depending on the observing mode, there may be 4 ({\em XX,
XY, YX, YY}), 2 ({\em XX, YY}) or 1 ({\em XX}) polarisation. 

\item   {\em Interferometer:} The WSRT consisting of 14 telescopes, the number
of interferometers formed between the elements can be 91 at most. Depending
upon the availability of telescopes and the observing modes, the actual number
(NIFR) is usually somewhat smaller. In the .SCN file, the interferometers are
sorted in order of ascending baseline and assigned a sequence number from 0 to
NIFR-1. 

\item   {\em Hour Angle:} The hour angle for the middle of the integration
interval to which the data pertain. Hour angles are referred to as such. The
integration interval is a multiple of 10 UT seconds and is defined by the user
when he reads in the data. 

\item   {\em Frequency Channel:} For observations in several (NCHN)
simultaneous frequency bands, these bands are numbered from 1 through NCHN. In
a single observation, the bands are not necessarily equally wide and
equidistant. By convention, the sum of all bands is called the {\em continuum
channel} and assigned channel number 0. 

\item   {\em Mosaic Subfield} or {\em Pointing Centre:} In a mosaic
observation, subfields are repeatedly observed in a fixed sequence. The fields
are assigned a sequence number starting at 0. By convention, field 0 is at the
mosaic centre. 

\item   {\em Observation:} An observation is a 'container' holding all the data
that were read from a single WSRT observation {\em label} in a single run of
NSCAN.load. 

\item   {\em Group:} A group is likewise a container holding all observations
loaded in a single NSCAN.load run. 
\ei 

        It is customary to loosely refer to the levels of this hierarchy as the
{\em dimensions} of a six-dimensional {\em data (hyper)cube}. One should bear
in mind, however, that the same 'dimension' may be of different magnitude in
different 'sub-cubes' of this hypercube or, in other words, that the 'cube' is
not regularly filled. 


\subsection{ .SCN-file hierarchy and indexing } 
\label{.hierarchy} 

        An overview of the .SCN-file hierarchy is shown in 
\figref{.scn.hierarchy}. The levels correspond to those discussed 
\textref{above}{.hypercube}. We shall discuss them from the bottom up. 

        It is worth noting that the scan and sector headers combined contain
all the information necessary to process the data. A sector may be divorced
from the .SCN file without impairing its ability to be processed. The index
blocks in the levels on top contain no 'scientific' information. Their only
purpose is to organise the sectors. 


\subsubsection{ The scan} 
\label{.scan} 

        The {\em scan} is the fundamental agggregate of data in a .SCN file. It
is a two-dimensional array of visibilities for a single integration interval,
with polarisation and interferometer as its dimensions. 

        Associated with this block of data is a {\em scan header}. It contains
descriptive data such as frequency and position parameters. In addition, it
holds tables of those gain and phase corrections that may change from scan to
scan, as well as data and correction statistics that are a measure of the
data's quality. The contents are listed in the \Srcref{definion
file}{nscan/sch.dsc}.  


\subsubsection{ The sector} 
\label{.sector} 

        A sequence of scans that is contiguous in hour angle is grouped
together in a {\em sector}. The sector's contents are described in a {\em
sector header}. Like the scan header, it contains descriptive data, corrections
parameters that may be assumed to be constant for the time period covered by
the sector and data/correction statistics for all the sector's scans combined.
The contents are listed in the \Srcref{definion file}{nscan/sth.dsc}. 


\subsubsection{ The indexing levels } 
\label{.levels} 
\label{.SCNSUM.indices} 


        The \textref{index structure}{file_indexing} in which the sectors are
organised follows the levels listed \textref{above}{.hypercube}. The 
\verb/<sequence number>/ is an extra index that allows one to have more than
one sector for which the first four indices are identical. A complete 
.SCN-sector designation reads: 
\bi 
\item[] \verb/<group>.<observation>.<field>.<channel>.<sequence number>/ 
\ei 

\input{mosaic_sectors.cap} 

        One particularly important application of this possibility is in the
organisation of the non-contiguous scans in a \whichref{mosaic observation}{ }.
In such an observation, the scans are not contiguous and therefore each
'hour-angle cut' forms a sector of its own (\figref{.mosaic.sectors}). 



\subsection{ How index values are allocated} 
\label{.index.define} 

\input{scn_indices.cap} 

        The method by which NSCAN allocates sector numbers is schematically
shown in \figref{.scn.indices}. The only control the user has is over the
allocation of groups: Every NSCAN.load operation creates a new group, and the
user defines which WSRT labels will be stored in that group. 



\section{ Visibility data, weights and flags } 

        Each visibility point is stored in 3 contiguous 16-bit words. Two of
these represent the real and imaginary parts of the complex visibility in {\em
Westerbork Units} ({\em W.U}), 1 W.U = 5 mJy). Of the remaining two bytes, one
holds the 8 \whichref{{\em flags}}{}, the other a \whichref{{\em weight}} value
related to the 'probable error' in the data. 


\section{ The headers } 

        We give an outline here of the header contents. To inspect them in
detail, use the NSCAN SHOW or NFLAG SHOW functions. More complete though very
terse descriptions can be found in the definition files (see below). 


\subsection{ The file header } 

        At the beginning of the .SCN file is the {\em file header}. It contains
some administrative information such as the last time the file was opened for
writing, and pointers into the index structure. For more details see the 
\srcref{definition}{wng/gfh.dsc} file. 


\subsection{ The sector header } 

        The sector header contains parametric data tha are constant throughout
the sector. The most important groups are the following: 

\bi 
\item   Coordinates: 
  \bi 
  \item   right ascension (RA) and declination (DEC), both apparent and epoch; 
  \item   start hour-angle (HAB) and increment between scans (HAI); 
  \item   LSR frequency, frequency-channel number (CHN); 
  \item   mosaic-subfield number (PTS); 
  \item   telescope positions (RTP). 
  \ei 

\item   Sector composition: Number of scans (SCN), number of 
        \whichref{polarisations}{} (PLN), bandwidth (BAND), interferometer list 
        (IFRP). 

\item   Corrections: \whichref{dipole errors}{}. 

\item   Noise statistics reported by the most recent 
        \textref{selfcal}{introduction.selfcal} extraction of corrections from 
        the visibilities. 

\item   Administrative information: length of a scan header plus data (SCNL). 
\ei For more details see the \srcref{definition}{nscan/sth.dsc} file. 


\subsection{ The scan header } 

        The scan header contains the parametric data that apply specifically to
one scan: 

\bi 
\item   Coordinate: Hour angle 
\item   Corrections: 
  \bi 
  \item   Gain and phase corrections per 
          \textref{telescope}{introduction.SCNSUM}; 
  \item   \textref{global}{introduction.selfcal} corrections that may vary with 
          hour angle: tropospheric and ionospheric refraction, extinction, 
          Faraday rotation, clock; 
  \ei 
\item   Noise statistics reported in the most recent 
        \textref{selfcal}{introduction.selfcal} extraction of corrections from 
        the visibilities. 
\ei For more details see the \srcref{definition}{nscan/sch.dsc} file. 


\section{ Operations on .SCN files in general} 
\label{.operations} 

\subsection{ Creation } 
\label{.creation} 

\bi 
\item   From WSRT observations files: \textref{NSCAN LOAD}{nscan_descr}. 
\item   From ATCA (Australia Telescope Compact Array) files: \whichref{NATNF 
        LOAD}{(undocumented)}. 
\item   From old (R-series) SCN-files: \textref{NSCAN FROM\_OLD}{nscan_descr}. 
\item   Simulated uv-data: \whichref{NSIMUL}{(undocumented)}. 
\ei 


\subsection{ Inspection } 
\label{.inspect} 

Displays in tabular form: 
\bi 
\item   Summary of contents: \textref{NSCAN SHOW}{nscan_descr}. 
\item   Summary of one sector or scan header at a time: \textref{NSCAN SHOW
CONT 
        [CONT]}{nscan_descr}. 
\item   Complete sector or scan header, one at a time: \textref{NSCAN SHOW CONT 
        [CONT] SHOW}{nscan_descr}. 
\item   Visibilities, weights, flags for one scan at a time: \textref{NSCAN
SHOW 
        CONT CONT DATA/WEIGHTS}{nscan_descr}. 
\item   Telescope gain/phase corrections averaged over a range of scans: 
        \textref{NCALIB SHOW}{ncalib_descr} 
\item   Telescope dipole corrections averaged over a range of scans: 
        \textref{NCALIB POLAR SHOW}{ncalib_polar} 
\ei 

\noindent Graphic displays: 
\bi 
\item   Object or model visibilities vs. hour angle: \whichref{NPLOT 
        DATA/MODEL}{ (undocumented)} 
\item   Object or model visibilities in the Cartesian UV plane: 
        \textref{NMAP..}{nmap_descr} 
\item   Gain/phase corrections vs. hour angle: \whichref{NPLOT 
        TELESCOPE}{ (undocumented)}. 
\item   Redundancy/Selfcal residuals: \whichref{NPLOT RESIDUAL}{
(undocumented)} 
\ei 

        Additional possibilities for extracting. manipulating and displaying a
variety of items from the .SCN file are available in 
\textref{NGCALC}{ngcalc_descr}. 


\subsubsection{ Editing } 
\label{.edit} 

        Almost every value (observation parameters, corrections, etc) in the
SCN-file headers may be edited manually through \textref{NSCAN SHOW
EDIT}{nscan_descr}. The effect of changing a value may range from trivial to
catastrophic. If you run into a situation that you think can only be remedied
by manual editing, you do better to contact the \textref{\NEWSTAR
group}{people} first. 


\subsection{ Export } 
\label{.export} 

\bi 
\item   In UVFITS format (AIPS): \textref{NSCAN UVFITS}{nscan_descr} 
\item   Inspect a UVFITS file: \textref{NSCAN PFITS}{nscan_descr} 
\ei 


\subsection{ Reorganising sectors } 
\label{.reorganise} 

        You may want to reorganise the indexing of your sectors. This can be
done to some extent through the \textref{NSCAN REGROUP}{nscan_descr} function. 


\subsection{ Deleting sectors } 
\label{.delete} 

        The most important reason why one would want to delete sectors is to
reclaim disk space. It would be fairly simple to provide a DELETE command that
would make sectors invisible, but this would not free any space. 

        An equivalent result can be achieved, however: Use 
\textref{NCOPY}{ncopy_descr} to copy those sectors that you want to keep to a
new .SCN file, then delete the old file. 

. 
\section{ Operation on corrections in a .SCN file} 
\label{.corrections} 

        Operations on corrections generally fall into either of two categories: 
\bi 
\item   Those that affect the gain/phase of individual dipole/interferometer 
        hannels. This encompasses all corrections, except 

\item   Those dealing with polarisation. These are 'special' in that they 
        produce an effect of 'mixing' or 'crosstalk' between channels that
would 
        ideally be independent. 
\ei 


\subsection{ Gain/phase corrections} 
\label{.gain.phase} 

\bi 
\item   Zero, manual set: \textref{NCALIB SET ZERO/MANUAL}{ncalib_descr} 
\item   Selfcal model fit: \textref{NCALIB REDUN}{ncalib_redun} 
\item   Shift the reference value of telescope phases: \textref{NCALIB 
        RENORM}{ncalib_redun} 
\item   Copy average corrections from one selection of scans to another one: 
        \textref{NCALIB SET COPY/CCOPY/LINE}{ncalib_descr} 
\ei 


\subsection{ Polarisation corrections } 
\label{.polarisation} 

Dipole errors: 
\bi 
\item   Zero, manual set: \textref{NCALIB POLAR ZERO/MANUAL}{ncalib_polar} 
\item   Estimation: \textref{NCALIB POLAR CALC}{ncalib_polar} 
\item   Copy average corrections from one selection of scans to another one: 
        \textref{NCALIB POLAR COPY}{ncalib_polar} 
\ei 
\noindent Phase-zero difference: 
\bi 
\item   Manual set: \textref{NCALIB POLAR VZERO MANUAL}{ncalib_polar} 
\item   Estimation: \textref{NCALIB POLAR VZERO CALC}{ncalib_polar} 
\item   Copy average corrections from one selection of scans to another one: 
        \textref{NCALIB POLAR VZERO COPY}{ncalib_polar} 
\ei 


\subsection{ Controlling the application of corrections } 
\label{.apply} 

        Corrections are selectively applied to the visibilities whenever the
data is read into memory to be processed. By default, all available corrections
are selected. The user has the option to make his own selection. To this end,
start your program with the \whichref{/ASK switch}{}: 
\bi 
\item[] \verb:  dwe <program> /ASK: 
\ei This will result in a long series of prompts. Simply type a \verb/<CR>/ to
all, except those for the \verb/APPLY/, \verb/DE_APPLY/ and \verb/FLAG/
parameters. More information can be found in the \textref{on-line
help}{introduction.help} texts. 

        (We acknowledge that the appearance of many irrelevant prompts in this
situation is unsatisfactory. We hope to correct this situation in the future.) 


\section{ Operations on flags } 

{\em Yet to be written } 


\section{ Operations on model visibilities } 
\label{.model} 

\bi 
\item   Inserting/modifying the internal model: Use the \textref{model 
        HANDLE}{nmodel_handle} interface in NMODEL, NCALIB or NMAP. 
\item   Deleting the indernal model: There is no operation to do this. See the 
        \textref{model HANDLE}{nmodel_handle} document for a workaround. 
\ei 




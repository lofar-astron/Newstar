
\chapter{SHOWing and EDITing data-file contents in \NEWSTAR} 
\\ \\
{\it 
\indent  Contributed by Wim Brouw, December 1993 \\ 
\indent  Revised/adapted by Johan Hamaker, June 1994 } 

\tableofcontents 


\section{ Introduction} 


	The programs NSCAN, NFLAG, NMAP and NGCALC have a SHOW option, with an
EDIT sub-option.  This EDIT sub-option can be used to examine and change any
field in the SCN, WMP and NGF files. 

	The EDIT option can show and/or edit the contents of \NEWSTAR files.
All values ('fields'), with the exception of values determining the structure
of the file, may be edited individually (or in groups) by hand.  This is
laborious, {\em but at least it is possible!} 

	The different files contain a variety of blocks (e.g.  STH - sector
header; MPH - map header).  In addition to manipulating these fields, \NEWSTAR
programs know how to present their names and values in an intelligible form to
the user, and conversely how to interpret the user's editing instructions. 

	After starting EDIT, the message: 

	\verb/**** Editing STH **** (or MPH or ....)/ 

\noindent will appear, indicating the actual block available for editing at
this instant. 

\section{SHOW and EDIT} 

	{\bf SHOW} gives a formatted display of the contents of the entire
"current" block (i.e.  the block currently accessed through a navigating
command such as {\bf CONTINUE} or {\bf NEXT}.  To display individual fields in
the block one uses the {\bf EDIT} option.  Once in edit mode, one may both
display and modify fileds. 

	The {\it display} commands for an individual field consist of the field
name, optionally followed by a colon and qualifying information.  The various
types of qualifier will be discussed below.  A display command becomes an {\bf
EDIT} command when followed by a comma plus one or more new values. 

\spbegin %.+.+.+.+.+.+.+.+.+.+.+.+.+.+.+.+.+.+.+.+.+.+.+.+.+.+.+.+.+.+.+.+
\svbegin \begin{verbatim} 
 *** Editing STH *** 
\end{verbatim}\svend 
%
\skeyword{EDIT} 
\sprompt{(Edit: name [(offset)][/length][:format] , val [, ...])} 
\sdefault{= "":} 
\suser{ra:d} 
%
\svbegin \begin{verbatim} 
 RA             0.06725881318561733
\end{verbatim} 
\svend 
\spend %.-.-.-.-.-.-.-.-.-.-.-.-.-.-.-.-.-.-.-.-.-.-.-.-.-.-.-.-.-.-.-.-.-

\noindent Note that \textref{pointer values}{.pointers} can not be shown this
way. 

\section{ Displaying data values and properties} 

\subsection{ Displaying data values in different formats } 
\label{.show.in.format} 

	Appending a format specifier to the colon overrides the default format
in which a value is displayed.  Some of the possible formats are: 

\begin{tabular}{lll} 

	&\verb/UB UI UJ/                        &unsigned byte, I, J \\ 
	&\verb/SB SI SJ/                        &signed same \\ 
	&\verb/XB XI XJ/                        &hexadecimal same \\ 
	&\verb/OB OI OJ/                        &octal same \\ 
	&\verb/AL<n>/                           &<n> characters \\ 
	&\verb/E[<n>[.<m>]]/                    &real \\ 
	&\verb/D[<n>[.<m>]]/                    &double precision \\ 
	&\verb/EC or DC[<n>[.<m>]]/             &complex \\ 
	&\verb/{ED}{AHD}{FRD}[<n>[.<m>]]/       &Angle, hh:mm:ss, dd.mm.ss
for\\ 
	&                                       &\ Fractions, radians,
degrees\\ 
\end{tabular} 

\spbegin %.+.+.+.+.+.+.+.+.+.+.+.+.+.+.+.+.+.+.+.+.+.+.+.+.+.+.+.+.+.+.+.+
\svbegin \begin{verbatim} 
 *** Editing STH *** 
\end{verbatim}\svend 
%
\skeyword{EDIT} 
\sprompt{(Edit: name [(offset)][/length][:format] , val [, ...])} 
\sdefault{= "":} 
\suser{ra:dhf} 
%
\svbegin \begin{verbatim} 
 RA                        01:36:51
\end{verbatim} 
\svend 
\spend %.-.-.-.-.-.-.-.-.-.-.-.-.-.-.-.-.-.-.-.-.-.-.-.-.-.-.-.-.-.-.-.-.-
%
\spbegin %.+.+.+.+.+.+.+.+.+.+.+.+.+.+.+.+.+.+.+.+.+.+.+.+.+.+.+.+.+.+.+.+
\skeyword{EDIT} 
\sprompt{(Edit: name [(offset)][/length][:format] , val [, ...])} 
\sdefault{= "":} 
\suser{ra:dhf10} 
%
\svbegin \begin{verbatim} 
 RA                   01:36:51.1615
\end{verbatim} 
\svend 
\spend %.-.-.-.-.-.-.-.-.-.-.-.-.-.-.-.-.-.-.-.-.-.-.-.-.-.-.-.-.-.-.-.-.-
%
\spbegin %.+.+.+.+.+.+.+.+.+.+.+.+.+.+.+.+.+.+.+.+.+.+.+.+.+.+.+.+.+.+.+.+
\skeyword{EDIT} 
\sprompt{(Edit: name [(offset)][/length][:format] , val [, ...])} 
\sdefault{= "":} 
\suser{ra:xj} 
%
\svbegin \begin{verbatim} 
 RA                        3fb137d4 
\end{verbatim} 
\svend 
\spend %.-.-.-.-.-.-.-.-.-.-.-.-.-.-.-.-.-.-.-.-.-.-.-.-.-.-.-.-.-.-.-.-.-
%

	The full list of available formats, including fieldsize indicators,
exists only in the form of a table in the program source file
\verb/wnctxt_x.for/. 


\subsection{ Displaying parts of arrays} 

	Elements of array fields can be selected by appending the start index
and the number of elements to the field name, as follows: 

	\verb:- <name>[(<offset>)][,/<number>]: 

\noindent will show the value of \verb/<name>/.  If an offset is given for a
multi-valued \verb/<name>/, the display will start at this value; if
\verb:/<number>: is given, only that many values will be shown. 
Examples are given \textref{below}{.edit}.  Remember that {\bf indices start at
0!}. 


\subsection{ Displaying information about data fields} 
\label{.show.meta} 

\begin{tabular}{lll} 
&\verb/*/       &will show the entire block formatted in the same way as \\ 
&               &\ for a {\bf SHOW} command \\ 
&\verb/:/       &will show the names and formats of all data fields in the
block \\ 
&\verb/<name>:/ &will show format information for \verb/<name>/ \\ 
\end{tabular} 

\noindent Examples: 

\spbegin %.+.+.+.+.+.+.+.+.+.+.+.+.+.+.+.+.+.+.+.+.+.+.+.+.+.+.+.+.+.+.+.+
\svbegin \begin{verbatim} 
 *** Editing STH *** 
\end{verbatim}\svend 
%
\skeyword{EDIT} 
\sprompt{(Edit: name [(offset)][/length][:format] , val [, ...])} 
\sdefault{= "":} 
\suser{:} 
%
\svbegin \begin{verbatim} 
 Known names: 
	      LINK,      XJ, , ;     0,     2,     1,     4
	      LEN,       SI, , ;     8,     1,     1,     2
	      VER,       SI, , ;    10,     1,     1,     2
	      FIELD,     AL, , ;    28,     1,     0,    12
			 .
			 .

	      RA,   DAF12.7, , ;    40,     1,     0,     8
	      DEC,  DAF12.7, , ;    48,     1,     0,     8
	      RTP,    E12.4, , ;   124,    14,     0,     4
	      NIFR,      SJ, , ;   180,     1,     1,     4
	      IFRP,      XJ, ,   P:IFRT;   184,     1,     1,     4
	      NFD,       SJ, , ;   188,     1,     1,     4
	      FDP,       XJ, ,    P:FDW;   192,     1,     1,     4
	      NOH,       SJ, , ;   196,     1,     1,     4
			.
			.

\end{verbatim} 
\svend 
\spend %.-.-.-.-.-.-.-.-.-.-.-.-.-.-.-.-.-.-.-.-.-.-.-.-.-.-.-.-.-.-.-.-.-

\noindent The entries in this table are: 

\begin{tabular}{lll} 
&column 2:      &the default \textref{display/value format}{.show.in.format} \\ 
&column 3:      &the byte offset of the field in the block \\ 
&column 4:      &the number of elements in the field \\ 
&column 5:      & *** ?? *** \\ 
&column 6:      &the byte size of a single element (which is the string
length\\ 
&               &\ for a character value) 
\end{tabular} 


\section{ Modifying data values } 
\label{.edit} 

	Most values in a data file can be edited.  The field descriptors
outlined above for {\bf SHOW} are also valid for {\bf EDIT}.  Edit commands are
charaterised by the presence of a {\bf comma} following the field descriptor: 

\indent \verb:<name>[<(offset)>][</number>], <value> [,<value>...]: 

\noindent will change the value(s) at the given field. Examples: 

%
\spbegin %.+.+.+.+.+.+.+.+.+.+.+.+.+.+.+.+.+.+.+.+.+.+.+.+.+.+.+.+.+.+.+.+
\svbegin \begin{verbatim} 
 *** Editing STH *** 
\end{verbatim} 
%
\skeyword{EDIT} 
\sprompt{(Edit: name [(offset)][/length][:format] , val [, ...])} 
\sdefault{= "":} 
\suser{RTP,12 } 
%
\svbegin \begin{verbatim} 
 RTP          12.0000     143.9919     287.9837     431.9756     575.9674
	     719.9592     863.9511    1007.9429    1151.9348    1295.9266
	    1367.9257    1439.9176    2663.8491    2735.8452
\end{verbatim} 
\svend 
\spend %.-.-.-.-.-.-.-.-.-.-.-.-.-.-.-.-.-.-.-.-.-.-.-.-.-.-.-.-.-.-.-.-.-
%
\spbegin %.+.+.+.+.+.+.+.+.+.+.+.+.+.+.+.+.+.+.+.+.+.+.+.+.+.+.+.+.+.+.+.+
\skeyword{EDIT} 
\sprompt{(Edit: name [(offset)][/length][:format] , val [, ...])} 
\sdefault{= "":} 
\suser{rtp(2)/2, 1, 2, 3, 4} 
\sinline{ Note that names are not case-sensitive!} 
%
\svbegin \begin{verbatim} 
 RTP           1.0000       2.0000
\end{verbatim} 
\svend 
\spend %.-.-.-.-.-.-.-.-.-.-.-.-.-.-.-.-.-.-.-.-.-.-.-.-.-.-.-.-.-.-.-.-.-
%
\spbegin %.+.+.+.+.+.+.+.+.+.+.+.+.+.+.+.+.+.+.+.+.+.+.+.+.+.+.+.+.+.+.+.+
\skeyword{EDIT} 
\sprompt{(Edit: name [(offset)][/length][:format] , val [, ...])} 
\sdefault{= "":} 
\suser{rtp, 0} 
%
\svbegin \begin{verbatim} 
 RTP           0.0000     143.9919       1.0000       2.0000     575.9674
	     719.9592     863.9511    1007.9429    1151.9348    1295.9266
	    1367.9257    1439.9176    2663.8491    2735.8452
\end{verbatim} 
\svend 
\spend %.-.-.-.-.-.-.-.-.-.-.-.-.-.-.-.-.-.-.-.-.-.-.-.-.-.-.-.-.-.-.-.-.-
%
\spbegin %.+.+.+.+.+.+.+.+.+.+.+.+.+.+.+.+.+.+.+.+.+.+.+.+.+.+.+.+.+.+.+.+
\skeyword{EDIT} 
\sprompt{(Edit: name [(offset)][/length][:format] , val [, ...])} 
\sdefault{= "":} 
\suser{rtp(2), 287.9837, 431.9756} 
%
\svbegin \begin{verbatim} 
 RTP         287.9837     431.9756     575.9674     719.9592     863.9511
	    1007.9429    1151.9348    1295.9266    1367.9257    1439.9176
	    2663.8491    2735.8452
\end{verbatim} 
\svend 
\spend %.-.-.-.-.-.-.-.-.-.-.-.-.-.-.-.-.-.-.-.-.-.-.-.-.-.-.-.-.-.-.-.-.-
%
\spbegin %.+.+.+.+.+.+.+.+.+.+.+.+.+.+.+.+.+.+.+.+.+.+.+.+.+.+.+.+.+.+.+.+
\skeyword{EDIT} 
\sprompt{(Edit: name [(offset)][/length][:format] , val [, ...])} 
\sdefault{= "":} 
\suser{link, 12} 
%
\svbegin \begin{verbatim} Edit of field LINK not allowed 
 LINK        0007ec60     00000098 
\end{verbatim} 
\svend 
\spend %.-.-.-.-.-.-.-.-.-.-.-.-.-.-.-.-.-.-.-.-.-.-.-.-.-.-.-.-.-.-.-.-.-
%

	The last example demonstrates that certain fields cannot be changed
because the integrity of the file depends on them.  This restriction may be
overridden by appending '{\bf ==}' to the field descriptor: 

\spbegin %.+.+.+.+.+.+.+.+.+.+.+.+.+.+.+.+.+.+.+.+.+.+.+.+.+.+.+.+.+.+.+.+
\svbegin \begin{verbatim} 
 *** Editing STH *** 
\end{verbatim}\svend 
%
\skeyword{EDIT} 
\sprompt{(Edit: name [(offset)][/length][:format] , val [, ...])} 
\sdefault{= "":} 
\suser{link(1)==, 1} 
%
\svbegin \begin{verbatim} 
 LINK        00000001
\end{verbatim} 
\svend 
\spend %.-.-.-.-.-.-.-.-.-.-.-.-.-.-.-.-.-.-.-.-.-.-.-.-.-.-.-.-.-.-.-.-.-

	The value given is interpreted with the default format descriptor
displayed by the \textref{name command}{.show.meta}).  A different format can
be used by giving a 
\textref{format specifier}{.show.in.format}. 

	For integers, the radix specifiers {\tt \%X} (hexadecimal), {\tt
\%D}(decimal) and {\tt \%O} (octal) can also be used.  Values given as

	\verb/hh:[[mm][:[ss][.ttt]]/ or \\ 
\indent \verb/dd.[mm].[ss][.ttt]]/ 

\noindent will be translated to degrees and saved in the appropriate F, 
R or D format; 



\section{ Secondary blocks and substructures} 
\label{.pointers} 

	The header blocks directly accessible through {\bf SHOW/EDIT} contain
pointers to other blocks.  When their value is listed, \verb/:P/ is appended,
meaning that the value is the address of a block on disk in the corresponding
format format. 

	Certain fields in a block are arrays not of single values but of
sub-blocks of some type; an example are the interferometer entries in the OHW
block.  When such a field is listed, \verb/:S/ is appended, meaning that this
is an array of subblocks. 

	All secondary-block and substructure types known to the system 
(i.e.  not only those associated with the current block) are displayed by the
{\bf ::} command: 

\spbegin %.+.+.+.+.+.+.+.+.+.+.+.+.+.+.+.+.+.+.+.+.+.+.+.+.+.+.+.+.+.+.+.+
\skeyword{EDIT} 
\sprompt{(Edit: name [(offset)][/length][:format] , val [, ...])} 
\sdefault{= "":} 
\suser{::} 
%
\svbegin \begin{verbatim} 
 Known P: types: STH, FDW, OHW, SCW, SHW, FDX, GFH, IFRT, MDH, MDD, SGH, MDL,
SCH 
	IFRC, B, I, J, E, D, X, Y, S:SET, S:SRC, S:BCOR, S:MOZP, S:IFR, IFH
\end{verbatim} 
\sinline{ {\rm :} indicates a sub-structure. The use of the data types {\em B,
I} etc. is explained \textref{elsewhere}{.data.arrays}. } 
\svend 
\spend %.-.-.-.-.-.-.-.-.-.-.-.-.-.-.-.-.-.-.-.-.-.-.-.-.-.-.-.-.-.-.-.-.-


\subsection{ Navigating secondary blocks} 

	Specifying the name of a pointer will steer the editing process to the
block pointed at: 

\svbegin \begin{verbatim} 
 *** Editing STH *** 
\end{verbatim} 
%
\spbegin %.+.+.+.+.+.+.+.+.+.+.+.+.+.+.+.+.+.+.+.+.+.+.+.+.+.+.+.+.+.+.+.+
\skeyword{EDIT} 
\sprompt{(Edit: name [(offset)][/length][:format] , val [, ...])} 
\sdefault{= "":} 
\suser{mdd} 
%
\svbegin \begin{verbatim} 
 *** Editing STH *** 
\end{verbatim} 
\sinline{ So nothing happened: There is no MDD block attached to this STH
block.} 
\svend 
\spend %.-.-.-.-.-.-.-.-.-.-.-.-.-.-.-.-.-.-.-.-.-.-.-.-.-.-.-.-.-.-.-.-.-
%
\spbegin %.+.+.+.+.+.+.+.+.+.+.+.+.+.+.+.+.+.+.+.+.+.+.+.+.+.+.+.+.+.+.+.+
\skeyword{EDIT} 
\sprompt{(Edit: name [(offset)][/length][:format] , val [, ...])} 
\sdefault{= "":} 
\suser{ifrp:} 
%
\svbegin \begin{verbatim} 
 Edit data: IFRP, XJ, , P:IFRT; 184, 1, 1, 4 
\end{verbatim} 
\svend 
\spend %.-.-.-.-.-.-.-.-.-.-.-.-.-.-.-.-.-.-.-.-.-.-.-.-.-.-.-.-.-.-.-.-.-
%
\spbegin %.+.+.+.+.+.+.+.+.+.+.+.+.+.+.+.+.+.+.+.+.+.+.+.+.+.+.+.+.+.+.+.+
\skeyword{EDIT} 
\sprompt{(Edit: name [(offset)][/length][:format] , val [, ...])} 
\sdefault{= "":} 
\suser{ifrp} 
%
\svbegin \begin{verbatim} 
 *** Editing IFRT *** 
\end{verbatim} 
\sinline{ This time we succeeded!} 
\svend 
\spend %.-.-.-.-.-.-.-.-.-.-.-.-.-.-.-.-.-.-.-.-.-.-.-.-.-.-.-.-.-.-.-.-.-
%
\spbegin %.+.+.+.+.+.+.+.+.+.+.+.+.+.+.+.+.+.+.+.+.+.+.+.+.+.+.+.+.+.+.+.+
\skeyword{EDIT} 
\sprompt{(Edit: name [(offset)][/length][:format] , val [, ...])} 
\sdefault{= "":} 
\suser{:} 
%
\svbegin \begin{verbatim} 
 Known names: 
	      IFR,       XI, , ;     0,    88,     0,     2
\end{verbatim} 
\sinline{The IFR block is just an array of hex integers} 
\svend 
\spend %.-.-.-.-.-.-.-.-.-.-.-.-.-.-.-.-.-.-.-.-.-.-.-.-.-.-.-.-.-.-.-.-.-
%
\spbegin %.+.+.+.+.+.+.+.+.+.+.+.+.+.+.+.+.+.+.+.+.+.+.+.+.+.+.+.+.+.+.+.+
\skeyword{EDIT} 
\sprompt{(Edit: name [(offset)][/length][:format] , val [, ...])} 
\sdefault{= "":} 
\suser{*} 
%
\svbegin \begin{verbatim} 
 IFR             0a09         0b0a         0d0c         0b09         0100 
		 0201         0302         0403         0504         0605
		   .
		   .

\end{verbatim} 
\svend 
\spend %.-.-.-.-.-.-.-.-.-.-.-.-.-.-.-.-.-.-.-.-.-.-.-.-.-.-.-.-.-.-.-.-.-
%
\spbegin %.+.+.+.+.+.+.+.+.+.+.+.+.+.+.+.+.+.+.+.+.+.+.+.+.+.+.+.+.+.+.+.+
\skeyword{EDIT} 
\sprompt{(Edit: name [(offset)][/length][:format] , val [, ...])} 
\sdefault{= "":} 
\suser{\scr} 
%
\svbegin \begin{verbatim} 
 *** Editing STH *** 
\end{verbatim} 
\sinline{ A null reply returns us to the parent block} 
\svend 
\spend %.-.-.-.-.-.-.-.-.-.-.-.-.-.-.-.-.-.-.-.-.-.-.-.-.-.-.-.-.-.-.-.-.-
%


\subsection{ Displaying/editing a substructure} 

	An array of substructures can be shown in its entirety by: 

	\verb/<name>/* 

\noindent Example: 

%
\spbegin %.+.+.+.+.+.+.+.+.+.+.+.+.+.+.+.+.+.+.+.+.+.+.+.+.+.+.+.+.+.+.+.+
\svbegin \begin{verbatim} 
 *** Editing SHW *** 
\end{verbatim} 
%
 \skeyword{EDIT} 
\sprompt{(Edit: name [(offset)][/length][:format] , val [, ...])} 
\sdefault{= "":} 
\suser{ifr/*} 
%
\svbegin \begin{verbatim} 
 INFNR          16667      WTEL               0      OTEL              13
 RBAS            2736      NIH             2040

 INFNR          16665      WTEL               0      OTEL              12
 RBAS            2664      NIH             2047
		 .
		 .

 *** Editing SHW *** 
\end{verbatim}\svend 
\sinline{ Note that the current block is still \verb/SHW/.} 
\spend %.-.-.-.-.-.-.-.-.-.-.-.-.-.-.-.-.-.-.-.-.-.-.-.-.-.-.-.-.-.-.-.-.-


\noindent A single sub-structure can be shown and edited by: 

	\verb/<name>[(<index>)]/ 

\noindent Example: 

\spbegin %.+.+.+.+.+.+.+.+.+.+.+.+.+.+.+.+.+.+.+.+.+.+.+.+.+.+.+.+.+.+.+.+
\svbegin \begin{verbatim} 
 *** Editing SHW *** 
\end{verbatim} 
%
\skeyword{EDIT} 
\sprompt{(Edit: name [(offset)][/length][:format] , val [, ...])} 
\sdefault{= "":} 
\suser{ifr} 
%
\svbegin \begin{verbatim} 
 *** Editing S:IFR *** 
\end{verbatim}\svend 
\spend %.-.-.-.-.-.-.-.-.-.-.-.-.-.-.-.-.-.-.-.-.-.-.-.-.-.-.-.-.-.-.-.-.-
%
\spbegin %.+.+.+.+.+.+.+.+.+.+.+.+.+.+.+.+.+.+.+.+.+.+.+.+.+.+.+.+.+.+.+.+
\skeyword{EDIT} 
\sprompt{(Edit: name [(offset)][/length][:format] , val [, ...])} 
\sdefault{= "":} 
\suser{*} 
%
\svbegin \begin{verbatim} 
 INFNR          16667      WTEL               0      OTEL              13
 RBAS            2736      NIH             9120
\end{verbatim} 
\svend 
\spend %.-.-.-.-.-.-.-.-.-.-.-.-.-.-.-.-.-.-.-.-.-.-.-.-.-.-.-.-.-.-.-.-.-
%
\spbegin %.+.+.+.+.+.+.+.+.+.+.+.+.+.+.+.+.+.+.+.+.+.+.+.+.+.+.+.+.+.+.+.+
\skeyword{EDIT} 
\sprompt{(Edit: name [(offset)][/length][:format] , val [, ...])} 
\sdefault{= "":} 
\suser{\scr} 
%
\svbegin \begin{verbatim} 
 *** Editing SHW *** 
\end{verbatim} 
\svend 
\spend %.-.-.-.-.-.-.-.-.-.-.-.-.-.-.-.-.-.-.-.-.-.-.-.-.-.-.-.-.-.-.-.-.-
%
\spbegin %.+.+.+.+.+.+.+.+.+.+.+.+.+.+.+.+.+.+.+.+.+.+.+.+.+.+.+.+.+.+.+.+
\skeyword{EDIT} 
\sprompt{(Edit: name [(offset)][/length][:format] , val [, ...])} 
\sdefault{= "":} 
\suser{ifr(5)} 
%
\svbegin \begin{verbatim} 
 *** Editing S:IFR *** 
\end{verbatim} 
\svend 
\spend %.-.-.-.-.-.-.-.-.-.-.-.-.-.-.-.-.-.-.-.-.-.-.-.-.-.-.-.-.-.-.-.-.-
%
\spbegin %.+.+.+.+.+.+.+.+.+.+.+.+.+.+.+.+.+.+.+.+.+.+.+.+.+.+.+.+.+.+.+.+
\skeyword{EDIT} 
\sprompt{(Edit: name [(offset)][/length][:format] , val [, ...])} 
\sdefault{= "":} 
\suser{*} 
%
\svbegin \begin{verbatim} 
 INFNR          17689      WTEL               2      OTEL              12
 RBAS            2376      NIH             9465
\end{verbatim} 
\svend 
\spend %.-.-.-.-.-.-.-.-.-.-.-.-.-.-.-.-.-.-.-.-.-.-.-.-.-.-.-.-.-.-.-.-.-
%
\spbegin %.+.+.+.+.+.+.+.+.+.+.+.+.+.+.+.+.+.+.+.+.+.+.+.+.+.+.+.+.+.+.+.+
\skeyword{EDIT} 
\sprompt{(Edit: name [(offset)][/length][:format] , val [, ...])} 
\sdefault{= "":} 
\suser{\scr} 
%
\svbegin \begin{verbatim} 
 *** Editing SHW *** 
\end{verbatim} 
\svend 
\spend %.-.-.-.-.-.-.-.-.-.-.-.-.-.-.-.-.-.-.-.-.-.-.-.-.-.-.-.-.-.-.-.-.-


\section{ Advanced options} 

	The options described here are available for system programmers. 
They assume a proper understanding of the architecture of Newstar data files
and of the mechanisms (.dsc and .def files) through which it is defined.  No
attempt will be made here to help the reader on these points... 

	The syntax of the advanced options is less intuitively clear than what
has been described above and its expressive power is limited. 
In using it, the user will have to feel his way around, but he can safely do so
as long as he doesnot rashly attempt to change any values. 


\subsection{ Linked lists} 

	Link pointers in linked lists are similar to other pointers, except
that the type of the target block is not known a priori.  It must therefore be
specified: 

	\verb+<name>[(<index>)]::<target block type>+ 

\noindent For example: 

\spbegin %.+.+.+.+.+.+.+.+.+.+.+.+.+.+.+.+.+.+.+.+.+.+.+.+.+.+.+.+.+.+.+.+
\svbegin \begin{verbatim} 
 *** Editing STH *** 
\end{verbatim}\svend 
%
\skeyword{EDIT} 
\sprompt{(Edit: name [(offset)][/length][:format] , val [, ...])} 
\sdefault{= "":} 
\suser{link::gfh} 
%
\svbegin \begin{verbatim} 
 *** Editing GFH *** 
\end{verbatim} 
\svend 
\spend %.-.-.-.-.-.-.-.-.-.-.-.-.-.-.-.-.-.-.-.-.-.-.-.-.-.-.-.-.-.-.-.-.-
%
\spbegin %.+.+.+.+.+.+.+.+.+.+.+.+.+.+.+.+.+.+.+.+.+.+.+.+.+.+.+.+.+.+.+.+
\skeyword{EDIT} 
\sprompt{(Edit: name [(offset)][/length][:format] , val [, ...])} 
\sdefault{= "":} 
\suser{*} 
%
\svbegin \begin{verbatim} 
 ID              .SCN      LEN              512      VER                1
 CDAT     21-Sep-1993      CTIM           17:23      RDAT     13-Jun-1994 
 RTIM           16:16      RCNT              50      NAME           A271A
 DATTP              7
 LINK        00006550     00500e08 
 ALHD        00006550     00500e08 
 NLINK              4      ALLEN              4
 LINKG       00000200     008069c8 :P 
 LHD         00000200     008069c8 :P 
 NLINKG             2      LLEN               2      IDMDL              0
 ID1                0      ID2                0      USER               0

\end{verbatim} 
\svend 
\spend %.-.-.-.-.-.-.-.-.-.-.-.-.-.-.-.-.-.-.-.-.-.-.-.-.-.-.-.-.-.-.-.-.-
%
\spbegin %.+.+.+.+.+.+.+.+.+.+.+.+.+.+.+.+.+.+.+.+.+.+.+.+.+.+.+.+.+.+.+.+
\skeyword{EDIT} 
\sprompt{(Edit: name [(offset)][/length][:format] , val [, ...])} 
\sdefault{= "":} 
\suser{\scr} 
\sinline{ Return to STH} 
%
\svbegin \begin{verbatim} 
 *** Editing STH *** 
\end{verbatim} 
\svend 
\spend %.-.-.-.-.-.-.-.-.-.-.-.-.-.-.-.-.-.-.-.-.-.-.-.-.-.-.-.-.-.-.-.-.-


\subsection{ Data arrays} 
\label{.data.arrays} 

	Some pointers point to "naked" arrays of data structures.  An example
is MDD in the STH which points to the array of complex model visibilities: 

\spbegin %.+.+.+.+.+.+.+.+.+.+.+.+.+.+.+.+.+.+.+.+.+.+.+.+.+.+.+.+.+.+.+.+
\svbegin \begin{verbatim} 
 *** Editing STH *** 
\end{verbatim}\svend 
%
\skeyword{EDIT} 
\sprompt{(Edit: name [(offset)][/length][:format] , val [, ...])} 
\sdefault{= "":} 
\suser{mdd::z} 
\sinline{ We try to display the array in complex format ...} 
%
\svbegin \begin{verbatim} 
 MDD         006179c8     00000000 :P 
\end{verbatim} 
\sinline{ but the system does not understand us!} 
\svend 
\spend %.-.-.-.-.-.-.-.-.-.-.-.-.-.-.-.-.-.-.-.-.-.-.-.-.-.-.-.-.-.-.-.-.-
%
\spbegin %.+.+.+.+.+.+.+.+.+.+.+.+.+.+.+.+.+.+.+.+.+.+.+.+.+.+.+.+.+.+.+.+
\skeyword{EDIT} 
\sprompt{(Edit: name [(offset)][/length][:format] , val [, ...])} 
\sdefault{= "":} 
\suser{mdd} 
%
\svbegin \begin{verbatim} 
 *** Editing MDD *** 
\end{verbatim} 
\svend 
\spend %.-.-.-.-.-.-.-.-.-.-.-.-.-.-.-.-.-.-.-.-.-.-.-.-.-.-.-.-.-.-.-.-.-
%
\spbegin %.+.+.+.+.+.+.+.+.+.+.+.+.+.+.+.+.+.+.+.+.+.+.+.+.+.+.+.+.+.+.+.+
\skeyword{EDIT} 
\sprompt{(Edit: name [(offset)][/length][:format] , val [, ...])} 
\sdefault{= "":} 
\suser{*} 
%
\svbegin \begin{verbatim} 
 MDD                -540.55+72.69I                0.00+0.00I
			0.00+0.00I                0.00+0.00I
		    -540.50+72.76I                0.00+0.00I
			0.00+0.00I                0.00+0.00I
			.
			.

\end{verbatim} 
\sinline{ This works, but it is not clear how we could select specific elements
for display.} 
\svend 
\spend %.-.-.-.-.-.-.-.-.-.-.-.-.-.-.-.-.-.-.-.-.-.-.-.-.-.-.-.-.-.-.-.-.-


	There are also data arrays whose address is defined implicitly by the
fact that they contiguously follow a header block.  An example is the
scan-visibilities block headed by the SCN header.  To access such data in the
absence of a pointer, one must use a 
\textref{byte offset}{.abs.rel.address}. 
For this we need the length of the header, which we can easily find.  The
technique is demonstrated below is a somewhat different context, {\it viz.}
that of finding an SCH which is not directly pointed at. 

\spbegin %.+.+.+.+.+.+.+.+.+.+.+.+.+.+.+.+.+.+.+.+.+.+.+.+.+.+.+.+.+.+.+.+
\svbegin \begin{verbatim} 
 *** Editing STH *** 
\end{verbatim}\svend 
%
\skeyword{EDIT} 
\sprompt{(Edit: name [(offset)][/length][:format] , val [, ...])} 
\sdefault{= "":} 
\suser{scnl} 
%
\svbegin \begin{verbatim} 
 SCNL            1552
\end{verbatim} 
\sinline{\\ This is the length of an SCH plus the data.} 
\svend 
\spend %.-.-.-.-.-.-.-.-.-.-.-.-.-.-.-.-.-.-.-.-.-.-.-.-.-.-.-.-.-.-.-.-.-
%
\spbegin %.+.+.+.+.+.+.+.+.+.+.+.+.+.+.+.+.+.+.+.+.+.+.+.+.+.+.+.+.+.+.+.+
\skeyword{EDIT} 
\sprompt{(Edit: name [(offset)][/length][:format] , val [, ...])} 
\sdefault{= "":} 
\suser{scnp} 
%
\svbegin \begin{verbatim} 
 *** Editing SCH *** 
\end{verbatim} 
\svend 
\spend %.-.-.-.-.-.-.-.-.-.-.-.-.-.-.-.-.-.-.-.-.-.-.-.-.-.-.-.-.-.-.-.-.-
%
\spbegin %.+.+.+.+.+.+.+.+.+.+.+.+.+.+.+.+.+.+.+.+.+.+.+.+.+.+.+.+.+.+.+.+
\skeyword{EDIT} 
\sprompt{(Edit: name [(offset)][/length][:format] , val [, ...])} 
\sdefault{= "":} 
\suser{*} 
%
\svbegin \begin{verbatim} 
 HA       -90.1250124 deg  MAX         1294.000 W.U. SCAL        0.000000 
 REDNS          0.000        0.000        0.000        0.000 W.U.
		.
		.

\end{verbatim} 
\svend 
\spend %.-.-.-.-.-.-.-.-.-.-.-.-.-.-.-.-.-.-.-.-.-.-.-.-.-.-.-.-.-.-.-.-.-
%
\spbegin %.+.+.+.+.+.+.+.+.+.+.+.+.+.+.+.+.+.+.+.+.+.+.+.+.+.+.+.+.+.+.+.+
\skeyword{EDIT} 
\sprompt{(Edit: name [(offset)][/length][:format] , val [, ...])} 
\sdefault{= "":} 
\suser{.1552::sch} 
\sinline{ Move to the next SCH. We can use multiples of SCNL to get to any of
the SCHs.} 
%
\svbegin \begin{verbatim} 
 *** Editing SCH *** 
\end{verbatim}\svend 
%
\skeyword{EDIT} 
\sprompt{(Edit: name [(offset)][/length][:format] , val [, ...])} 
\sdefault{= "":} 
\suser{*} 
%
\svbegin \begin{verbatim} 
 HA       -89.8743278 deg  MAX         1349.000 W.U. SCAL        0.000000 
 REDNS          0.000        0.000        0.000        0.000 W.U.
 ALGNS          0.000        0.000        0.000        0.000 W.U.
		.
		.

\end{verbatim} 
\svend 
\spend %.-.-.-.-.-.-.-.-.-.-.-.-.-.-.-.-.-.-.-.-.-.-.-.-.-.-.-.-.-.-.-.-.-



\subsection{ Absolute and relative file addresses} 
\label{.abs.rel.address} 

	Absolute {\em file addresses} (i.e.  byte offsets in the file) can be
used instead of field names. 

	Relative addresses are byte offset relative to the start of the current
block.  They are specified in the form 

	\verb/.<offset>/ 

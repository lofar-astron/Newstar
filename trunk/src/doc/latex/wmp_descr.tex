%
% @(#) wmp_descr.tex  v1.2 04/08/93 JEN
%       JPH 940520      modernised
% 	HjV 950619	Correct some typo's
%
%
\chapter{The WMP-file (image data)} 
\tableofcontents 

\section{Organisation of the WMP file: Maps} 
\label{.organ} 

	The \NEWSTAR WMP file contains `image data', i.e.  a collection of
2-dimensional arrays of data.  The various `images' in a WMP file {\it are
related in some way}, but do not have to have the same dimensions.  Examples
are radio maps at various frequencies (line observations), polarisations, or
pointing directions (mosaicking). 
There may also be antenna patterns and various kinds of residual maps, or even
rectangular arrays ov uv-data. 

%\include{fig_wmp_structure}            % figure

The basic unit in the WMP file is the Map (a 2-dimensional array of pixel
values). It can be selected by the user by means of 6 integer indices: 
\\ \\
\begin{tabular}{lll} 
1)      &group          &spare index \\ 
2)      &field          &pointing centre (mosaicing) \\ 
3)      &channel        &frequency channel, or DCB band \\ 
4)      &polarisation   &0=I or XX, 1=Q or XY, 2=U or YX, 3=V or YY \\ 
5)      &type           &0=map, 1=ap, 2=cov, 3=real, 4=imag, 5=ampl, 6=phase \\ 
6)      &nseq           &sequence number within each type \\ 
\end{tabular} 
\\ \\
Indices may also be ranges of indices, or wildcards ($\ast$), as explained in
more detail in the section `Overview of \NEWSTAR files' in this Cookbook. 
Note that the Map in the WMP-file plays the same role as the Set in the 
SCN-file. 

	All indices are just running numbers ({\bf starting at 0!}), except
`type' and `polarisation', which have fixed codes (see above). 

	Neither the `field' nor the `channel' index nrs correspond with the
`field' or `channel' nrs in the SCN-file.  The reason for this is that a map
may be made from a {\em combination} of fields or channels. 

	Usually, all Maps belong to the same `group' (0).  Therefore, the first
index is called a `spare index' here.  However, any selection of Maps may be
put into a new group in the same WMP file, using the NMAP option REGROUP. 

	The 6th index allows for a sequence of Maps of a certain type, usually
derived from each other.  Examples are residual Maps after 
\textref{CLEANing}{nclean_descr}, or the Maps that result from 
\textref{combining other Maps}{nmap_descr.fiddle}. 


\section{File layout and file header} 
\label{.file.layout} 

	A summary of the WMP-file contents and layout may be obtained by using
the program NMAP, option SHOW: 


****** Put new script here ****** 


This particular WMP-file actually contains the following (rather strange)
collection of Maps: 
\sline{g.f.c.p.t.n (\#abs)} 
   \sinline{group.field.chan.pol.type.nseq} 
\sline{0.0.0.0.2.0(\#0) type COVE } 
   \sinline{uv-coverage for XY-map} 
\sline{0.0.0.0.6.0(\#1) type PHAS } 
   \sinline{map of XY phases} 
\sline{0.0.0.0.0.0(\#2) type MAP } 
   \sinline{XY-map} 
\sline{0.0.0.0.1.0(\#3) type AP } 
   \sinline{antenna pattern for XY-map} 
\sline{0.0.0.1.6.0(\#4) type PHAS } 
   \sinline{map of Q phases} 
\sline{0.0.0.1.0.0(\#5) type MAP } 
   \sinline{Q-map} 
\sline{0.0.0.2.6.0(\#6) type PHAS } 
   \sinline{map of V phases} 
\sline{0.0.0.2.0.0(\#7) type MAP } 
   \sinline{V-map} 
\sline{0.0.0.3.6.0(\#8) type PHAS } 
   \sinline{map of iV phases} 
\sline{0.0.0.3.0.0(\#9) type MAP } 
   \sinline{iV-map} 


The 10 `datasets' (Maps) in this WMP-file belong to 1 `group' (1st index, =0). 
The number in parentheses indicates the {\em absolute Map nr} within the file. 

Note that the number of polarisations is not really 4, and that the number of
types is not really 7.  Indicated are the highest index values present, plus
one. 

The WMP file header only contains book-keeping information that allows the
program to find its way around: 


****** Put new script here ***** 


\section{The Map header} 
\label{.map.header} 

Each Map in a WMP file contains a header with information, which can be
inspected with the program \textref{NMAP, option SHOW}{nmap_descr.show}: 


****** Put new script here ***** 


%------------------------------------------------------------------------------

\subsection{Explanation of items in the Map header} 
\label{.header.items} 

\begin{itemize} 
\item  LINK: Link (pointer) to other Maps 
\item  LEN: Length of header block (bytes?) 
\item  VER: Version nr of the header 
\item  SETN: Abs Map (Set) nr, i.e. the one needed for direct reference 
	(e.g. \#setn) 
\item  FNM: Field (pointing centre) name 
\item  EPO: Epoch (e.g. 1950.0) 
\item  RA: RA of field centre (degr) 
\item  DEC: DEC of field centre (degr) 
\item  FRQ: Central frequency (MHz) 
\item  BDW: Bandwidth (MHz) 
\item  RAO, DECO, FRQO: Observed RA (degr), DEC (degr), freq (MHz) 
\item  ODY, OYR: Observed day (since January 0th) and year (since 1900) 
\item  DCD: Data code (2=I, 4=J, 5=E, 8=D) 
\item  PCD: Program code (0=NMAP) 
\item  SRA, SDEC, SFRQ: Separation in RA (degr), DEC (degr) and freq (MHz) 
\item  NRA, NDEC, NFRQ: Nr of points in RA, DEC and frequ 
\item  ZRA, ZDEC, ZFRQ: Centre RA (1st point=0), DEC (1st line=0), frequ (1st
map=0) 
\item  MXR, MXD, MXF: Position max in RA, DEC, frequ 
\item  MNR, MND, MNF: Position min in RA, DEC, frequ 
\item  MAX, MIN: Max, min map value 
\item  SHR, SHD, SHF: Shift in RA, DEC (add, degr?) or frequ (add, MHz) 
\item  SUM: Normalising sum 
\item  UNI: Multiplier to get Jy 
\item  UCM: User comment 
\item  NPT: Nr of input uv-data points 
\item  TYP: Map type (MAP, AP, COV, PHAS etc) 
\item  POL: Polarisation type (I,Q,U,V or XX,XY,YX,YY) 
\item  CD: Codes (array of 8 integer switches, 0-7): 
	taper type (0), convolution type (1), 
	correct for convolution (2), clipping done (3), 
	source subtraction (4), data type (5), 
	uv coordinate type (6), de-beam count (7) 
\item  EPT: Map epoch used (0=apparent, 1=as specified in EPO above) 
\item  OEP: Observation epoch (e.g. 1985.78) 
\item  NOS: Map rms noise (W.U.) 
\item  FRA, FDEC, FFRQ: Field size in RA (degr), DEC (degr), frequ (MHz) 
\item  TEL: Telescope name (e.g. WSRT) 
\item  FSR, FSD: FFT size RA, DEC 
\item  MDP: Map data pointer 
\item  NBL: Nr of baselines that have contributed to the Map 
\item  NST: Nr of uv-data sets that have contributed to the Map 
\item  VEL: Velocity (m/s) 
\item  VELC: Velocity code (0=continuum, 1=heliocentric radio, 2=LSR radio, 
	3=heliocentric optical, 4=LSR optical) 
\item  VELR: Velocity at reference frequ (FRQC) 
\item  INST: Instrument code (0=WSRT, 1=ATCA) 
\item  FRQ0, FRQV, FRQC: Rest, Real and Centre frequency for line (MHz) 
%
\end{itemize} 



\section{The Map data} 
\label{.data} 

	The actual data in a Map can be displayed on the X-screen as a color
map by using the program NGIDS, or as a contour or gray-scale plot with the
program NPLOT.  However, its is also possible to inspect small areas of a WMP
Map, or its statistics, with the program NMAP, option 
SHOW:


****** Put new script here ***** 


Note that the `noise' option gives the rms of the pixel values in the selected
area(s), while the `offset' option gives the rms with respect to their average
value (offset). A histogram of pixel values is printed in the log-file
(NMAP.LOG). 


%       JPH 940916      Make compilable

\chapter{ WSRT fact sheet} 


\section{Array} 

	The WSRT array is shown in the figure \figref{.wsrt.layout}.  The
regular spacings of the 1-dimensional array are designed to minimise the
side-lobe level in the un-CLEANed map (the WSRT array was designed in the early
sixties, before the invention of the CLEAN method for deconvolution in 1972).
The `waste' of uv-coverage is compensated somewhat by using the redundant
spacings for model-independent calibration.  The very regular WSRT `beam'
enhances the map-reliability: its residual structure can be readily traced back
to the source of the problem, and it cannot easily be mistaken for real
structure in the map. 

	The 1-dimensional East-West orientation of the WSRT array is
advantageous for wide-field mapping and deconvolution, because the instrumental
`point-spread function' (or `beam') with which the image is convolved, is
constant over the entire observed field.  Disadvantages are the poor
North-South resolution for low-declination sources, and the impossibility of
making 2D `snap-shot' images. 

\input{../fig/wsrt_layout.cap} 


	The redundant spacings in the regular WSRT array can be used for
model-independent internal calibration.  The visibilities measured by two
redundant interferometers (i.e.  with the same baseline length and orientation)
must obviously be identical.  Thus, any differences between them must be caused
by instrumental errors, including the effects of the Earth troposhere and
ionosphere. 

	Maximum redundancy occurs if 9A=72m in the standard configuration
\figref{.wsrt.layout}: baselines 9A=AB=CD=72m, and 09=1B=AC=BD=1296m.  In this
case, there is a `full redundancy solution' for both phase and gain, linking
the errors of all 14 telescopes.  For other values of the distance 9A, there
will still be a full solution for the telescope gain errors, but the phase
solution will usually be split up into three independent groups: the fixed
array (0-9), A/C and B/D. 



\section{Telescopes} 


\section{Sensitivity} 
\label{.sensitivity} 

The following tables gives WSRT interferometer system temperatures and
sensitivities. The theoretical rms continuum sensitivity for 12 hrs observation
is based on the given $T_{sys}$ and maximum bandwidth. 

\begin{center} 
\begin{tabular}{|c|r@{--}l|c|c|c|} 
\hline 
\multicolumn{1}{|c|}{$\lambda$} & 
\multicolumn{2}{|c|}{Frequ} & 
\multicolumn{1}{|c|}{$T_{sys}$} & 
\multicolumn{1}{|c|}{Max BW} & 
\multicolumn{1}{|c|}{Sensitivity} \\    % end of first header line
\multicolumn{1}{|c|}{cm}  & 
\multicolumn{2}{|c|}{MHz} & 
\multicolumn{1}{|c|}{K}  & 
\multicolumn{1}{|c|}{MHz} & 
\multicolumn{1}{|c|}{mJy} \\            % end of second header line
\hline 92 &  320 & 330  & 130 &  5   & 0.5 \\ 49 &  607 & 610  & 110 &  2.5 &
0.6 \\ 21 & 1365 & 1425 &  60 & 60   & 0.06 \\ 18 & 1590 & 1730 &  60 & 80   &
0.15 \\ 
 6 & 4770 & 5020 & 110 & 80 & 0.07 \\ \hline 
\end{tabular} 
\end{center} 




\section{Backend} 

	The WSRT digital backends contribute only very small `closure 
errors' (interferometer-dependent errors), of the order of $0.01\%$.
For comparison, the VLA has closure errors of up to a few percent in continuum
mode, mainly due to the wideband quadrature networks).  The problem is much
less in the VLA line mode. 

	Closure errors violate the basic assumption of techniques like SELFCAL
and Redundant Spacing Calibration, i.e.  that all errors are telescope-based.
Closure errors can be calibrated if they are more or less constant over the
length of the observation, but it is obviously better if they are small. 

